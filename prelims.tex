\section{Preliminaries}
\label{sec:prelims}
\subsection{Basic Classical Cryptographic Notions}
\label{sec:classical-prelims}
Let $\mathbb{N}$ be the set of positive integers. For $n \in \mathbb{N}$, we set $[n] = \{1, \cdots, n\}.$ We denote the set of all  binary strings of length $n$ by $\bit{n}.$
 An element $s \in \bit{n}$ is called a bitstring, and $|s|=n$ denotes its length. We denote an arbitrary polynomial from the set $\mathbb{N}$ to $\mathbb{N}$ by $poly(\cdot).$


A function ${\rm negl}:\mathbb{N}\rightarrow\mathbb{R}^{+}\cup \{0\}$ is \emph{negligible} if for every positive polynomial $p(n)$ there exists a positive integer $n_0$ such that  for all  $n>n_0,$ ${\rm negl}(n) < 1/ p(n).$   A typical use of negligible functions is to indicate that the probability of success of some algorithm is too small to be amplified to a constant by a feasible (\emph{i.e.}, polynomial) number of repetitions. Given two bit strings $x$ and $y$ of equal length, we denote their bitwise XOR by $x \oplus y.$

\subsubsection{Classical Circuits and Algorithms}
 For $n, m\in\mathbb{N}$ let  $f:\bit{n}\rightarrow \bit{m}$ be a function. We say a circuit $C$ computes $f$ if for every $s \in \bit{n},\; C(s)=f(s).$  We define the size of a circuit $C$ as the number of gates in it and is denoted by $|C|.$ A set of gates for classical computation is universal if, for all $n, m\in\mathbb{N}$, and for every function $f:\bit{n}\rightarrow \bit{m}$ a circuit can be constructed for computing~$f$ using only gates from that set. It is a well-known fact that \{AND, OR, NOT\} is a set of universal gates for classical circuits. A family of circuits $\mathcal{F}=\{C_n\mid n\in\mathbb{N}\},$ one for each input  size $n\in \mathbb{N}$ is called uniform if there exists a deterministic Turing machine $M,$ such that
 \begin{itemize}
 \item For each $n\in\mathbb{N},$ $M$ outputs a description of $C_n \in \mathcal{F}$ on input $1^n.$
 \item  For each $n\in\mathbb{N},$ $M$ runs in time $poly(n).$
 \end{itemize}

\Anne{We need to give references where this definition comes from.}

\begin{definition}\label{def:qiO} {\rm({\bf Quantum Secure Indistinguishability Obfuscation} $i\mathcal{O}$)}
A probabilistic polynomial-time algorithm is a \emph{quantum-secure computational indistinguishability obfuscator} $i\mathcal{O}$ for a class of circuits ${\mathcal C},$ if the following conditions hold:

\begin{enumerate}
\item {\tt Functionality:} For any circuit $C\in {\mathcal C},$ and for all inputs $x$ $$i\mathcal{O}(C)(x)=C(x).$$
\item  {\tt Polynomial Slowdown:}  For every $C\in \mathcal{C},$  $|i\mathcal{O}(C)| \in poly(|C|).$
\item {\tt Indistinguishability:} For any two circuits $C_0,C_1\in {\mathcal C},$ of the same size  that compute the same function
 and for every polynomial time quantum distinguisher $\mathcal{D}_q,$  there exists a negligible function {\rm negl} such that:

					$$\Big | {\rm Pr}[\mathcal{D}_q(i\mathcal{O}(C_0))=1]-{\rm Pr}[\mathcal{D}_q(i\mathcal{O}(C_1))=1] \Big |\leq  {\rm negl}(|C_0|).$$			
\end{enumerate}										
\end{definition}



\subsection{Basic Quantum Notions}
\label{sec:quantum-prelims}
%%%%%%%%%%%%%%%%%%%%%%%%%%
Given an $n$-bit string $x$, the corresponding  $n$-qubit quantum basis state is denoted~$\ket{x}$. The $2^n$-dimensional Hilbert space spanned by $n$-qubit basis states is denoted:
%\anote{in general, we will try to include equation numbers unless there is a reason not to. This will help in reviewing and referencing our work.}
\begin{equation}
\label{eq:hilbert-space}
\mathcal{H}_n := \textbf{span} \left\{ \ket{x} : x \in \bit{n} \right\}\,.
\end{equation}
We denote by $\mathcal{D}(\mathcal{H}_n)$ the set of density operators (\emph{i.e.}, valid quantum states) on~$\mathcal{H}_n$. These are linear operators on $\mathcal{D}(\mathcal{H}_n)$ which are positive-semidefinite and have trace equal to $1.$


%\anote{In general, there is some confusion throughout related to pure and mixed states. This preliminaries is good, but need to make sure that it matches the level of formalism in the main body.}
%. When considering different physical subsystems, we denote them with uppercase Latin letters; when a Hilbert space corresponds to a subsystem, we place the subsystem label in the subscript. For instance, if $F \cup G \cup H = [n]$ then $\mathcal{H}_n = \mathcal{H}_F \otimes \mathcal{H}_G \otimes \mathcal{H}_H.$ Sometimes we  write explicitly the subsystems a state belongs to as subscripts; this will be useful when considering, \emph{e.g.}, the reduced state on some of the subspaces. For example, wesometimes express the statement $\rho \in \mathcal{D}(\mathcal{H}_F \otimes \mathcal{H}_G \otimes \mathcal{H}_H)$ simply by calling the state $\rho_{FGH}$; in that case, the state obtained by tracing out the subsystem~$H$ will be denoted~$\rho_{FG}$.


%Given $\rho, \sigma \in \mathcal{D}(\mathcal{H})$, the trace distance between $\rho$ and $\sigma$ is given by half the trace norm $\|\rho - \sigma\|_1$ of their difference. When $\rho$ and $\sigma$ are classical probability distributions, the trace distance reduces to the total variation distance. Physically realizable maps from a state space $\mathcal{D}(\mathcal{H})$ to another state space $\mathcal{D}(\mathcal{H}')$ are called \emph{admissible}---these are the completely positive trace-preserving (CPTP) maps. For the purpose of distinguishability via input/output operations, the appropriate norm for CPTP maps is the diamond norm, denoted $\|\cdot\|_\diamond$. The set of admissible maps coincides with the set of all maps realizable by composing (i.) addition of ancillas, (ii.) unitary evolutions, (iii.) measurements in the computational basis, and (iv.) tracing out subspaces. We remark that unitaries $U \in U(\mathcal{H}_n)$ act on $\mathcal{D}(\mathcal{H}_n)$ by conjugation: $\rho \mapsto U \rho U^\dagger$. The identity operator~$\mathds{1}_n \in U(\mathcal{H}_n)$ is thus both a valid map, and (when normalized by $2^{-n}$) a valid state in $\mathcal{D}(\mathcal{H}_n)$---corresponding to the classical uniform distribution.


\subsubsection{Quantum Gates}
We will work with the following set of unitary gates

$$\xgate = \left[\begin{array}{cc} 0 & 1\\ 1 & 0\end{array}\right],
\quad\zgate = \left[\begin{array}{cc} 1 & 0\\ 0 & -1\end{array}\right],
\quad\pgate = \left[\begin{array}{cc} 1 & 0\\ 0 & i\end{array}\right],
 \quad\hgate = \frac{1}{\sqrt{2}}\left[\begin{array}{cc}1 & 1\\1 & -1\end{array}\right], $$
$$\quad\cnot = \left[\begin{array}{cccc} 1 & 0 & 0 & 0\\ 0 & 1 & 0 & 0\\ 0 & 0 & 0 & 1\\ 0 & 0 & 1 & 0\end{array}\right], \text{and} 
\quad\tgate = \left[\begin{array}{cc} 1 & 0\\ 0 & e^{i\pi/4}\end{array}\right].$$

\subsubsection*{Quantum One Time Pad}
For any single-qubit density operator $\rho \in \mathcal{D} (\mathcal{H}_1)$ we encrypt it by sampling at uniform bits $s$ and $t$, and producing 
 $\xgate^s \zgate^t \rho \zgate^t \xgate^s$. To an observer that has no knowledge of $s$ and $t$, this system is information-theoretically indistinguishable from the state $\mathds{1}_1/2$ \cite{AMTW00}.
 \Anne{Did we define $\mathds{1}$? }

%The above map can be straightforwardly extended to the $n$-qubit case in order to obtain an elementary {\em quantum encryption scheme} called the {\em quantum one-time pad}\cite{AMTW00}.
%We first set $\xgate_j = \mathds{1}^{\otimes j-1} \otimes \xgate \otimes \mathds{1}^{\otimes n-j}$ and likewise for $\ygate_j$ and $\zgate_j$. We define the $n$-qubit Pauli group $\mathcal P_n$ to be the subgroup of $\operatorname{SU}(\mathcal{H}_n)$ generated by $\{\xgate_j, \ygate_j, \zgate_j : j = 1, \dots, n \}$. Note that Hermiticity is inherited from the single-qubit case, \emph{i.e.}, $P^\dag = P$ for every $P \in \mathcal{P}_n$.
%
%\anote{It is not clear to me that, above, the identity is part of $\mathcal{P}_n$. I think it should be.}
%\anote{$\xgate$ and $\zgate$ already defined. I would re-phrase, first defining the new gates, and then defining the Clifford group.}

\begin{definition}
\label{defn:Clifford+T:family}
 \noindent{\bf Clifford Group}: The set of gates $\{\xgate, \zgate, \pgate, \cnot,\hgate\}$ applied to arbitrary wires (redundantly) generates the Clifford group, where

\Anne{something missing here?}

We note the following relations between these gates:
$$\xgate\zgate = - \zgate\xgate,\quad \tgate^2=\pgate,\quad\pgate^2=\zgate,\quad\hgate\xgate\hgate=\zgate,\quad \tgate\pgate=\pgate\tgate,\quad\pgate\zgate=\zgate\pgate.$$
Also, for any $a,b\in\{0,1\}$ we have
$\hgate\xgate^b\zgate^a=\xgate^a\zgate^b \hgate$\,.
\end{definition}

\subsubsection{Quantum Circuits and Algorithms}
A quantum circuit is an acyclic network of quantum gates connected by wires. The quantum gates represent quantum operations and wires represent the qubits on which gates act. In, general a quantum circuits can have $n$-input qubits and $m$-output qubits for any integer $n,m\geq 0.$
%\anote{todo: define what is a quantum circuit, and also any type of notation that will be used in the rest of the paper. For instance, what does $\hgate-\cnot-\pgate-\cnot-\pgate-\cnot-\hgate-\pgate-\cnot-\pgate-\cnot$ mean? Do I apply gates from left-to-right, or from right-to-left?}

For $n\in \mathbb{N},$ the set of all $n\times n$ unitary matrices is denoted by $O(n,\mathbb{C})=\{U\in \mathbb{C}^{n\times n} \mid U\cdot U^\dagger={\bf I}\}.$ We say a quantum circuit $C_q$ computes $U\in O(n,\mathbb{C})$ if for every $\rho\in \mathcal{H}_n$
$$U(\rho )=C_q(\rho).$$ \Anne{Hmm, for density matrices, usually the action is more like $U\rho U^\dag$. Alternatively, if we work with pure states, then yes we can write $U \ket{\psi}$}
A quantum circuit that computes a unitary matrix is called a \emph{reversible  quantum circuit}, \emph{i.e.}, it always possible to uniquely recover the input, given the output. A set of gates are said to be universal if, for any a unitary matrix~$U$  a quantum circuit can be constructed for computing~$U$ using only gates from that set. It is a well-known fact that Clifford gates are not universal, but adding any non-Clifford gate, such as $\tgate$, gives a universal set of gates, where
$$\quad\tgate = \left[\begin{array}{cc} 1 & 0\\ 0 & e^{i\pi/4}\end{array}\right]\,.$$ \Anne{Hmm, for quantum circuits, we actually only have \emph{approximate} universality. We should probably mention this, and give a reference. }

A family of quantum circuits $C=\{C_n\in \mathbb{N} \mid \}$  \Anne{notation?} one for each input  size $n\in \mathbb{N}$ is called \emph{uniform} if there exists a deterministic Turing machine $M,$ such that
 \begin{itemize}
 \item For each $n\in\mathbb{N},$ $M$ outputs a description of $C_n \in \mathcal{F}$ on input $1^n.$
 \item  For each $n\in\mathbb{N},$ $M$ runs in $poly(n).$
 \end{itemize}

In general, quantum operations are not necessarily unitary (reversible), nevertheless a general (possibly irreversible) quantum operation also called a \emph{superoperator} can efficiently be simulated by a reversible quantum operations by adding auxiliary states to the original system, then performing a unitary operation on the joint system, and then tracing out %$Tr$ 
some subsystem \cite{KLM07}. More precisely, this can be described as the map:
%\anote{todo: find ref (I think Sebastien has some references in his thesis). Actually, do we use this?}
$$\rho_{in} \xmapsto{\mbox{superoperator}} \rho_{out}=Tr_B(U(\rho_{in} \otimes \ket{00\cdots 0} \bra{00\cdots 0})U^\dagger)$$\,m
where $\rho_{in} \in \mathcal{H}_n,$ is the original state and $\ket{00\cdots 0}$ is an auxiliary state of dimension at most $n^2.$ A circuit that computes a general quantum operation is called a general quantum circuit.  Therefore, general quantum circuits can refer to both reversible or irreversible circuits. A polynomial-time quantum algorithm is a uniform family of general quantum circuits. \Anne{Note to self: I am wondering where we use this in the paper.}

%\noindent {\em Remark}: From now we use the term  quantum circuits to refer to reversible quantum circuits only and the term quantum algorithm is reserved for some family of general quantum circuits. \anote{ok, well actually I do believe that we could define quantum circuits that also include auxiliary qubit preparation and measurements/trace outs, and that we can obfuscate these also using gate teleportation. Hence the distinction circuit/algorithm is not very natural. Need to think about this.}



\subsection{Gate Teleportation}
\label{protocol: gate-teleportation}
%We recall the \emph{Bell states}: %$\ket{\beta_{00}},$  $\ket{\beta_{01}},$ $\ket{\beta_{10}}$ and $\ket{\beta_{11}}.$
%$\ket{\beta_{00}}=\frac{1}{\sqrt2}\left(\ket{00}+\ket{11}\right)$, ${\beta_{01}}=\frac{1}{\sqrt2}\left(\ket{01}+\ket{10}\right)$,
%$\ket{\beta_{10}}=\frac{1}{\sqrt2}\left(\ket{00}-\ket{11}\right)$,  $\ket{\beta_{11}}=\frac{1}{\sqrt2}\left(\ket{01}-\ket{10}\right)$\,.
Suppose we want to evaluate a single qubit gate $U\in\{\xgate, \zgate, \pgate,\hgate\}$ on some qubit $\ket{\psi}.$ Then using gate teleportation  \cite{GC99} we can compute $U (\ket{\psi})$ as in \Cref{algo:gate-teleport}.
\begin{algorithm}[]
\caption{Gate Teleportation.}
\label{algo:gate-teleport}
\begin{enumerate}
\item  Prepare a $2$ qubit Bell state $\ket{\beta_{00}}=\frac{\ket{00}+\ket{11})}{\sqrt{2}}.$
\item Write the joint system as
 \begin{equation*}
 \label{eq:telepprtation1}
 \begin{aligned}
\ket{\psi}_C \ket{\beta_{00}}_{AB}=\frac{1}{2}\ket{\beta_{00}}_{AC} \ket{\psi}_{B} + \frac{1}{2}\ket{\beta_{01}}_{AC} ( \xgate( \ket{\psi})_{B}  +\\
 \frac{1}{2}\ket{\beta_{10}}_{AC} ( \zgate( \ket{\psi})_{B} + \frac{1}{2}\ket{\beta_{11}}_{AC} ( \xgate \zgate( \ket{\psi})_{B}.
\end{aligned}
\end{equation*}
 Where $\beta_{ij},$  denotes the 2-qubit Bell basis.
\item Apply the Clifford gate $U$ on the subsystem $B,$
 \begin{equation*}
  \label{eq:telepprtation2}
 \begin{aligned}
\mathbb{I}\otimes \mathbb{I} \otimes U(\ket{\psi}_A \ket{\beta_{00}}_{BC})=\frac{1}{2}\ket{\beta_{00}}_{AC} U(\ket{\psi})_{B}+ \frac{1}{2}\ket{\beta_{01}}_{AC}U( \xgate( \ket{\psi})_{B} +\\ \frac{1}{2}\ket{\beta_{10}}_{AC}U( \zgate( \ket{\psi})_{B}+ \frac{1}{2}\ket{\beta_{11}}_{AC}U( \xgate \zgate( \ket{\psi})_{B}.
\end{aligned}
 \end{equation*}
 \item Measure the subsystem AC  in the Bell basis and obtain the classical bits $(a,b).$ The system is now in the state
 \begin{equation}
  \label{eq:telepprtation3}
							\ket{\beta_{ab}}_{AC}\otimes U( \xgate^b \zgate^a( \ket{\psi}))_B.
\end{equation}
\item Trace out the subsystem $AC,$
 \begin{equation}
  \label{eq:telepprtation5}
U( \xgate^b \zgate^a( \ket{\psi}))_B=Tr_{AB}(\ket{\beta_{ab}}_{AC}\otimes U( \xgate^b \zgate^a( \ket{\psi}))_B.)
 \end{equation}
\item Compute the update function $f_U(a,b)=(a^\prime,b^\prime)$ associated with the gate $U.$ (See \Cref{update function}).
\item Apply the correction unitary $\zgate^{a^\prime}  \xgate^{b^\prime}$ to the system B.
 \begin{equation}
  \label{eq:telepprtation5}
  \zgate^{a^\prime}  \xgate^{b^\prime} U( \xgate^b \zgate^a( \ket{\psi}))_B=\zgate^{a^\prime}  \xgate^{b^\prime} \left(\xgate^{b^\prime} \zgate^{a^\prime} U( \ket{\psi})\right)\\
  =U( \ket{\psi})
   \end{equation}
\end{enumerate}	
\end{algorithm}	

\subsection{Update Functions for Quantum Circuits}
\label{update function}
In the gate teleportation (\Cref{algo:gate-teleport}) we have to compute bits $(a^\prime, b^\prime)\in\mathbb{F}_2^2$ such that
\begin{equation*}
U\xgate^b\zgate^a \ket{\psi}=\xgate^{b^\prime}\zgate^{a^\prime}U(\ket{\psi}),
\end{equation*}
The ordered pair $(a^\prime, b^\prime)$ are computed by an update rule that is completely determined by the gate $U$ and $(a,b)\in\mathbb{F}_2^2.$ We can describe this rule concisely by an update function
\begin{equation*}
f_U:\mathbb{F}^2\rightarrow \mathbb{F}^2, \; (a,b)\mapsto (a^\prime,b^\prime).
\end{equation*}
%\anote{I am not sure what these $f$ functions are doing, and how to read these equations.  Are there typos in the subscripts? }
For any $(a,b)\in\mathbb{F}_2^2$ and any single Clifford gate $U,$ the corresponding update functions are

\begin{equation*}
\begin{aligned}
\mbox{ If } U=\xgate,  \mbox{ then } f_\xgate(a,b)=(a,b).\\
\mbox{ If } U=\zgate,  \mbox{ then } f_\zgate(a,b)=(a,b).\\
\mbox{ If } U=\hgate,  \mbox{ then } f_\hgate(a,b)=(b,a).\\
\mbox{ If } U=\pgate,  \mbox{ then } f_ \pgate(a,b)=(a,a\oplus b).\\
%\cnot (\xgate^{b_1}\zgate^{a_1} \otimes \xgate^{b_2}\zgate^{a_2})=({\xgate^{b_1} {\zgate^{a_1\oplus a_2}}} \otimes {\xgate^{b_1\oplus b_2}}\zgate^{a_2})\cnot, \mbox{ where } f_{\cnot}(a_1,b_1,a_2,b_2)=(a_1\oplus a_2,b_1,a_2, b_1\oplus b_2).\\
\end{aligned}
\end{equation*}
Note we can evaluate $\cnot$ gate on any 2-qubit state $\ket{\psi}$ using gate-teleportation. In this case we have to prepare $4$ qubit Bell state $\ket{\beta_{00}}\otimes \ket{\beta_{00}}$ in step~1 ((\Cref{algo:gate-teleport})). After step~4 of \Cref{algo:gate-teleport}, the system will be in the state
\begin{equation*}
\begin{aligned}
 \ket\beta_{a_1,b_1}\otimes \ket\beta_{a_2,b_2} \otimes \cnot(\xgate^{a_1}\zgate^{b_1}\otimes \xgate^{a_2}\zgate^{b_2})\ket{\psi}
\end{aligned}
\end{equation*}
The update rule for the $\cnot$ gate is
\begin{equation*}
\begin{aligned}
f_{\cnot}:\mathbb{F}_2^4\rightarrow \mathbb{F}_2^4, \; (a_1,b_1,a_2,b_2)\mapsto (a_1\oplus a_2,b_1,a_2, b_1\oplus b_2).
\end{aligned}
\end{equation*}
Therefore we have
\begin{equation*}
\begin{aligned}
  \cnot(\xgate^{a_1}\zgate^{b_1}\otimes \xgate^{a_2}\otimes \zgate{b_2}))\ket{\psi}= (\xgate^{a_1\oplus a_2}\zgate^{b_1}\otimes \xgate^{a_2} \zgate^{b_1\oplus b_2})\cnot(\ket{\psi})
 \end{aligned}
\end{equation*}
	
In general, let $\mathcal{C}_q$ be an $n$-qubit Clifford circuit\footnote{A Clifford circuit is a quantum circuit in which every gate is from the Clifford group.} and $\mathcal{C}_q$ is asequence of gates $g_k,\ldots,g_2,g_1$ for $k=|\mathcal{C}_q|.$\footnote{In this paper we assume that we apply gates from right-to-left.}
\begin{equation}
\label{relation:update}
\begin{aligned}
 \ket\beta_{a_1,b_1}\otimes \cdots \otimes \ket\beta_{a_n,b_n} \otimes C_q(\xgate^{a_1}\zgate^{b_1}\otimes \cdots \otimes\xgate^{a_n}\zgate^{b_n})\ket{\psi}
\end{aligned}
\end{equation}
where  $(a_1,b_1,\ldots,a_n,b_n)\in\mathbb{F}_2^{2n}$ obtained in the step 4 of \Cref{algo:gate-teleport}.
The update function corresponding to the circuit $C_q$ is computed by composing the update functions of each gate in $\mathcal{C}_q$
\begin{equation*}
\begin{aligned}
F_{\mathcal{C}_q}:\, \mathbb{F}_2^{2n} \rightarrow  \mathbb{F}_2^{2n}, \; F_{\mathcal{C}_q}=f_k\circ\cdots \circ f_2 \circ f_1 \\(a_1,b_1,\dots, a_n,b_n)\mapsto (a_1^\prime,b_2^\prime,\dots, a_n^\prime,b_n^\prime)
 \end{aligned}
\end{equation*}
We can rewrite \Cref{relation:update} as
\begin{equation*}
\begin{aligned}
\ket\beta_{a_1,b_1}\otimes \cdots \otimes \ket\beta_{a_n,b_n} \otimes (\xgate^{a_1^\prime}\zgate^{b_1^\prime}\otimes \cdots \otimes\xgate^{a_n^\prime}\zgate^{b_n^\prime})C_q(\ket{\psi}),
 \end{aligned}
\end{equation*}
where $(a_1^\prime,b_1^\prime,\dots, a_n^\prime,b_n^\prime)=F_{C_q}(a_1,b_1,\ldots,a_n,b_n).$ \\

\Anne{I suggest maybe to remove this example?}
\noindent{\bf Example for Update Function:} $C_q=\cnot \cdot (\igate \otimes \hgate)$ is 2-qubit circuit the update function $F_{C_q}$ for this circuit is

\begin{equation*}
\begin{aligned}
F_{C_q}(a_1,b_1,a_2,b_2)=(b_1\oplus a_2,a_1,a_2, a_1\oplus b_2),\\
\end{aligned}
\end{equation*}
And computed as by first applying the $f_\hgate$ to the first two input bits and then $f_{\cnot}$ to all four bits.
\begin{equation*}
\begin{aligned}
(a_1,b_1,a_2,b_2) \xmapsto{\igate \otimes \hgate} (b_1,a_1,a_2,b_2)\xmapsto{\cnot} (b_1\oplus a_2,a_1,a_2, a_1\oplus b_2).
\end{aligned}
\end{equation*}

\begin{remark}
Using gate teleportation we can also evaluate a  $\tgate,$  hence any $n$-qubit circuit. However, the update function becomes more complicated. This is discuss in \Cref{QiO:Clifford+T:family+GT}.
%						$$\tgate \xgate^b \zgate^a= \xgate^b \zgate^{a\oplus b} \pgate^b \tgate$$
\end{remark}





			
				




