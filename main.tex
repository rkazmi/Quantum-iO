\documentclass[english,11pt]{article}
\usepackage[english]{babel}
\usepackage{latexsym,amssymb,amsmath,amsfonts,epsfig,amsthm,color,dsfont,mathtools}
 \usepackage{braket,graphicx,authblk,colonequals,algorithm,algpseudocode,ifthen,framed}
\usepackage{float,array}
\usepackage{xcolor} %
%\usepackage[left=1in,right=1in]{geometry}
\usepackage[leftbars,color]{changebar}
\newtheorem*{remark}{Remark}
\usepackage[utf8]{inputenc}
\usepackage[english]{babel}
%\usepackage[toc]{multitoc}
%\usepackage{cleveref}
\floatname{algorithm}{Algorithm}


%********************
\newcommand{\Anne}[1]{{\color{red}\sc note}\footnote{\color{red} \underline{{\sc anne}:} #1}}
\newcommand{\Raza}[1]{{\color{red}\sc note}\footnote{\color{red} \underline{{\sc gus}:} #1}}
\newcommand{\bit}[1]{\{0,1\}^{#1}}


\newboolean{ElectronicVersion}
\setboolean{ElectronicVersion}{true}

\ifthenelse{\boolean{ElectronicVersion}}{
    \usepackage[letterpaper=true,pdftex,bookmarks,pagebackref,
	plainpages=false, % Needed if Roman numbers in frontpages
        pdfpagelabels=true % Adds page number as label in Acrobat's page count
        ]{hyperref}}{}

%\makeatletter
%\newtheorem*{rep@theorem}{\rep@title}
%\newcommand{\newreptheorem}[2]{%
%\newenvironment{rep#1}[1]{%
% \def\rep@title{#2 \ref*{##1}}%
% \begin{rep@theorem}}%
% {\end{rep@theorem}}}
%\makeatother


\newtheorem{theorem}{Theorem}[section]
\newtheorem{lemma}[theorem]{Lemma}
\newtheorem{corollary}[theorem]{Corollary}
\newtheorem{proposition}[theorem]{Proposition}
\newtheorem{definition}{Definition}
\newtheorem{cor}[theorem]{Corollary}
\newtheorem{fact}[theorem]{Fact}
\newtheorem{claim}[theorem]{Claim}
\newtheorem{conjecture}[theorem]{Conjecture}
\newtheorem{example}[theorem]{Example}
%\newreptheorem{theorem}{Theorem}
\newtheorem{goal}{Research Goal}[section]


% highlight
\newcommand{\hlight}[1]{\ \\ \fcolorbox{black}{white}{\begin{minipage}{6.5in}{\color{black} #1}\end{minipage}} \\}

% pretty TODOs
\definecolor{darkgreen}{rgb}{0,.5,0}
\newcommand{\TODO}[1]{\ \\ \fcolorbox{darkgreen}{white}{\begin{minipage}{6in}{\color{darkgreen}{\underline{\textbf{TODO:}}} #1}\end{minipage}} \\}

\renewenvironment{proof}{\setlength{\parindent}{\parindent}\def\FrameCommand{{\color{black}\vrule\hspace{3pt}}}\MakeFramed {\noindent\emph{Proof.}}{}}
{\qed\endMakeFramed} %left barred proofs

%%% Environment for schemes
\newcommand{\scheme}[2]{\begin{minipage}{6in}
\hrule
\vspace{2pt}
#1
\vspace{2pt}
\hrule
#2
\hrule
\end{minipage}}


\def\ket#1{{\lvert}#1\rangle}
\def\bra#1{{\langle}#1\rvert}
\def\braket#1#2{{{\langle}#1\vert}#2\rangle}
\def\abs#1{\left| #1 \right|}
\def\ceil#1{{\lceil}#1\rceil}
\def\floor#1{{\lfloor}#1\rfloor}
\def\norm#1{\left\| #1 \right\|}
\DeclareMathOperator{\Tr}{Tr}
\def\span{\mbox{span}}
\DeclareMathOperator{\Var}{Var}
\DeclareMathOperator{\E}{E}
\def\cost#1{\mathfrak{#1}}
\def\a{f_A}
\def\b{f_B}
\newcommand{\eps}{\varepsilon}
\def\O{\mathrm{O}}
\def\tO{\widetilde{\mathrm{O}}}
\def\set#1{\mathcal{#1}}
\renewcommand{\th}[1]{${#1}^{\textrm{th}}$}
\renewcommand{\(}{\left(}
\renewcommand{\)}{\right)}
\newcommand{\defeq}{\colonequals}
\newcommand{\BSR}{\mathbb{B}}
\def\m{m}

\newcommand{\sA}{{\sf A}}
\newcommand{\sB}{{\sf B}}
\newcommand{\sR}{{\sf R}}
\newcommand{\cA}{{\mathcal{A}}}
\newcommand{\cB}{{\mathcal{B}}}


\newcommand{\igate}{{\sf I}}
\newcommand{\hgate}{{\sf H}}
\newcommand{\pgate}{{\sf P}}
\newcommand{\tgate}{{\sf T}}
\newcommand{\xgate}{{\sf X}}
\newcommand{\ygate}{{\sf Y}}
\newcommand{\zgate}{{\sf Z}}
\newcommand{\cnot}{{\sf CNOT}}
\newcommand{\cEnc}{{\mathsf{Enc}}^{\mathsf c}}
\newcommand{\cDec}{{\mathsf{Dec}}^{\mathsf c}}
\newcommand{\cEval}{{\mathsf{Eval}}^{\mathsf c}}
\newcommand{\cKeyGen}{{\mathsf{KeyGen}}^{\mathsf c}}
\newcommand{\cek}{ek^{\mathsf{c}}}
%\mathbb{N}ewcommand{\csk}{sk^{\mathsf{c}}} 
%********************




\title{Indistinguishability obfuscation \\for\\ quantum circuits of low $\tgate$-gate complexity}


\author{Anne Broadbent and Raza Ali Kazmi}
\affil{University of Ottawa, Department of Mathematics and Statistics,\\\texttt{\{abroadbe,rkazmi\}@uottawa.ca.}}
%\affil[2]{Raza affiliation}
\date{} %leave blank for the submission

\begin{document}

\maketitle

\begin{abstract}
An \emph{indistiguishability obfuscator} takes as input a circuit %C$ 
and outputs a new circuit, %$C'$, 
such that for any two circuits that compute the  \emph{same} function, the ouputs of the indistinguishability obfuscator are 
%$C'_1$ and $C'_2$ are
 computationally indistinguishable. Here, we initiate the study of feasibility for indistinguishability obfuscation for \emph{quantum} circuits. We construct an indistinguishability  obfuscator that takes as input a quantum circuit, and outputs a quantum state and a quantum circuit, which together can be used to evaluate the original quantum circuit, on any quantum input; the security guarantee being that for two quantum circuits that have the same input-output behaviour, the outputs of the obfuscator are indistinguishable. In our construction, the size of the output of the obfuscator is exponential in the number of non-Clifford gates, which means that the construction is efficient as long as the number of non-Clifford gates is logarithmic. \anote{Need to update prior text, based on final results. }
\end{abstract}

\newpage

\setcounter{tocdepth}{2}
\tableofcontents
\clearpage
\pagenumbering{arabic}
\setcounter{page}{1}

%%%%%%%%%%%%%%%%%%%%%%%%%%%%%%%%%%%%%%%%%%%%%%%%%%%%
%Section Introduction
%%%%%%%%%%%%%%%%%%%%%%%%%%%%%%
%********************
\section{Introduction}
%\anote{start more smoothly. What is iO, how did it come about, why do we care, what is the state-of-the-art? Also, mention somewhere that a canonical form is an iO}
%\anote{Add more background about the state-of-the art in iO: what are the current candidates, and is there a post-quantum one?}
%\anote{introduce quantum cryptography. Introduce Quantum iO as per \cite{AF16arxiv}.}
A program obfuscation is a probabilistic polynomial-time algorithm that can transform a circuit $C$ into another circuit $C’$ that has the same functionality as $C$ but does reveal anything about $C,$ except its functionality i.e. anything that can be learned from $C’$ about $C$ can also be learned from a black-box access to the functionality (input/output) of $C.$ This is called virtual black-box obfuscation. Unfortunately, Barak el~\cite{BGI+12}  proved that virtual black-box obfuscation is impossible in general.\footnote{They construct a family of booleans functions which is inherently unobfuscatable.} Motivated by this impossibility result Barak el~\cite{BGI+12} proposed a weaker called Indistinguishability Obfuscation $i\mathcal{O}.$ 

An, $i\mathcal{O}$ is a probabilistic polynomial-time algorithm that takes a circuit $C$ as an input and outputs a circuit $i\mathcal{O}(C)$ such that $i\mathcal{O}(C)(x)=C(x)$ for all inputs $x$ and size of $i\mathcal{O}(C)$ is at most polynomial in the size $C.$ Moreover, it must be that for any two circuits $C_1$ and $C_2$ of the same size, the obfuscations of both circuits remain computationally indistinguishable, i.e. $iO(C_1)\sim^c iO(C_2).$ The $i\mathcal{O}$ achieves the notion of best possible obfuscations, which state that any information that is not hidden by the obfuscated circuit is also not hidden by any other similar size circuit computing the same functionality \cite{GR2014}. The indistinguishability obfuscation is a very powerful cryptographic tool which that has been shown to enable digital signatures, public key encryption~\cite{SW14}, multiparty key agreement, broadcast encryption~\cite{BZ14}, fully homomorphic encryption~\cite{CLTV15} and zero knowledge~\cite{BP15} etc. 

The first candidate construction of $i\mathcal{O}$ was published in \cite{GGH+13}, whose security relied on the presume hardness of multilinear maps \cite{CLT13,GCH13, LSS14, GGH15}. Unfortunately, there have been many quantum attacks on some multilinear maps \cite{ABD16, CDPR16, CGH17}. We want to emphasize that there are $i\mathcal{O}$ constructions that have been so far remain secure against quantum adversaries \cite{Z19}, but no provably secure quantum $i\mathcal{O}$ is known today. Recently, a new $i\mathcal{O}$ scheme has been proposed under the some different assumptions \cite{ALMS19}. However, It is too premature to say anything about the quantum security of this new construction \cite{ALMS19}. 

In this work, we continue the study of computational indistinguishable obfuscation in the quantum world. The main contribution of this paper is to provide a new definition of quantum indistinguishability obfuscation $Qi\mathcal{O}$ and show how to construct a $Qi\mathcal{O}$ for certain families of quantum circuits. The first construction is based on the canonical representation of Clifford circuits~\cite{AG04}, while the second construction is based on the principle of gate teleportation~\cite{GC99}. The two constructions present different advantages: the technique using the canonical form is straightforward and does not require any computational assumption. Moreover, the obfuscated circuits are classical, and hence can be easily communicated, stored, used and copied.  In contrast, the gate-teleportation scheme requires the assumption of quantum-secure classical $i\mathcal{O}$ for a certain family of functions (section \ref{correction function}) and the obfuscated circuits are quantum.\footnote{Note that our construction does not require a full $i\mathcal{O}$ but rather rely on a weaker $i\mathcal{O}$ that can obfuscate a certain family of functions.} While this presents a technological challenge to communication, storage and also usage, this methodology allow us to obfuscate a more general set of quantum circuit than the canonical base construction. Moreover the gate-teleportation methodology could also enable a new functionality related to the \emph{unclonability} of quantum states.

%In this work, we continue the study of $i\mathcal{O}$ in the quantum world. We give two main contributions, both assuming the existence of quantum-secure $q\mathcal{O}$ for classical circuits.
%Our first contribution echoes the private-to-public key transformation of~\cite{SW14}, by showing its applicability to the encryption of \emph{quantum} messages. In this case, the security is based on (quantum-secure one-way functions?)..., and is shown according to the notion of quantum semantic security ~\cite{BJ15,ABF+16}. Our second contribution shows how to build $i\mathcal{O}$ for certain families of quantum circuits. Namely, we show two different ways in which Clifford circuits admit an $i\mathcal{O}$ scheme. The first method is based on the canonical representation of Clifford circuits~\cite{AG04,Nest10}, while the second method is based on the principle of gate teleportation~\cite{GC99}. We  extend these schemes to circuits that include layers of non-Clifford $\tgate$-gates, as long as each Clifford layer is within a family of circuits that admit the $i\mathcal{O}$ obfuscation described above, and as long as the $\tgate$-gates are in fixed positions.
% The two above methods present different advantages: the technique using the canonical form is straightforward and in fact does not require the assumption of a quantum-secure classical $i\mathcal{O}$. By construction, the obfuscated circuits is classical, and hence can be easily communicated, stored, used and copied. In contrast, the gate-teleportation scheme requires the assumption of quantum-secure classical $i\mathcal{O}$. By construction, the obfuscated circuits are quantum. While this presents a technological challenge to communication, storage and also usage, it could enable a new functionality related to the \emph{unclonability} of quantum states (see Open Questions below).

%\Anne{note to Raza: when adding bib entries to quantum.bib, just add a quick entry at the bottom of quantum.bib. Then Sebastien will clean it up and send us the corrected .bib (Sébastien has a small contract as admin of the quantum.bib).}

\subsection{Obfuscation in a Quantum World}
The quantum obfuscation was first studied in \cite{AJJ14}, where a new notion called
$(G,\Gamma)$-{\em indistinguishability obfuscation} was proposed. Where $G$ is a set of gates and $\Gamma$ is a set of relations satisfied by the elements of $G.$ In this notion any two circuits over the set of gates $G$ are perfectly indistinguishable if they differ by some sequence of applications of the relations in $\Gamma.$ One of the motivations of their work was to provide a weaker definition of perfectly indistinguishable obfuscation, which is shown to be impossible under certain complexity-theoretic assumptions \cite{AJJ14}. However, $(G,\Gamma)$-{\em indistinguishability obfuscation}  appears to be incomparable with the computational indistinguishability obfuscation \cite{BGI+12,GGH+13 }, which is the topic of this paper.

The quantum obfuscation is studied more rigorously in \cite{AF16a}, where the following notions of quantum obfuscation are defined.
\begin{enumerate}
\item Quantum black-box obfuscation.
\item Quantum information-theoretic black-box obfuscation.
\item Quantum indistinguishability obfuscation (perfect, statistical, computational).
\item Quantum Best-Possible obfuscation (perfect, statistical, computational).
\end{enumerate}

 The main contribution of their work was to extend the classical impossibility results to the quantum settings such as a generic transformation of quantum circuits into black-box-obfuscated quantum circuits is impossible \cite{AF16arxiv},  statistical indistinguishability obfuscation is impossible, up to an unlikely complexity-theoretic collapse \cite{AF16arxiv}. However, no concrete instantiation was provided in their paper for any of type quantum obfuscations. They also discussed a number of applications of quantum black-box obfuscation such as CPA-secure quantum encryption, quantum fully-homomorphic encryption, and public-key quantum money, however it is not clear what impact these impossibility results have on these applications. They also showed that an existence of a computational quantum indistinguishability obfuscation would imply a witness encryption scheme for all languages in QMA \cite{arXiv:1602.01771v1}.


\subsection{Our Results}
In this work we define a new definition of computational quantum obfuscation $Qi\mathcal{O}$ and provide two concrete instantiation of it. The first instantiation is based on the work of Aaronson and Gottesman~\cite{AG04}, In this paper, an efficient algorithm to compute a canonical form of any Clifford circuit was describe. The second instantiation is based on gate-teleportation \cite{} and assume an existence of a quantum-secure classical $i\mathcal{O}$ for certain class of functions (section \ref{correction function}). Using this technique we can obfuscate any $n$-qubit quantum circuit as far as the number of $\tgate$-gates are in $O(\log(n)).$






%Quantum indistinguishability  for the encryption of quantum messages was first defined in~\cite{BJ15}, and was shown to be equivalent to quantum semantic security~\cite{ABF+16}. In terms of concrete schemes, \cite{BJ15} showed... while, \cite{ABF+16} showed...
%In contrast, our contribution using $i\mathcal{O}$ shows.....

%To the best of our knowledge, our second contribution is the first that describes how to achieve $i\mathcal{O}$ for some families of quantum circuits
%(however, as noted above, the Clifford canonical form is well-known, and its extension to $i\mathcal{O}$ (see Section~\ref{sec:Clifford-$i\mathcal{O}$-canonical}) is relatively straightforward.


\subsubsection{Cliffords} TODO: place somewhere 
Although Clifford circuits can be efficiently simulated on a classical computer \cite{Got98}, this simulation is with respect to a \emph{classical} distribution, hence for a purely quantum computation, quantum circuits are required. While Clifford circuits are known not to be universal for quantum computation, they are an important building block for fault-tolerant quantum computing, for instant, due to the fact that Cliffords admit transversal computations on many fault-tolerant codes\cite{BCL+06}.



\anote{Make sure we emphasize somewhere why the paper belong in the ICT conference--- perhaps because the reduction is information-theoretic, assuming classical iO?}

\anote{TODO: emphasize other work where a limited number of $T$-gates was tolerated, e.g. \cite{BJ15}., and also how the $T$-gate is often seen as the bottle-neck, e.g. in fault-tolerant quantum computing. }

\Anne{Is there an advantage in using quantum states to obfuscate programs? I would suspect that one big advantage would be unclonable programs (they cannot be copied). Is this too ambitious to look at?  What does it even mean to have unclonable programs? (this links with work that I am currently doing with Sébastien Lord, a PhD student).}

%%%%%%%%%%%%%%%%%%%%%%%%%%%%%%%%%%%%%%%%%%%%%%%%%%%%
%Section applications
%%%%%%%%%%%%%%%%%%%%%%%%%%%
\subsection{Applications}


%%%%%%%%%%%%%%%%%%%%%%%%%%%%%%%%%%%%%%%%%%%%%%%%%%%%
%Section Conclusion and Open Problems
%%%%%%%%%%%%%%%%%%%%%%%%%%%
\subsection{Conclusion and Open Questions}

The main open questions related to this work are:

$i\mathcal{O}$ obfuscation for general quantum circuits

applications of gate-teleportation based quantum $i\mathcal{O}$ (for instance, to unclonable programs~\cite{Aar09}).


\subsection*{Outline}
In section \ref{sec:QiO-Cliffords and more} we provide a new definition of quantum indistinguishability obfuscation. We provide two methodologies for constructing a quantum indistinguishability obfuscation for Clifford circuits. One is based on canonical form of Clifford circuits  while other is based on the principle of gate teleportation.  The construction based on canonical form requires no computational assumptions and is in fact perfectly indistinguishability obfuscation (information theoratically secure), where as the gate teleportation based construction assumes the existence of a quantum secure classical $i\mathcal{O}.$ In section \ref{QiO:Clifford+T:family} we show how to a construct quantum indistinguishability obfuscation from gate teleportation and $i\mathcal{O}$ for any reversible quantum circuit that has at most logarithmic $\tgate$-gates.


%********************



%%%%%%%%%%%%%%%%%%%%%%%%%%%%%%%%%%%%%%%%%%%%%%%%%%%%
%Section Prelims
%%%%%%%%%%%%%%%%%%%%%%%%%%%
%********************
\section{Preliminaries}
\subsection{Basic Classical Cryptographic Notions}
\label{sec:classical-prelims}
Let $\mathbb{N}$ be the set of positive integers. For $n \in \mathbb{N}$, we set $[n] = \{1, \cdots, n\}.$ We denote the set of all  binary strings of length $n$ by $\bit{n}.$
 An element $s \in \bit{n}$ is called a bitstring, and $|s|=n$ denotes its length. We denote an arbitrary polynomial from the set $\mathbb{N}$ to $\mathbb{N}$ by $poly( ).$


A function ${\rm negl}:\mathbb{N}\rightarrow\mathbb{R}^{+}\cup \{0\}$ is \emph{negligible} if for every positive polynomial $p(n)$ there exists a positive integer $n_0$ such that  for all  $n>n_0,$ ${\rm negl}(n) < 1/ p(n).$   A typical use of negligible functions is to indicate that the probability of success of some algorithm is too small to be amplified to a constant by a feasible (\emph{i.e.}, polynomial) number of repetitions. Given two bit strings $x$ and $y$ of equal length, we denote their bitwise XOR by $x \oplus y.$

\subsubsection{Classical Circuits and Algorithms}
 For $n, m\in\mathbb{N}$ let  $f:\bit{n}\rightarrow \bit{m}$ be a function. We say a circuit $C$ computes $f$ if for every $s \in \bit{n},\; C(s)=f(s).$  We define the size of a circuit $C$ as the number of gates in it and is denoted by $|C|.$ A set of gates for classical computation  are universal if, for all $n, m\in\mathbb{N}$, and for every function $f:\bit{n}\rightarrow \bit{m}$ a circuit can be constructed for computing~$f$ using only gates from that set. It is a well known fact that \{AND, OR, NOT\} is a set of universal gates for classical circuits. A family of circuits $\mathcal{F}=\{C_n\mid n\in\mathbb{N}\},$ one for each input  size $n\in \mathbb{N}$ is called uniform if there exists a deterministic Turing machine $M,$ such that
 \begin{itemize}
 \item For each $n\in\mathbb{N},$ $M$ outputs a description of $C_n \in \mathcal{F}$ on input $1^n.$
 \item  For each $n\in\mathbb{N},$ $M$ runs in time $poly(n).$
 \end{itemize}


\begin{definition}\label{def:qiO} {\rm({\bf Quantum Secure Indistinguishability Obfuscation} $i\mathcal{O}$)}
A probabilistic polynomial-time algorithm is a \emph{quantum-secure computational indistinguishability obfuscator} $i\mathcal{O}$ for a class of circuits ${\mathcal C},$ if the following conditions hold:

\begin{enumerate}
\item {\tt Functionality:} For any circuit $C\in {\mathcal C},$ and for all inputs $x$ $$i\mathcal{O}(C)(x)=C(x).$$
\item  {\tt Polynomial Slowdown:}  For every $C\in \mathcal{C},$  $|i\mathcal{O}(C)| \in poly(|C|).$
\item {\tt Indistinguishability:} For any two circuits $C_0,C_1\in {\mathcal C},$ of the same size  that compute the same function
 and for every polynomial time quantum distinguisher $\mathcal{D}_q,$  there exists a negligible function {\rm negl} such that:

					$$\Big | {\rm Pr}[\mathcal{D}_q(i\mathcal{O}(C_0))=1]-{\rm Pr}[\mathcal{D}_q(i\mathcal{O}(C_1))=1] \Big |\leq  {\rm negl}(|C_0|).$$			
\end{enumerate}										
\end{definition}



\subsection{Basic Quantum Notions}
\label{sec:quantum-prelims}
%%%%%%%%%%%%%%%%%%%%%%%%%%
Given an $n$-bit string $x$, the corresponding quantum-computational $n$-qubit basis state is denoted~$\ket{x}$. The $2^n$-dimensional Hilbert space spanned by $n$-qubit basis states is denoted:
%\anote{in general, we will try to include equation numbers unless there is a reason not to. This will help in reviewing and referencing our work.}
\begin{equation}
\label{eq:hilbert-space}
\mathcal{H}_n := \textbf{span} \left\{ \ket{x} : x \in \bit{n} \right\}\,.
\end{equation}
We denote by $\mathcal{D}(\mathcal{H}_n)$ the set of density operators (\emph{i.e.}, valid quantum states) on~$\mathcal{H}_n$. These are linear operators on $\mathcal{D}(\mathcal{H}_n)$ which are positive-semidefinite and have trace equal to $1.$


%\anote{In general, there is some confusion throughout related to pure and mixed states. This preliminaries is good, but need to make sure that it matches the level of formalism in the main body.}
%. When considering different physical subsystems, we denote them with uppercase Latin letters; when a Hilbert space corresponds to a subsystem, we place the subsystem label in the subscript. For instance, if $F \cup G \cup H = [n]$ then $\mathcal{H}_n = \mathcal{H}_F \otimes \mathcal{H}_G \otimes \mathcal{H}_H.$ Sometimes we  write explicitly the subsystems a state belongs to as subscripts; this will be useful when considering, \emph{e.g.}, the reduced state on some of the subspaces. For example, wesometimes express the statement $\rho \in \mathcal{D}(\mathcal{H}_F \otimes \mathcal{H}_G \otimes \mathcal{H}_H)$ simply by calling the state $\rho_{FGH}$; in that case, the state obtained by tracing out the subsystem~$H$ will be denoted~$\rho_{FG}$.


%Given $\rho, \sigma \in \mathcal{D}(\mathcal{H})$, the trace distance between $\rho$ and $\sigma$ is given by half the trace norm $\|\rho - \sigma\|_1$ of their difference. When $\rho$ and $\sigma$ are classical probability distributions, the trace distance reduces to the total variation distance. Physically realizable maps from a state space $\mathcal{D}(\mathcal{H})$ to another state space $\mathcal{D}(\mathcal{H}')$ are called \emph{admissible}---these are the completely positive trace-preserving (CPTP) maps. For the purpose of distinguishability via input/output operations, the appropriate norm for CPTP maps is the diamond norm, denoted $\|\cdot\|_\diamond$. The set of admissible maps coincides with the set of all maps realizable by composing (i.) addition of ancillas, (ii.) unitary evolutions, (iii.) measurements in the computational basis, and (iv.) tracing out subspaces. We remark that unitaries $U \in U(\mathcal{H}_n)$ act on $\mathcal{D}(\mathcal{H}_n)$ by conjugation: $\rho \mapsto U \rho U^\dagger$. The identity operator~$\mathds{1}_n \in U(\mathcal{H}_n)$ is thus both a valid map, and (when normalized by $2^{-n}$) a valid state in $\mathcal{D}(\mathcal{H}_n)$---corresponding to the classical uniform distribution.


\subsubsection{Quantum Gates}
We will work with the following set of unitary gates

$$\xgate = \left[\begin{array}{cc} 0 & 1\\ 1 & 0\end{array}\right],
\quad\zgate = \left[\begin{array}{cc} 1 & 0\\ 0 & -1\end{array}\right],
\quad\pgate = \left[\begin{array}{cc} 1 & 0\\ 0 & i\end{array}\right],
 \quad\hgate = \frac{1}{\sqrt{2}}\left[\begin{array}{cc}1 & 1\\1 & -1\end{array}\right], \quad\mbox{and}$$
$$\quad\cnot = \left[\begin{array}{cccc} 1 & 0 & 0 & 0\\ 0 & 1 & 0 & 0\\ 0 & 0 & 0 & 1\\ 0 & 0 & 1 & 0\end{array}\right].$$

\subsubsection*{Quantum One Time Pad}
For any single-qubit density operator $\rho \in \mathcal{D} (\mathcal{H}_1)$ and for any uniformly chosen bits $s$ and $t,$ 
$$
\rho \mapsto \xgate^s \zgate^t \rho \zgate^t \xgate^s\,.
$$
 the resulting state is information-theoretically indistinguishable from $\mathds{1}_1/2$ to an observer that has no knowledge of $s$ and $t$ \cite{AMTW00}.

%The above map can be straightforwardly extended to the $n$-qubit case in order to obtain an elementary {\em quantum encryption scheme} called the {\em quantum one-time pad}\cite{AMTW00}.
%We first set $\xgate_j = \mathds{1}^{\otimes j-1} \otimes \xgate \otimes \mathds{1}^{\otimes n-j}$ and likewise for $\ygate_j$ and $\zgate_j$. We define the $n$-qubit Pauli group $\mathcal P_n$ to be the subgroup of $\operatorname{SU}(\mathcal{H}_n)$ generated by $\{\xgate_j, \ygate_j, \zgate_j : j = 1, \dots, n \}$. Note that Hermiticity is inherited from the single-qubit case, \emph{i.e.}, $P^\dag = P$ for every $P \in \mathcal{P}_n$.
%
%\anote{It is not clear to me that, above, the identity is part of $\mathcal{P}_n$. I think it should be.}
%\anote{$\xgate$ and $\zgate$ already defined. I would re-phrase, first defining the new gates, and then defining the Clifford group.}

\begin{definition}
\label{defn:Clifford+T:family}
 \noindent{\bf Clifford Group}: The set of gates $\{\xgate, \zgate, \pgate, \cnot,\hgate\}$ applied to arbitrary wires (redundantly) generates the Clifford group, where

We note the following relations between these gates:
$$\xgate\zgate = - \zgate\xgate,\quad \tgate^2=\pgate,\quad\pgate^2=\zgate,\quad\hgate\xgate\hgate=\zgate,\quad \tgate\pgate=\pgate\tgate,\quad\pgate\zgate=\zgate\pgate.$$
Also, for any $a,b\in\{0,1\}$ we have
$\hgate\xgate^b\zgate^a=\xgate^a\zgate^b \hgate$\,.
\end{definition}

\subsubsection{Quantum Circuits and Algorithm}
A quantum circuit is an acyclic network of quantum gates connected by wires. The quantum gates represent quantum operations and wire represents the qubits on which gates act. In, general a quantum circuits can have $n$-input qubits and $m$-output qubits for any integer $n,m\geq 0.$
%\anote{todo: define what is a quantum circuit, and also any type of notation that will be used in the rest of the paper. For instance, what does $\hgate-\cnot-\pgate-\cnot-\pgate-\cnot-\hgate-\pgate-\cnot-\pgate-\cnot$ mean? Do I apply gates from left-to-right, or from right-to-left?}

For $n\in \mathbb{N},$ the set of all $n\times n$ unitary matrix is denoted by $O(n,\mathbb{C})=\{U\in \mathbb{C}^{n\times n} \mid U\cdot U^\dagger={\bf I}\}.$ We say a quantum circuit $C_q$ computes $U\in O(n,\mathbb{C})$ if for every $\rho\in \mathcal{H}_n$
$$U(\rho )=C_q(\rho).$$
A quantum circuit that computes unitary matrix is called a reversible  quantum circuit, i.e it always possible to uniquely recover the input, given the output. A set of gates are said to be universal if, for any a unitary matrix $U$  a quantum circuit can be constructed for computing~$U$ using only gates from that set. It is a well-known fact that Clifford gates are not universal, but adding any non-Clifford gate, such as $\tgate$, gives a universal set of gates, where
$$\quad\tgate = \left[\begin{array}{cc} 1 & 0\\ 0 & e^{i\pi/4}\end{array}\right],.$$

A family of quantum circuits $C=\{C_n\in \mathbb{N} \mid \}$ one for each input  size $n\in \mathbb{N}$ is called \emph{uniform} if there exists a deterministic Turing machine $M,$ such that
 \begin{itemize}
 \item For each $n\in\mathbb{N},$ $M$ outputs a description of $C_n \in \mathcal{F}$ on input $1^n.$
 \item  For each $n\in\mathbb{N},$ $M$ runs in $poly(n).$
 \end{itemize}

All quantum operations are not unitary (reversible), nevertheless a general (possibly irreversible) quantum operation also called superoperator can efficiently be simulated by a reversible quantum operations by adding auxiliary states to the original system, then performing a unitary operation on the joint system, and then tracing out $Tr$ some subsystem \cite{KLM07}. More precisely, this can be described as the map:
%\anote{todo: find ref (I think Sebastien has some references in his thesis). Actually, do we use this?} 
$$\rho_{in} \xmapsto{\mbox{superoperator}} \rho_{out}=Tr_B(U(\rho_{in} \otimes \ket{00\cdots 0} \bra{00\cdots 0})U^\dagger)$$

where $\rho_{in} \in \mathcal{H}_n,$ is the original state and $\ket{00\cdots 0}$ is an auxiliary state of dimension at most $n^2.$ A circuit that computes a general quantum operation is called a general quantum circuit.  Therefore, general quantum circuits can refer to both reversible or irreversible circuits. A polynomial-time quantum algorithm is a uniform family of general quantum circuits. \\

%\noindent {\em Remark}: From now we use the term  quantum circuits to refer to reversible quantum circuits only and the term quantum algorithm is reserved for some family of general quantum circuits. \anote{ok, well actually I do believe that we could define quantum circuits that also include auxiliary qubit preparation and measurements/trace outs, and that we can obfuscate these also using gate teleportation. Hence the distinction circuit/algorithm is not very natural. Need to think about this.}








\subsection{Gate Teleportation}
\label{protocol: gate-teleportation}

We recall the \emph{Bell states}: %$\ket{\beta_{00}},$  $\ket{\beta_{01}},$ $\ket{\beta_{10}}$ and $\ket{\beta_{11}}.$
$\ket{\beta_{00}}=\frac{1}{\sqrt2}\left(\ket{00}+\ket{11}\right)$, ${\beta_{01}}=\frac{1}{\sqrt2}\left(\ket{01}+\ket{10}\right)$,
$\ket{\beta_{10}}=\frac{1}{\sqrt2}\left(\ket{00}-\ket{11}\right)$,  $\ket{\beta_{11}}=\frac{1}{\sqrt2}\left(\ket{01}-\ket{10}\right)$\,. 

Suppose we want to evaluate a single qubit gate $U\in\{\xgate, \zgate, \pgate, \cnot,\hgate\}$ on some qubit $\ket{\psi}.$ Then using gate teleportation  \cite{GC99} we can compute $U (\ket{\psi})$ as follows.
\begin{algorithm}[H]
\caption{Gate Teleportation.}
\begin{enumerate}
\item  Prepare a $2$ qubit Bell state $\ket{\beta_{00}}=\frac{\ket{00}+\ket{11})}{\sqrt{2}}.$
\item Write the joint system as
 \begin{equation}
 \label{eq:telepprtation1}
 \begin{aligned}
\ket{\psi}_C \ket{\beta_{00}}_{AB}=\frac{1}{2}\ket{\beta_{00}}_{AC} \ket{\psi}_{B} + \frac{1}{2}\ket{\beta_{01}}_{AC} ( \xgate( \ket{\psi})_{B}  +\\
 \frac{1}{2}\ket{\beta_{10}}_{AC} ( \zgate( \ket{\psi})_{B} + \frac{1}{2}\ket{\beta_{11}}_{AC} ( \xgate \zgate( \ket{\psi})_{B}.
\end{aligned}
\end{equation}
 Where $\beta_{ij},$  denotes the 2 qubit Bell basis.
\item Apply the Clifford gate $U$ on the subsystem $B,$
 \begin{equation}
  \label{eq:telepprtation2}
 \begin{aligned}
\mathbb{I}\otimes \mathbb{I} \otimes U(\ket{\psi}_A \ket{\beta_{00}}_{BC})=\frac{1}{2}\ket{\beta_{00}}_{AC} U(\ket{\psi})_{B}+ \frac{1}{2}\ket{\beta_{01}}_{AC}U( \xgate( \ket{\psi})_{B} +\\ \frac{1}{2}\ket{\beta_{10}}_{AC}U( \zgate( \ket{\psi})_{B}+ \frac{1}{2}\ket{\beta_{11}}_{AC}U( \xgate \zgate( \ket{\psi})_{B}.
\end{aligned}
 \end{equation}
 \item Measure the subsystem AC  in the Bell basis and obtain the classical bits $(a,b).$ The system is now in the state
 \begin{equation}
  \label{eq:telepprtation3}
							\ket{\beta_{ab}}_{AC}\otimes U( \xgate^b \zgate^a( \ket{\psi}))_B.
\end{equation}
 \label{eq:telepprtation4}
\item Compute the correction function $f_U(a,b)=(a^\prime,b^\prime)$ associated with the gate $U.$ (section \ref{correction function}).
\item Trace out the subsystem $AC,$
 \begin{equation}
  \label{eq:telepprtation5}
U( \xgate^b \zgate^a( \ket{\psi}))_B=Tr_{AB}(\ket{\beta_{ab}}_{AC}\otimes U( \xgate^b \zgate^a( \ket{\psi}))_B.)
 \end{equation}
\item Apply the correction unitary $\zgate^{a^\prime}  \xgate^{b^\prime}$ to the system B.
 \begin{equation}
  \label{eq:telepprtation5}
  \zgate^{a^\prime}  \xgate^{b^\prime} U( \xgate^b \zgate^a( \ket{\psi}))_B=\zgate^{a^\prime}  \xgate^{b^\prime} \left(\xgate^{b^\prime} \zgate^{a^\prime} U( \ket{\psi})\right)\\
  =U( \ket{\psi})
   \end{equation}
\end{enumerate}	
\end{algorithm}	

\subsection{Correction and Update Functions for Clifford Circuits}
\label{correction function}
For a quantum unitary gate we define a correction function $f_U$ such that
\begin{equation*}
U\xgate^b\zgate^a=\xgate^{b^\prime}\zgate^{a^\prime} U,  \quad \mbox{ where }(a^\prime,b^\prime)=f_\xgate(a,b) \mbox{ for any } a, b\in\{0,1\}.
\end{equation*}
%\anote{I am not sure what these $f$ functions are doing, and how to read these equations.  Are there typos in the subscripts? }
If $U\in\{\xgate, \zgate, \pgate, \cnot,\hgate\}$ is a Clifford gate, then
\begin{equation*}
\begin{aligned}
\xgate(\xgate^b\zgate^a)=(-1)^a(\xgate^{b}\zgate^{a}) \xgate,\quad \mbox{ where }f_\xgate(a,b)=(a,b), \mbox{ for any } a, b\in\{0,1\}.\\
\zgate(\xgate^b\zgate^a)=(-1)^b(\xgate^{b}\zgate^{a}) \zgate,\quad \mbox{ where }f_\zgate(a,b)=(a,b), \mbox{ for any } a, b\in\{0,1\}.\\
\hgate(\xgate^b\zgate^a)=(\xgate^{a}\zgate^{b}) \hgate,\quad \mbox{ where }f_\hgate(a,b)=(b,a), \mbox{ for any } a, b\in\{0,1\}.\\
 \pgate(\xgate^b\zgate^a)=((-i)^b\xgate^a\zgate^{a\oplus b}) \pgate, \quad \mbox{ where }f_ \pgate(a,b)=(a,a\oplus b), \mbox{ for any } a, b\in\{0,1\}.
\cnot (\xgate^{b_1}\zgate^{a_1} \otimes \xgate^{b_2}\zgate^{a_2})=({\xgate^{b_1} {\zgate^{a_1\oplus a_2}}} \otimes {\xgate^{b_1\oplus b_2}}\zgate^{a_2})\cnot, \mbox{ where } f_{\cnot}(a_1,b_1,a_2,b_2)=(a_1\oplus a_2,b_1,a_2, b_1\oplus b_2).\\
\end{aligned}
\end{equation*}
%%\anote{Note that the usual notation would be $X^aZ^b$, but you've consistently used $X^bZ^a$, which is strange, but let's go with it, since this is done already.}
\begin{remark}
Using gate teleportation we can also evaluate a  $\tgate$-gate however, the correction function becomes more complicated, since:
						\begin{equation}\
						\tgate \xgate^b \zgate^a= \xgate^b \zgate^{a\oplus b} \pgate^b \tgate 
						\end{equation}
\end{remark}	

Similarly for any $n$-qubit Clifford Circuit\footnote{A Clifford circuit is a quantum circuit in which every gate is from the Clifford group.} $\mathcal{C}_q$ we define an update function $F_{\mathcal{C}_q}:\, \mathbb{F}_2^n \rightarrow  \mathbb{F}_2^n$, that relates $\mathcal{C}_q$ and a unitary $( \xgate^{\otimes_{i=1}^{n} b_{i}} \cdot  \zgate^{\otimes_{i=1}^{n} a_{i}})$ in the following manner,
$$\mathcal{C}_q(  \xgate^{\otimes_{i=1}^{n} b_{i}} \cdot  \zgate^{\otimes_{i=1}^{n} a_{i}})=( \xgate^{\otimes_{i=1}^{n} b_{i}^\prime} \cdot  \xgate^{\otimes_{i=1}^{n} a_{i}^\prime})\mathcal{C}_q $$
%\anote{like on previous expression, I am not sure how to read this.}
where $a_i,b_i,a_i^\prime,b_i^\prime \in\{0,1\}$ for all integers  $1\leq i\leq n$ and  $F_{\mathcal{C}_q}(a_1,b_1,a_2,b_2, \ldots a_n,b_n)=(a_1^\prime,b_1^\prime, \ldots, a_n^\prime,b_n^\prime).$ The update function $F_{c_q}$ is computed by the composition of the correction functions of the gates in the circuit $\mathcal{C}_q.$ For example, if we want to evaluate a 2-qubit circuit $C_q=\cnot \cdot (\igate \otimes \hgate)$\footnote{In this paper we assume that we apply apply gates from R.H.S.} using gate teleportation, the update function for this circuit is
%
%\begin{equation*}
%\begin{aligned}
%F_{C_q}(a_1,b_1,a_2,b_2)=(b_1\oplus a_2,a_1,a_2, a_1\oplus b_2),\\
%\end{aligned}
%\end{equation*}
%And computed as by first applying the $f_\hgate$ to the first two input bits and then $f_{\cnot}$ to all four bits.
%\begin{equation*}
%\begin{aligned}
%(a_1,b_1,a_2,b_2) \xmapsto{\igate \otimes \hgate} (b_1,a_1,a_2,b_2)\xmapsto{\cnot} (b_1\oplus a_2,a_1,a_2, a_1\oplus b_2).
%\end{aligned}
%\end{equation*}







			
				





%********************

%%%%%%%%%%%%%%%%%%%%%%%%%%%%%%%%%%%%%%%%%%%%%%%%%%%%
%Section newIdea iO
%%%%%%%%%%%%%%%%%%%%%%%%%%%
%\input{newIdeaiO.tex}

%%%%%%%%%%%%%%%%%%%%%%%%%%%%%%%%%%%%%%%%%%%%%%%%%%%%
%Section Constructing iO
%%%%%%%%%%%%%%%%%%%%%%%%%%%
%********************

%%%%%%%%%%%%%%%%%%%%%%%%%%%%%%%%%%%%%%%%%%%%%%%%%%%%
%Section
\section{Quantum Indistinguishability Obfuscation}
\label{sec:QiO-Cliffords and more}
In this section we give a definition of equivalent quantum circuits and also define a notion of Quantum Indistinguishability Obfuscation $(Qi\mathcal{O})$. In  \Cref{QiO:Clifford-Circuits} we show how to construct $Qi\mathcal{O}$ for some families of quantum circuits. We give two alternative methods, the first based on canonical forms, and the second based on gate teleportation.

%\anote{Need some blah blah at the beginning of a section to say what we present in this section.}

\subsection{Definitions}
\label{def:equivalent-circuits}
\begin{definition} {\rm (Equivalent Quantum Circuits):}
Let $C_{q_0}$ and $C_{q_1}$ be two $n$-qubit quantum circuits. We say $C_{q_0}$ and $C_{q_1}$ are equivalent if for every $n$-qubit state $\ket{\psi}$
$$C_{q_1}(\ket{\psi}) =C_{q_2}(\ket{\psi}).$$
\end{definition}
%\anote{Need some blah blah before an important definition to describe what it is saying, but in words}

%\anote{why is computational in brackets below?}
\begin{definition}{\rm (Quantum Indistinguishability Obfuscation)}
\label{def:QiO}
A  polynomial-time quantum algorithm for a class of quantum circuits $\mathcal{C}_Q$ is a quantum computational indistinguishability obfuscator $Qi\mathcal{O}$  if the following conditions hold:

\begin{enumerate}
\item {\tt Functionality:}  For every $C_q\in \mathcal{C}_Q$ and every quantum state $\ket{\phi},$
$$(\rho, C_q^\prime)\leftarrow Qi\mathcal{O}(C_q) \;  \mbox{ and }\; C_q^\prime(\rho,\ket{\phi})=C_{q}(\ket{\phi})$$
%where $\rho \in Q(m)$ and $C_q^\prime$  is a $m+n$ qubit quantum circuit.
											
%\anote{above, we are mixing pure states such as $\ket{\phi}$ with mixed states, such as $\rho$. This gets confusing, but I don't know exacly how to fix it. Maybe $\rho$ could be assumed wlog to be pure, so we can use use $\ket{\psi}$ instead of $\rho?$. }

\item  {\tt Polynomial Slowdown:}  For every $C_{q}\in \mathcal{C}_Q,$
\begin{itemize}
\item  $\rho$ is at most a $poly(|C_{q}|)$ qubit state.
\item $|C_{q}^\prime| \in poly(|C_{q}|).$
\end{itemize}
%\anote{in below, did we define what we mean by size? If this is the number of gates, I would find the definition quite restrictive. Maybe you mean the same dimensions for input- output?}

\item {\tt Indistinguishability:} For any two equivalent quantum circuits $C_{q_1},C_{q_2}\in \mathcal{C}_Q,$ of the same size
 and for every polynomial-time quantum distinguisher $\mathcal{D}_q,$ there exists a negligible function {\rm negl} such that:
					$$\Big |{\rm Pr}[\mathcal{D}_q(Qi\mathcal{O}(C_{q_1}))=1]-{\rm Pr}[\mathcal{D}_q(Qi\mathcal{O}(C_{q_2}))=1] \Big |\leq  {\rm negl}(k).$$		
Where $k=|C_{q_1}|=|C_{q_2}|.$						
\end{enumerate}
\end{definition}


\subsection{Quantum Indistinguishability Obfuscation for Clifford Circuits}
\label{QiO:Clifford-Circuits}
 In this section we present two methods to obfuscate any Clifford circuit,  one using a canonical form and the other using gate teleportation.

\subsubsection{$Qi\mathcal{O}$ for Clifford via Canonical Form}
\label{sec:Clifford-iO-canonical}
Aaronson and Gottesman invented a polynomial-quantum algorithm (polynomial size circuit) that takes a Clifford circuit $C_q$ and output its canonical form~\cite{AG04} (section VI). This canonical form is invariant for any two equivalent circuits. Moreover the size of the canonical form remain $poly(|C_q|).$ Therefore, the $Qi\mathcal{O}$ in essentially the Aaronson and Gottesman algorithm that takes a Clifford circuit $C_q,$ as an input and output its canonical form $C_q^\prime$ and an empty register $\rho$ \footnote{Note we output the empty register to satisfy the definition of quantum indistinguishability obfuscation (\ref{def:QiO}).}
														$$(\rho, C_q^\prime)\leftarrow Qi\mathcal{O}(C_q).$$

 \begin{lemma}
The above construction (section \ref{sec:Clifford-iO-canonical}) is a Quantum Indistinguishability Obfuscation for Clifford Circuits.
\end{lemma}

\noindent {\bf Proof}: We have to show that the construction satisfies definition \ref{def:QiO} . Note $Qi\mathcal{O}$ is a polynomial-time quantum algorithm, since it is essentially the same a Aaronson and Gottesman a polynomial-quantum algorithm for computing a canonical form of a Clifford circuit ~\cite{AG04} (section VI).
To show the construction satisfies all 3 properties of the definition \ref{def:QiO}; let $C_q^\prime$ be the canonical form of some Clifford circuit $C_q.$  Since, $C_q^\prime$ and $C_q$ are equivalent circuits (definition \ref{def:equivalent-circuits}) they have the same function functionality. Note $\rho$ is an empty register and the size of $|C_q^\prime| \in poly(|C_q|)$ ~\cite{AG04}, hence the construction is efficient (polynomial slowdown). Moreover the canonical form reveals no knowledge about the input circuit except the unitary it computes, therefore this $Qi\mathcal{O}$ is perfectly indistinguishable against any quantum adversary. Therefore, also computationally indistinguishable against any quantum adversary.



\subsubsection{$Qi\mathcal{O}$ for Clifford via Gate Teleportation}
\label{sec:Clifford-iO-teleportaion}
In this section we  show that how gate teleportation can be used to construct a quantum indistinguishability obfuscation for Clifford circuits Our algorithm relies on the assumption that there exists a quantum secure $i\mathcal{O}$ for classical circuits, this seems problematic at first since, there is no provably quantum secure $i\mathcal{O}$ known for general classical circuits.
%\anote{Really? I thought there were some candidates.... }
However,  our construction relies on the assumption that a quantum secure $i\mathcal{O}$ exists for a very specific class of classical circuits the compute update functions (section \ref{update function}). And In fact, it is easy to construct an $i\mathcal{O}$ for classical circuits that compute update functions for Clifford circuits (appendix \ref{sec: iO-clifford-functions}). %section \ref{sec: iO-clifford-functions}). %\anote{Indeed, we're looking at some classical circuits that are probably quite easy to deal with. I'm not sure they are Cliffords, however, since I think they are classical functions.}


%\anote{below, $\rho$ is usually a density matrix, which does not require a $\ket{\cdot}$.}
%\anote{below, say something about what ``lower half'' means, or establish earlier on that another way to write the tensor product of $n$ Bell states is $\sum_i \ket{i}{i}$ (or something like this; need to check). Then it is clear which register we are talking about for applying $C_q$.}

\begin{algorithm}[H]
\label{QiO:Clifford-teleportation}
   \caption{$Qi\mathcal{O}$ using Gate Teleportation}
  \begin{itemize}
  \item Input: A $n$-qubit Clifford Circuit $C_q.$
  \begin{enumerate}
  \item Prepare $2n$ qubit Bell state $\ket{\beta^{ 2n}}=\ket{\beta_{00}}\otimes \cdots \otimes \ket{\beta_{00}}.$
  \item Apply the circuit $C_q$ on the right most $n$ qubits and obtain a system $\rho$
  										 $$\rho=(\igate_n\otimes C_q) \ket{\beta^{2n}}.$$
 \item Compute the classical circuit $C_{F_{C_q}}$ that computes the update function $F_{C_q}.$	
 \item  Set $C=i\mathcal{O}(C_{F_{C_q}}).$							
  \item Let $B_q$ denote the circuit that takes a $3n$ qubit state and perform a Bell measurement on the leftmost $2n$-qubits of the input state and obtain the bits (output of measurement), $(a_1,b_1\ldots,a_n,b_n)$
  \item Set $C_q^\prime=(B_q, C^\prime).$
  \item Output $\left(\rho,C_q^\prime \right).$
  \end{enumerate}
  \end{itemize}
\end{algorithm}

%\anote{For step 2 below, shall we add to preliminaries that classical circuits can be embedded into quantum ones? I'm not sure, actually, how you envisage this classical part of the corrections- maybe I haven't fully understood yet.}
%On receiving $\left(\rho,C_q^\prime \right),$ one can evaluate the original circuit $C_q$ on any $n$-qubit state $\ket{\psi}$ as follows

\begin{algorithm}[H]
\label{Eval:Clifford-teleportation}
\caption{Computing $C_q$ from $\left(\rho,C_q^\prime \right)$}
 \begin{itemize}
  \item Input $\left(\rho,C_q^\prime \right)$ and a quantum state $\ket{\psi}.$
  \begin{enumerate}
  \item   Write $C_q^\prime=(B_q, C)$ and the joint system $\rho_{AB}=\rho \otimes \ket{\psi}.$
  \item   Obtain the state $\rho_{AB}^\prime=B_q(\rho_{AB})$ and classical output $(a_1,b_1\ldots,a_n,b_n).$
  \item   Compute $C_q(\xgate^{\otimes_{i=1}^{n} b_{i}} \cdot \zgate^{\otimes_{i=1}^{n}})\ket{\psi}=Tr_A(\rho_{AB}^\prime)$ (see  gate teleportation protocol \ref{protocol: gate-teleportation}).
  \item   Compute the update function $$(a_1^\prime, b_1^\prime,\ldots, a_n^\prime, b_n^\prime)=C(a_1,b_1\ldots,a_n,b_n).$$
  \item    Compute the correction unitary $(\xgate^{\otimes_{i=1}^{n} b_{i}^\prime} \cdot \zgate^{\otimes_{i=1}^{n} a_{i}^\prime})$
  \item   Apply  $(\xgate^{\otimes_{i=1}^{n} b_{i}^\prime} \cdot \zgate^{\otimes_{i=1}^{n} a_{i}^\prime})$ to the system $C_q(\xgate^{\otimes_{i=1}^{n} b_{i}} \cdot \zgate^{\otimes_{i=1}^{n} a_{i}})\ket{\psi}$ to obtain the state $C_q(\ket{\psi})$ as follows,
  \begin{equation*}
  \begin{aligned}
 (\xgate^{\otimes_{i=1}^{n} b_{i}^\prime} \cdot \zgate^{\otimes_{i=1}^{n} a_{i}^\prime}) \cdot C_q(\xgate^{\otimes_{i=1}^{n} b_{i}} \cdot \zgate^{\otimes_{i=1}^{n}a_{i}})\ket{\psi}\\
 =(\xgate^{\otimes_{i=1}^{n} b_{i}^\prime} \cdot \zgate^{\otimes_{i=1}^{n} a_{i}^\prime}) \cdot (\zgate^{\otimes_{i=1}^{n} a_{i}^\prime} \cdot \xgate^{\otimes_{i=1}^{n}b_{i}^\prime})C_q(\ket{\psi})\\
 = C_q(\ket{\psi}).
 \end{aligned}
 \end{equation*}
  \end{enumerate}
  \end{itemize}
\end{algorithm}

\begin{theorem}
If $i\mathcal{O}$ is a quantum secure classical indistinguishability obfuscation, then $Qi\mathcal{O}$ constructed from the gate teleportation  a quantum Indistinguishability Obfuscation for all Clifford Circuits.
\end{theorem}

\anote{Note to self: I have not checked the below}

\noindent{\bf Proof}:
\begin{itemize}
\item  {\tt Functionality:} The functionality $Qi\mathcal{O}$ is followed from the gate teleportation (section \ref{protocol: gate-teleportation}).
\item {\tt Polynomial Slowdown:} Note $\rho$ is a $O(n)$-qubit state  and the size of $|C_q|=|B_q|+|C^\prime|.$ The circuit $B_q$ perform the Bell measurement of a $O(n)$ qubit system. Therefore, $B_q$ can be constructed with $O(n)$ gates. Note from complexity theory we know that if a function can be computed by a deterministic algorithm in time $t(n),$ then it can be computed by a circuit of size $O(t^2(n)).$ Therefore if we can show  that  ${F_{q_c}}$ can be computed in $poly(|C_q|),$ then we can conclude that the size of $|C|\in poly(|C_q|).$ Note the function $F_{C_q}$ takes  $2n$ bits as an input and perform $O(n)$ operations on the input bits for each gate in $C_{q}.$ There are at most $|C_q|$ gates, therefore the total cost of computing $F_{C_q}$ is $O(n)|C_q|.$  Note any $n$-qubit circuit $C_q,$ $n\leq |C_q|,$ the function $F_{C_q}$ can be computed by a circuit  of size at most $poly(|C_{q}|).$ Therefore,
\begin{equation*}
\rho\in poly(|C_q|)    \mbox{ and } |C_q^\prime |=|B_q|+|C| \in poly(|C_q|)
\end{equation*}

\item  {\tt Indistinguishability:} Let $C_{q_1}$ and $C_{q_2}$ be two $n$-qubit equivalent Clifford circuits
\begin{equation*}
\left(\rho_1 ,\left(B_q, i\mathcal{O}(C_{F_{C_{q_1}}})\right)\right)= Qi\mathcal{O}(C_{q_1}) \mbox{ and } \left(\rho_2 ,\left(B_q, i\mathcal{O}(C_{F_{C_{q_2}}})\right)\right)= Qi\mathcal{O}(C_{q_2})
\end{equation*}

Since $C_{q_1}(\ket{\tau})=C_{q_2}(\ket{\tau})$ for every quantum state $\ket{\tau}$ we have,

\begin{equation}
\label{eq:1}
\rho_1=(I\otimes C_{q_1}) \ket{\beta^{2n}}=(I\otimes C_{q_2}) \ket{\beta^{2n}}=\rho_2.
\end{equation}
And of course $B_q$ will always be the same circuit for circuits. So the only thing we need to show that $i\mathcal{O}(C_{F_{C_{q_1}}})$ and $i\mathcal{O}(C_{F_{C_{q_2}}})$ are computationally indistinguishable against any polynomial-time quantum adversary. Note if we can prove that $F_{C_{q_1}}$ and $F_{C_{q_2}}$ are equivalent functions then we can conclude that the classical circuits ($C_{q_1}$ and $C_{q_2}$) that computes these functions are also be equivalent. Then from the definition of classical quantum secure $i\mathcal{O}$ we can conclude that  $iO(C_{F_{C_{q_2}}})$ and $iO(C_{F_{C_{q_2}}})$ are computationally indistinguishable against any polynomial-time quantum adversary. From Theorem \ref{sec:Clifford Functions} we have $F_{C_{q_1}}$ and $F_{C_{q_2}}$ equivalent functions, therefore $Qi\mathcal{O}$ is a quantum secure indistinguishability obfuscation for all Clifford Circuits.\footnote{Note without loss of generality we can assume that $|C_{F_{C_{q_1}}}|=|C_{F_{C_{q_2}}}|,$ because we can always increase the size of a circuit by adding trivial gates to it. }
\end{itemize}


\begin{theorem}\label{sec:Clifford Functions}
Let $C_{q_1}$ and $C_{q_2}$ be two equivalent Clifford circuits and  $F_{C_{q_1}}$ and $F_{C_{q_2}}$  denote the update functions for the circuits $C_{q_1}$ and $C_{q_2},$ then $F_{C_{q_1}}({\bf s})=F_{C_{q_2}}({\bf s})$ for every ${\bf s}\in\{0,1\}^{2n}.$
\end{theorem}

\begin{flushleft}
{\bf Proof:} For any {\em n}-qubit state $\ket \psi$ we have $C_{q_1}(\ket \psi)=C_{q_2}(\ket \psi)$ and let $F_{C_{q_1}}$ and $F_{C_{q_2}}$ denote the update functions for the circuits $C_{q_1}$ and $C_{q_2}.$ Suppose $F_{C_{q_1}}\neq F_{C_{q_2}},$  then there must exist at least one binary string $a_1b_1,\ldots, a_nb_n$ such that
\begin{equation*}
F_{C_{q_1}}(a_1,b_1,\ldots,a_n,b_n)\neq F_{C_{q_2}}(a_1,b_1,\ldots,a_n,b_n)
\end{equation*}

From the above (see indistinguishability) we have (see equation \ref{eq:1})
\begin{equation*}
\left(\rho ,\left(B_q, i\mathcal{O}(C_{F_{C_{q_1}}})\right)\right)= Qi\mathcal{O}(C_{q_1}) \mbox{ and } \left(\rho ,\left(B_q, i\mathcal{O}(C_{F_{C_{q_2}}})\right)\right)= Qi\mathcal{O}(C_{q_2}).
\end{equation*}

Suppose we want to evaluate circuits on some state $\ket \psi$ follows that for both circuits $C_{q_1}$ and $C_{q_2}$ in algorithm 3 we have the same state $\rho_{AB}=\rho\otimes \ket \psi$ (see algorithm 3). Therefore, the classical output bits we obtain after performing the Bell measurement in algorithm 3 are independent of the circuits $C_{q_1}$ and $C_{q_2}$). Suppose for both circuits we obtain the same classical outputs $a_1b_1,\ldots, a_nb_n.$ From algorithm 3 we have

 \begin{equation}
  \label{eq:2}
C_q(\xgate^{\otimes_{i=1}^{n} b_{i}} \cdot \zgate^{\otimes_{i=1}^{n}})\ket{\psi}=C_q^\prime (\xgate^{\otimes_{i=1}^{n} b_{i}} \cdot \zgate^{\otimes_{i=1}^{n}})\ket{\psi}
\end{equation}
Let $s=a_1b_1,\ldots, a_nb_n$ and let

 \begin{equation}
 \label{eq:3}
F_{C_{q_1}}(s)=(a^\prime_1,b^\prime_1,\ldots, a^\prime_n,b^\prime_n) \mbox{ and } F_{C_{q_2}}(s)=(d_1^\prime,e_1^\prime,\ldots, d_n^\prime,e^\prime_n)
\end{equation}

We can rewrite equation \ref{eq:2} as (see gate teleportation \ref{protocol: gate-teleportation} or algorithm 3.
 \begin{equation}
  \label{eq:4}
(\zgate^{\otimes_{i=1}^{n} a_{i}^\prime} \cdot \xgate^{\otimes_{i=1}^{n} b_{i}^\prime})C_q(\ket{\psi})=(\zgate^{\otimes_{i=1}^{n} d_{i}^\prime} \cdot \xgate^{\otimes_{i=1}^{n} e_{i}^\prime})C_q^\prime(\ket{\psi})
\end{equation}

Since $C_q^\prime(\ket{\psi})=C_q^\prime(\ket{\psi})$ we can substitute $C_q^\prime$ for $C_q^\prime$ in equation \ref{eq:4}

\begin{equation}
  \label{eq:5}
(\zgate^{\otimes_{i=1}^{n} a_{i}^\prime} \cdot \xgate^{\otimes_{i=1}^{n} b_{i}^\prime})C_q(\ket{\psi})=(\zgate^{\otimes_{i=1}^{n} d_{i}^\prime} \cdot \xgate^{\otimes_{i=1}^{n} e_{i}^\prime})C_q(\ket{\psi})
\end{equation}

It follows from equation \ref{eq:5} that
\begin{equation}
  \label{eq:6}
(\zgate^{\otimes_{i=1}^{n} a_{i}^\prime} \cdot \xgate^{\otimes_{i=1}^{n} b_{i}^\prime})=(\zgate^{\otimes_{i=1}^{n} d_{i}^\prime} \cdot \xgate^{\otimes_{i=1}^{n} e_{i}^\prime})
\end{equation}

It follows from equation \ref{eq:6}  that $$\zgate^{a_i^\prime} \cdot \xgate^{b_i^\prime}=\zgate^{d_i^\prime} \cdot \xgate^{e_i^\prime} \mbox{ for all } i\in[n].$$ This means $a_i^\prime=d_i^\prime$ and  $b_i^\prime=e_i^\prime$ and therefore, if $C_{q_1}$ and $C_{q_2},$ are two equivalent circuits, then they have the same update function.
\end{flushleft}





%\subsection{Indistinguishability Obfuscation for the Clifford Update Functions}
%\label{sec: iO-clifford-functions}
%
%
%\anote{I am not sure how am I to understand that the $F_{C_q}$ are Cliffords.... I thought they were classical circuits?}
%
%In this section we present a (quantum secure classical) $i\mathcal{O}$ that can obfuscate the update function $F_{C_q},$ for any $n$-qubit Clifford circuit $C_q.$
%
%\begin{algorithm}[H]
%   \caption{$i\mathcal{O}$ for Clifford update Functions $F_{\tt Clifford}$}
%  \begin{itemize}
%  \item Input: A $n$-qubit Clifford Circuit $C_q,$ a security parameter.
%  \begin{enumerate}
%  \item Compute $C_{q_c}\leftarrow$ {\tt AG-Canonical-Unique}$(C_q).$
%  \item Compute the circuit $C_{F_{q_c}}$ for the update function $F_{C_{q_c}}.$
%  \item Output $C_{F_{q_c}}.$
%  \end{enumerate}
%  \end{itemize}
%\end{algorithm}

%********************

%%%%%%%%%%%%%%%%%%%%%%%%%%%%%%%%%%%%%%%%%%%%%%%%%%%%
%Section Application of quantum-secure classical iO
%%%%%%%%%%%%%%%%%%%%%%%%%%%
%%%%%%%%%%%%%%%%%%%%%%%%%%%%%%%%%%%%%%%%%%%%%%%%%%%%%
%Section
\section{Quantum Public-Key Encryption from iO}
\label{sec:Quantum-public-key}
Here, \emph{quantum public-key encryption} refers to the  encryption of a quantum register using a classical public key, see Section~\ref{sec:prelim-quantum-PKE} for related definitions.


\subsection{Obfuscation-based Scheme}
The scheme is based on an idea for ``classical public-key encryption based from indistinguishability obfuscation'' see~\cite{SW14}.
\label{sec:obfuscation}
\begin{itemize}
\item Common Parameters:
\begin{enumerate}
\item Let  ${\rm G}: \{0,1\}^n\rightarrow \{0,1\}^{2n}$ be a quantum secure pseudorandom number generator.
\item Let  $F_k:\{0,1\}^{2n} \times \{0,1\}^{2n} \longrightarrow \{0,1\}^{2n}$ be a quantum secure puncturable pseudorandom function.
\item Let $qi\mathcal{O}$ be a quantum secure classical indistinguishability obfuscation
\item Plaintext space  $\mathcal{D}(\mathcal{H}_{M}),$ where $M=2n$ and Ciphertext space $\mathcal{H}_C=\mathcal{D}(\mathcal{H}_{M})\otimes \mathcal{D}(\mathcal{H}_{M}).$
\end{enumerate}
\end{itemize}

\begin{itemize}
\item ${\rm KeyGen}(1^n)$
\begin{enumerate}
\item  Pick a random key $k\rightarrow \{0,1\}^{2n}$ and set $k$ as the puncturable qPRF key.
\item Construct  program $P_k(s)$ that takes an $n$ bit string as an input and the key $k$ is hardcoded in $P_k.$
\begin{itemize}
\item[] $P_k(s) \{$
\begin{itemize}
\item[] $r\leftarrow {\rm G}(s)$
\item[] $y \leftarrow F_k(r)$
\item[] return $(r,y).$
\end{itemize}
\item[] \}
\end{itemize}
\item Set public key $pk=qi\mathcal{O}(P_k,1^n)$ and the secret key $sk=k.$
\item Output $(pk,sk).$
\end{enumerate}
\end{itemize}


\begin{itemize}
\item ${\rm Enc}(qi\mathcal{O}(P_k,1^n), \rho):$
\begin{enumerate}
\item Pick a random string $s\rightarrow \{0,1\}^n.$
\item Set $(y,r)=qi\mathcal{O}(P_k,1^n)(s).$
\item  Output the ciphertext $\ket{r}\bra{r} \otimes P_y \rho P_y.$
\end{enumerate}
\end{itemize}

\begin{itemize}
\item ${\rm Dec}(k, \sigma):$
\begin{enumerate}
\item Measure the first $2n$ qubits in the computational basis to obtain $r^\prime \in \{0,1\}^n.$
\item Compute $y^\prime=F_k(r^\prime).$
\item Output the plaintext $P_{y^\prime}\sigma P_{y^\prime}.$
\end{enumerate}
\end{itemize}

\begin{theorem}
If $G$ is a quantum secure qPRNG, $F_k$ is a quantum secure puncturable qPRF  and $qi\mathcal{O}$ is quantum secure classical obfuscator, then Scheme \ref{sec:obfuscation} is an IND-CPA secure
 quantum public key encryption scheme.
\end{theorem}

\noindent{\bf Proof}. Our proof is consists of a sequence of hybrid experiments where the first hybrid corresponds to the
IND-CPA experiment, in which a challenger{\em C} interacts with a polynomial-time adversary $\mathcal{A}=(\mathcal{A}_1, \mathcal{A}_2).$ We prove that adversaries's advantage must be the negligibly close between each successive hybrid and that the adversary has zero advantage in the last experiment. Hence, concluding that for any polynomial-time adversary the advantage of winning IND-CPA  experiment is at most negligible.

\begin{itemize}
\item ${\bf Hybrid_0}$
\begin{enumerate}
\item {\em C} generates pair of secret/ public keys $(k, qi\mathcal{O}(P_k)) \leftarrow{\rm KeyGen}(1^n).$
\item $\mathcal{A}_1$ obtained the public-key $qi\mathcal{O}(P_k)$ from {\em C}
\item $\mathcal{A}_1$ outputs a quantum state  $\rho_{ME}$ on  $\mathcal{D}(\mathcal{H}_{2n}) \otimes \mathcal{E}.$
\item {\em C} computes $c_0={\rm Enc}(pk,\ket{0}\bra{0}_\mathcal{M}), c_1={\rm Enc}(pk,\rho_M),$ $r^* \xleftarrow[]{\$} \bit{}$ and outputs $c_i=\ket{r}\bra{r} \otimes P_y \rho P_y,$ where $\rho=\ket{0}\bra{0}_\mathcal{M}$ if $i=0$ and  $\rho=\rho_M$ otherwise.
\item  $\mathcal{A}_2$ obtains the challenge ciphertext $\ket{r}\bra{r} \otimes P_y \rho P_y$ from{\em C} and $\rho_E$ from $\mathcal{A}_1.$
\end{enumerate}
\end{itemize}

\begin{itemize}
\item${\bf Hybrid_1} \hspace{-0.12cm}:$ This hybrid is similar to $\mbox{Hybrid}_0$ except the ciphertext is given as $\ket{r^*}\bra{r^*} \otimes P_{f_k(r^*)} \rho P_{f_k(r^*)},$ where $r^* \xleftarrow[]{\$} \bit{2n}.$
\begin{enumerate}
\item {\em C} generates pair of secret/public keys $(k, qi\mathcal{O}(P_k)) \leftarrow{\rm KeyGen}(1^n).$
\item $\mathcal{A}_1$ obtained the public-key $qi\mathcal{O}(P_k)$ from{\em C}
\item $\mathcal{A}_1$ picks a quantum state  $\rho_{ME}$ on  $\mathcal{D}(\mathcal{H}_{2n}) \otimes \mathcal{E}$ and outputs $\rho_{M}.$
\item{\em C} picks $r^* \xleftarrow[]{\$} \bit{n},\, i \xleftarrow[]{\$} \bit{}$  and output the ciphertext $\ket{r^*}\bra{r^*} \otimes P_{f_k(r^*)} \rho P_{f_k(r^*)},$ where $\rho=\ket{0}\bra{0}_\mathcal{M}$ if $i=0$ and  $\rho=\rho_M$ otherwise.
\item  $\mathcal{A}_2$ obtains the challenge ciphertext $\ket{r^*}\bra{r^*} \otimes P_{f_k(r^*)} \rho P_{f_k(r^*)}$ from{\em C} and $\rho_E$ from $\mathcal{A}_1.$
\end{enumerate}
\end{itemize}



\begin{itemize}
\item ${\bf Hybrid_2}:$ In this hybrid we replace the public key in hybrid 1 by the obfuscation of the program $P_{k\{r^*\}}$ (define below).
\begin{enumerate}
\item{\em C} picks a random key $k\rightarrow \{0,1\}^{2n}$ and constructs a program $P_{k\{r^*\}}.$
\begin{itemize}
\item[] $P_{k\{r^*\}}(s) \{$
\begin{itemize}
\item[] $r\leftarrow {\rm G}(s)$
\item[] $y \leftarrow \textsf{Eval}_F(k\{r^*\},r)$
\item[] return $(r,\textsf{Eval}_F(k\{r^*\}).$
\end{itemize}
\item[] \}
\end{itemize}
 \item Set the secret key $sk=k$ and the public key $pk^*=qi\mathcal{O}(P_{k\{r^*\}},1^n).$
 \item  $\mathcal{A}_1$ obtains $pk$ from{\em C} and outputs a quantum state  $\rho_{ME}$ on $\mathcal{D}(\mathcal{H}_{2n}) \otimes \mathcal{E}.$
 \item {\em C} picks $s\xleftarrow[]{\$} \bit{2n}$ and obtain $(r,y)= qi\mathcal{O}(P_{k\{r^*\}},1^n)(s).$
 \item{\em C} picks $r^* \xleftarrow[]{\$} \bit{n},\, i \xleftarrow[]{\$} \bit{}$ and output the ciphertext $c_i=\ket{r^*}\bra{r^*} \otimes P_y \rho P_y,$ where $\rho=\ket{0}\bra{0}_\mathcal{M}$ if $i=0$ and  $\rho=\rho_M$ otherwise.
\item  $\mathcal{A}_2$ obtains the challenge ciphertext $\ket{r^*}\bra{r^*} \otimes P_y \rho P_y$ from{\em C} and $\rho_E$ from $\mathcal{A}_1.$
\end{enumerate}
\end{itemize}

\begin{itemize}
\item ${\bf Hybrid_3}\hspace{-0.09cm}:$ This hybrid is similar to $\mbox{Hybrid}_2$ except the ciphertext is given as $\ket{r^*}\bra{r^*} \otimes P_{y^*} \rho P_{y^*}.$ Where $y^* \xleftarrow[]{\$} \bit{2n}.$
\end{itemize}

\begin{itemize}
\item ${\bf Hybrid_0}$ and ${\bf Hybrid_1}\hspace{-0.09cm}:$ If there is a polynomial-time quantum adversary can distinguish between the two hybirds, then there exists a  polynomial-time quantum adversary that can break the security of PRNG. Suppose $\mathcal{A}$ on input  $\ket{G(s)}\bra{G(s)} \otimes P_{G(s)} \rho P_{G(s)}$  outputs 0 and on input $\ket{r^*}\bra{r^*} \otimes P_{f_k(r^*)} \rho P_{f_k(r^*)}$  outputs 1 with non-negligible advantage, then we construct polynomial-time quantum adversary $\mathcal{B}$ that can break the security of PRNG G.

 \begin{enumerate}
\item $\mathcal{B}$ runs the PRNG experiment and obtain a challenge $z.$
\item $\mathcal{B}$ then runs  $\mbox{Hybrid}_0$ and obtain the plaintext $\rho_M.$
\item $\mathcal{B}$ picks a random bit $i \xleftarrow[]{\$} \bit{}$ and sets the ciphertext $c_i=\ket{z}\bra{z} \otimes P_z \rho P_z,$ where $\rho=\ket{0}\bra{0}_\mathcal{M}$ if $i=0$ and  $\rho=\rho_M$ otherwise.
\item  $\mathcal{B}$ runs $\mathcal{A}$ on the input $c_i.$ If $\mathcal{A}$ outputs bit $j,$ then $\mathcal{B}$ outputs $j.$ Note $j=0$ corresponds to ${\bf Hybrid_o}$ which means $z$ is the output of PRNG and $j=1$ corresponds to ${\bf Hybrid_1}$ which means $z$ is randomly picked. Therefore, if $\mathcal{A}$ exist then we can construct $\mathcal{B}$ breaks the security of Puncturable PRF with non-negligible advantage.
\end{enumerate}
\end{itemize}




\begin{itemize}
\item ${\bf Hybrid_1}$ and ${\bf Hybrid_2}\hspace{-0.09cm}:$ In both of these hydrids $r^*$ is picked randomly. All we have to show that  $P_{k\{r^*\}}$ and $P_{k}$ have same functionality. Note if $r^*\notin \{G(s) \mid s\in \{0,1\}^n\}$ then both $P_{k\{r^*\}}$ and $P_{k}$ will have the same functionality. When $r^*$ is picked randomly from $\bit{2n}$ then with probability $\frac{1}{2^n},$  $r^*\in \{G(s) \mid s\in \{0,1\}^n\}$ is $1-\frac{1}{2^n}.$ Therefore with overwhelming probability both programs will have same functionality. We can apply apply iO security to conclude that $qi\mathcal{O}(P_k,1^n)$ and $qi\mathcal{O}(P_{k\{r^*\}},1^n)$ are computationally indistinguishable.
\end{itemize}


\begin{itemize}
\item ${\bf Hybrid_2}$ and ${\bf Hybrid_3}\hspace{-0.09cm}:$ The ciphertext in $\mbox{Hybrid}_2$ is given as $$\ket{r^*}\bra{r^*} \otimes P_{f_k(r^*)} \rho P_{f_k(r^*)}$$ and the ciphertext in $\mbox{Hybrid}_3$ is given as $$\ket{r^*}\bra{r^*} \otimes P_{y^*} \rho P_{y^*},$$ where $r^*\xleftarrow[]{\$} \bit{2n}$ and $y^*\xleftarrow[]{\$} \bit{2n}.$ If there exists is a polynomial-time adversary $\mathcal{A}$ that can distinguish between the two hybrids with non-negligible advantage, i.e

\[
    \mathcal{A}(1^n, (\ket{r^*}\bra{r^*} \otimes P_{y} \rho P_{y} ))=
\begin{cases}
    0, & \text{if } y\xleftarrow[]{\$} \bit{2n}.\\
    1, & \text{if }  y=F_k(r^*),
\end{cases}
\]
then we can construct a polynomial time adversary $\mathcal{B}$ that can break the Puncturable PRF security.

\begin{enumerate}
\item $\mathcal{B}$ picks a random secret key $k\rightarrow \{0,1\}^{2n}$ and set public key as  $qi\mathcal{O}(P_{k\{r^*\}},1^n).$
\item $\mathcal{B}$ runs the experiment $\textsf{PunctPRF}_{D,F}(n)$ and obtain $(k\{r^*\}, y),$ where $y=f_k(r^*)$ or $y=y\xleftarrow[]{\$} \bit{2n}.$
\item $\mathcal{B}$ then  run  $\mbox{Hybrid}_2$  and obtain the ciphertext $\ket{r^*}\bra{r^*} \otimes P_{f_k(r^*)} \rho P_{f_k(r^*)}.$
\item $\mathcal{B}$ creates modified ciphertext  as $\ket{r^*}\bra{r^*} \otimes P_{y} \rho^\prime P_{y}.$
\item  $\mathcal{B}$ runs $\mathcal{A}$ on the input  $\ket{r^*}\bra{r^*} \otimes P_{y} \rho^\prime P_y.$
\item $\mathcal{B}$ outputs whatever the bit outputs by $\mathcal{A}.$  Note if $y$ is the value of PRF at the punctured point, then we are in hybrid 2 and $\mathcal{A}$ outputs 1 with non-negligible advantage. Otherwise $y$ is random and we are in hybrid 3 and $\mathcal{A}$ outputs 0 with non-negligible advantage. Therefore, $\mathcal{B}$ breaks the security of Puncturable PRF.
\end{enumerate}
\end{itemize}

\begin{itemize}
\item ${\bf Hybrid_3}\hspace{-0.09cm}:$ The ciphertext in this experiment leaks no information about the plaintext. Since the advantage of all polynomial-time attacker's are negligibly close in each successive hybrid, this proves that the  scheme \ref{sec:obfuscation} is IND-CPA secure.
\end{itemize}

\subsection{Instantiation of Quantum Public-Key Cryptosystems}

{\color{red} So far all seems to be broken. Need to work on this.}
%This scheme is based on an idea for ``classical public-key encryption based from indistinguishability obfuscation'' (see Sahai and Water. \emph{How to Use Indistinguishability Obfuscation: Deniable Encryption, and More} \url{https://eprint.iacr.org/2013/454.pdf}).
%
%The public key consists in an obfuscation of $f(r) = (PRG(r), PRF(K, PRG(r))$, where $PRF$ is a pseudo-random function, $PRG$ is a pseudo-random generator.
%
%The encryption procedure consists in computing:  $a = PRF(K, PRG(x))$ and $b =   PRF(K, PRG(y))$ for randomly-chosen $x, y$.
%
%Next, we encrypt register $R$ with $Z^a Z^b$. The full ciphertext consists in $PRG(x), PRG(y)$ as well as register $R$.
%
%The security of this scheme depends on the existence of a quantum-secure \emph{indistinguishability obfuscator}, as well as on the $PRF$ and $PRG$, I guess.
%
%Actually, the scheme can be more abstractly defined as Section 5.1 of Sahai-Waters. The description above follows along the lines of the more intuitive description they give in the introduction.
%
%The proof, I anticipated would consists in a nice hybrid argument, using a technique of \emph{puncture programs} as introduced by Sahai-Waters. All of this would need to be adapted to the quantum case, see their Theorem 4.




%%%%%%%%%%%%%%%%%%%%%%%%%%%%%%%%%%%%%%%%%%%%%%%%%%%%

\section*{Acknowledgements}
This material is based upon work supported by the Air Force Office of Scientific Research under award number FA9550-17-1-0083, Canada's   NFRF and NSERC, an Ontario ERA, and the University of Ottawa’s Research Chairs program.


%\bibliographystyle{alpha}
%\bibliography{full.bib}
%\bibliography{quantum.bib}

%\bibliography{more.bib}
%\bibliographystyle{./bibtex/bst/alphaarxiv}
%\bibliography{bibtex/bib/full.bib,bibtex/bib/quantum.bib}

%add new entries to more.bib, they will be cleaned and entered into quantum.bib by the BibMaster
\bibliographystyle{bst/alphaarxiv}
\bibliography{bib/full,bib/quantum,bib/more}

\end{document} 