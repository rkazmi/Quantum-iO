
%%%%%%%%%%%%%%%%%%%%%%%%%%%%%%%%%%%%%%%%%%%%%%%%%%%%
%Section
\section{Quantum Indistinguishability Obfuscation}
\label{sec:QiO-Cliffords and more}
In this section we give a definition of equivalent quantum circuits and also define a notion of Quantum Indistinguishability Obfuscation $Qi\mathcal{O}.$ In section \ref{QiO:Clifford-Circuits} we show how to construct $Qi\mathcal{O}$ for some families of quantum circuits. We give two alternative methods, the first based on canonical forms, and the second based on gate teleportation.

%\anote{Need some blah blah at the beginning of a section to say what we present in this section.}

\subsection{Definitions}
\label{def:equivalent-circuits}
\begin{definition} {\rm (Equivalent Quantum Circuits):}
Let $C_{q_0}$ and $C_{q_1}$ be two $n$-qubit quantum circuits. We say $C_{q_0}$ and $C_{q_1}$ are equivalent if for every $n$-qubit state $\ket{\psi}$
$$C_{q_1}(\ket{\psi}) =C_{q_2}(\ket{\psi}).$$
\end{definition}
%\anote{Need some blah blah before an important definition to describe what it is saying, but in words}

%\anote{why is computational in brackets below?}
\begin{definition}{\rm (Quantum Indistinguishability Obfuscation)}
\label{def:QiO}
A  polynomial-time quantum algorithm for a class of quantum circuits $\mathcal{C}_Q$ is a quantum computational indistinguishability obfuscator $Qi\mathcal{O}$  if the following conditions hold:

\begin{enumerate}
\item {\tt Functionality:}  For every $C_q\in \mathcal{C}_Q$ and every quantum state $\ket{\phi},$
$$(\rho, C_q^\prime)\leftarrow Qi\mathcal{O}(C_q) \;  \mbox{ and }\; C_q^\prime(\rho,\ket{\phi})=C_{q}(\ket{\phi})$$
%where $\rho \in Q(m)$ and $C_q^\prime$  is a $m+n$ qubit quantum circuit.
											
%\anote{above, we are mixing pure states such as $\ket{\phi}$ with mixed states, such as $\rho$. This gets confusing, but I don't know exacly how to fix it. Maybe $\rho$ could be assumed wlog to be pure, so we can use use $\ket{\psi}$ instead of $\rho?$. }

\item  {\tt Polynomial Slowdown:}  For every $C_{q}\in \mathcal{C}_Q,$
\begin{itemize}
\item  $\rho$ is at most a $poly(|C_{q}|)$ qubit state.
\item $|C_{q}^\prime| \in poly(|C_{q}|).$
\end{itemize}
%\anote{in below, did we define what we mean by size? If this is the number of gates, I would find the definition quite restrictive. Maybe you mean the same dimensions for input- output?}

\item {\tt Indistinguishability:} For any two equivalent quantum circuits $C_{q_1},C_{q_2}\in \mathcal{C}_Q,$ of the same size
 and for every polynomial-time quantum distinguisher $\mathcal{D}_q,$ there exists a negligible function {\rm negl} such that:
					$$\Big |{\rm Pr}[\mathcal{D}_q(Qi\mathcal{O}(C_{q_1}))=1]-{\rm Pr}[\mathcal{D}_q(Qi\mathcal{O}(C_{q_2}))=1] \Big |\leq  {\rm negl}(k).$$		
Where $k=|C_{q_1}|=|C_{q_2}|.$						
\end{enumerate}
\end{definition}


\subsection{Quantum Indistinguishability Obfuscation for Clifford Circuits}
\label{QiO:Clifford-Circuits}
 In this section we present two methods to obfuscate any Clifford circuit,  one using a canonical form and the other using gate teleportation.

\subsubsection{$Qi\mathcal{O}$ for Clifford via Canonical Form}
\label{sec:Clifford-iO-canonical}
Aaronson and Gottesman invented a polynomial-quantum algorithm (polynomial size circuit) that takes a Clifford circuit $C_q$ and output its canonical form ~\cite{AG04} (section VI). This canonical form is invariant for any two equivalent circuits. Moreover the size of the canonical form remain $poly(|C_q|).$ Therefore, the $Qi\mathcal{O}$ in essentially the Aaronson and Gottesman algorithm that takes a Clifford circuit $C_q,$ as an input and output its canonical form $C_q^\prime$ and an empty register $\rho$ \footnote{Note we output the empty register to satisfy the definition of quantum indistinguishability obfuscation (\ref{def:QiO}).}
														$$(\rho, C_q^\prime)\leftarrow Qi\mathcal{O}(C_q).$$

 \begin{lemma}
The above construction (section \ref{sec:Clifford-iO-canonical}) is a Quantum Indistinguishability Obfuscation for Clifford Circuits.
\end{lemma}

\noindent {\bf Proof}: We have to show that the construction satisfies definition \ref{def:QiO} . Note $Qi\mathcal{O}$ is a polynomial-time quantum algorithm, since it is essentially the same a Aaronson and Gottesman a polynomial-quantum algorithm for computing a canonical form of a Clifford circuit ~\cite{AG04} (section VI).
To show the construction satisfies all 3 properties of the definition \ref{def:QiO}; let $C_q^\prime$ be the canonical form of some Clifford circuit $C_q.$  Since, $C_q^\prime$ and $C_q$ are equivalent circuits (definition \ref{def:equivalent-circuits}) they have the same function functionality. Note $\rho$ is an empty register and the size of $|C_q^\prime| \in poly(|C_q|)$ ~\cite{AG04}, hence the construction is efficient (polynomial slowdown). Moreover the canonical form reveals no knowledge about the input circuit except the unitary it computes, therefore this $Qi\mathcal{O}$ is perfectly indistinguishable against any quantum adversary. Therefore, also computationally indistinguishable against any quantum adversary.



\subsubsection{$Qi\mathcal{O}$ for Clifford via Gate Teleportation}
\label{sec:Clifford-iO-teleportaion}
In this section we  show that how gate teleportation can be used to construct a quantum indistinguishability obfuscation for Clifford circuits Our algorithm relies on the assumption that there exists a quantum secure $i\mathcal{O}$ for classical circuits, this seems problematic at first since, there is no provably quantum secure $i\mathcal{O}$ known for general classical circuits.
%\anote{Really? I thought there were some candidates.... }
However,  our construction relies on the assumption that a quantum secure $i\mathcal{O}$ exists for a very specific class of classical circuits the compute update functions (section \ref{update function}). And In fact, it is easy to construct an $i\mathcal{O}$ for classical circuits that compute update functions for Clifford circuits (appendix \ref{sec: iO-clifford-functions}). %section \ref{sec: iO-clifford-functions}). %\anote{Indeed, we're looking at some classical circuits that are probably quite easy to deal with. I'm not sure they are Cliffords, however, since I think they are classical functions.}


%\anote{below, $\rho$ is usually a density matrix, which does not require a $\ket{\cdot}$.}
%\anote{below, say something about what ``lower half'' means, or establish earlier on that another way to write the tensor product of $n$ Bell states is $\sum_i \ket{i}{i}$ (or something like this; need to check). Then it is clear which register we are talking about for applying $C_q$.}

\begin{algorithm}[H]
\label{QiO:Clifford-teleportation}
   \caption{$Qi\mathcal{O}$ using Gate Teleportation}
  \begin{itemize}
  \item Input: A $n$-qubit Clifford Circuit $C_q.$
  \begin{enumerate}
  \item Prepare $2n$ qubit Bell state $\ket{\beta^{ 2n}}=\ket{\beta_{00}}\otimes \cdots \otimes \ket{\beta_{00}}.$
  \item Apply the circuit $C_q$ on the right most $n$ qubits and obtain a system $\rho$
  										 $$\rho=(\igate_n\otimes C_q) \ket{\beta^{2n}}.$$
 \item Compute the classical circuit $C_{F_{C_q}}$ that computes the update function $F_{C_q}.$	
 \item  Set $C=i\mathcal{O}(C_{F_{C_q}}).$							
  \item Let $B_q$ denote the circuit that takes a $3n$ qubit state and perform a Bell measurement on the leftmost $2n$-qubits of the input state and obtain the bits (output of measurement), $(a_1,b_1\ldots,a_n,b_n)$
  \item Set $C_q^\prime=(B_q, C^\prime).$
  \item Output $\left(\rho,C_q^\prime \right).$
  \end{enumerate}
  \end{itemize}
\end{algorithm}

%\anote{For step 2 below, shall we add to preliminaries that classical circuits can be embedded into quantum ones? I'm not sure, actually, how you envisage this classical part of the corrections- maybe I haven't fully understood yet.}
%On receiving $\left(\rho,C_q^\prime \right),$ one can evaluate the original circuit $C_q$ on any $n$-qubit state $\ket{\psi}$ as follows

\begin{algorithm}[H]
\label{Eval:Clifford-teleportation}
\caption{Computing $C_q$ from $\left(\rho,C_q^\prime \right)$}
 \begin{itemize}
  \item Input $\left(\rho,C_q^\prime \right)$ and a quantum state $\ket{\psi}.$
  \begin{enumerate}
  \item   Write $C_q^\prime=(B_q, C)$ and the joint system $\rho_{AB}=\rho \otimes \ket{\psi}.$
  \item   Obtain the state $\rho_{AB}^\prime=B_q(\rho_{AB})$ and classical output $(a_1,b_1\ldots,a_n,b_n).$
  \item   Compute $C_q(\xgate^{\otimes_{i=1}^{n} b_{i}} \cdot \zgate^{\otimes_{i=1}^{n}})\ket{\psi}=Tr_A(\rho_{AB}^\prime)$ (see  gate teleportation protocol \ref{protocol: gate-teleportation}).
  \item   Compute the update function $$(a_1^\prime, b_1^\prime,\ldots, a_n^\prime, b_n^\prime)=C(a_1,b_1\ldots,a_n,b_n).$$
  \item    Compute the correction unitary $(\xgate^{\otimes_{i=1}^{n} b_{i}^\prime} \cdot \zgate^{\otimes_{i=1}^{n} a_{i}^\prime})$
  \item   Apply  $(\xgate^{\otimes_{i=1}^{n} b_{i}^\prime} \cdot \zgate^{\otimes_{i=1}^{n} a_{i}^\prime})$ to the system $C_q(\xgate^{\otimes_{i=1}^{n} b_{i}} \cdot \zgate^{\otimes_{i=1}^{n} a_{i}})\ket{\psi}$ to obtain the state $C_q(\ket{\psi})$ as follows,
  \begin{equation*}
  \begin{aligned}
 (\xgate^{\otimes_{i=1}^{n} b_{i}^\prime} \cdot \zgate^{\otimes_{i=1}^{n} a_{i}^\prime}) \cdot C_q(\xgate^{\otimes_{i=1}^{n} b_{i}} \cdot \zgate^{\otimes_{i=1}^{n}a_{i}})\ket{\psi}\\
 =(\xgate^{\otimes_{i=1}^{n} b_{i}^\prime} \cdot \zgate^{\otimes_{i=1}^{n} a_{i}^\prime}) \cdot (\zgate^{\otimes_{i=1}^{n} a_{i}^\prime} \cdot \xgate^{\otimes_{i=1}^{n}b_{i}^\prime})C_q(\ket{\psi})\\
 = C_q(\ket{\psi}).
 \end{aligned}
 \end{equation*}
  \end{enumerate}
  \end{itemize}
\end{algorithm}

\begin{theorem}
If $i\mathcal{O}$ is a quantum secure classical indistinguishability obfuscation, then $Qi\mathcal{O}$ constructed from the gate teleportation  a quantum Indistinguishability Obfuscation for all Clifford Circuits.
\end{theorem}

\anote{Note to self: I have not checked the below}

\noindent{\bf Proof}:
\begin{itemize}
\item  {\tt Functionality:} The functionality $Qi\mathcal{O}$ is followed from the gate teleportation (section \ref{protocol: gate-teleportation}).
\item {\tt Polynomial Slowdown:} Note $\rho$ is a $O(n)$-qubit state  and the size of $|C_q|=|B_q|+|C^\prime|.$ The circuit $B_q$ perform the Bell measurement of a $O(n)$ qubit system. Therefore, $B_q$ can be constructed with $O(n)$ gates. Note from complexity theory we know that if a function can be computed by a deterministic algorithm in time $t(n),$ then it can be computed by a circuit of size $O(t^2(n)).$ Therefore if we can show  that  ${F_{q_c}}$ can be computed in $poly(|C_q|),$ then we can conclude that the size of $|C|\in poly(|C_q|).$ Note the function $F_{C_q}$ takes  $2n$ bits as an input and perform $O(n)$ operations on the input bits for each gate in $C_{q}.$ There are at most $|C_q|$ gates, therefore the total cost of computing $F_{C_q}$ is $O(n)|C_q|.$  Note any $n$-qubit circuit $C_q,$ $n\leq |C_q|,$ the function $F_{C_q}$ can be computed by a circuit  of size at most $poly(|C_{q}|).$ Therefore,
\begin{equation*}
\rho\in poly(|C_q|)    \mbox{ and } |C_q^\prime |=|B_q|+|C| \in poly(|C_q|)
\end{equation*}

\item  {\tt Indistinguishability:} Let $C_{q_1}$ and $C_{q_2}$ be two $n$-qubit equivalent Clifford circuits
\begin{equation*}
\left(\rho_1 ,\left(B_q, i\mathcal{O}(C_{F_{C_{q_1}}})\right)\right)= Qi\mathcal{O}(C_{q_1}) \mbox{ and } \left(\rho_2 ,\left(B_q, i\mathcal{O}(C_{F_{C_{q_2}}})\right)\right)= Qi\mathcal{O}(C_{q_2})
\end{equation*}

Since $C_{q_1}(\ket{\tau})=C_{q_2}(\ket{\tau})$ for every quantum state $\ket{\tau}$ we have,

\begin{equation}
\label{eq:1}
\rho_1=(I\otimes C_{q_1}) \ket{\beta^{2n}}=(I\otimes C_{q_2}) \ket{\beta^{2n}}=\rho_2.
\end{equation}
And of course $B_q$ will always be the same circuit for circuits. So the only thing we need to show that $i\mathcal{O}(C_{F_{C_{q_1}}})$ and $i\mathcal{O}(C_{F_{C_{q_2}}})$ are computationally indistinguishable against any polynomial-time quantum adversary. Note if we can prove that $F_{C_{q_1}}$ and $F_{C_{q_2}}$ are equivalent functions then we can conclude that the classical circuits ($C_{q_1}$ and $C_{q_2}$) that computes these functions are also be equivalent. Then from the definition of classical quantum secure $i\mathcal{O}$ we can conclude that  $iO(C_{F_{C_{q_2}}})$ and $iO(C_{F_{C_{q_2}}})$ are computationally indistinguishable against any polynomial-time quantum adversary. From Theorem \ref{sec:Clifford Functions} we have $F_{C_{q_1}}$ and $F_{C_{q_2}}$ equivalent functions, therefore $Qi\mathcal{O}$ is a quantum secure indistinguishability obfuscation for all Clifford Circuits.\footnote{Note without loss of generality we can assume that $|C_{F_{C_{q_1}}}|=|C_{F_{C_{q_2}}}|,$ because we can always increase the size of a circuit by adding trivial gates to it. }
\end{itemize}


\begin{theorem}\label{sec:Clifford Functions}
Let $C_{q_1}$ and $C_{q_2}$ be two equivalent Clifford circuits and  $F_{C_{q_1}}$ and $F_{C_{q_2}}$  denote the update functions for the circuits $C_{q_1}$ and $C_{q_2},$ then $F_{C_{q_1}}({\bf s})=F_{C_{q_2}}({\bf s})$ for every ${\bf s}\in\{0,1\}^{2n}.$
\end{theorem}

\begin{flushleft}
{\bf Proof:} For any {\em n}-qubit state $\ket \psi$ we have $C_{q_1}(\ket \psi)=C_{q_2}(\ket \psi)$ and let $F_{C_{q_1}}$ and $F_{C_{q_2}}$ denote the update functions for the circuits $C_{q_1}$ and $C_{q_2}.$ Suppose $F_{C_{q_1}}\neq F_{C_{q_2}},$  then there must exist at least one binary string $a_1b_1,\ldots, a_nb_n$ such that
\begin{equation*}
F_{C_{q_1}}(a_1,b_1,\ldots,a_n,b_n)\neq F_{C_{q_2}}(a_1,b_1,\ldots,a_n,b_n)
\end{equation*}

From the above (see indistinguishability) we have (see equation \ref{eq:1})
\begin{equation*}
\left(\rho ,\left(B_q, i\mathcal{O}(C_{F_{C_{q_1}}})\right)\right)= Qi\mathcal{O}(C_{q_1}) \mbox{ and } \left(\rho ,\left(B_q, i\mathcal{O}(C_{F_{C_{q_2}}})\right)\right)= Qi\mathcal{O}(C_{q_2}).
\end{equation*}

Suppose we want to evaluate circuits on some state $\ket \psi$ follows that for both circuits $C_{q_1}$ and $C_{q_2}$ in algorithm 3 we have the same state $\rho_{AB}=\rho\otimes \ket \psi$ (see algorithm 3). Therefore, the classical output bits we obtain after performing the Bell measurement in algorithm 3 are independent of the circuits $C_{q_1}$ and $C_{q_2}$). Suppose for both circuits we obtain the same classical outputs $a_1b_1,\ldots, a_nb_n.$ From algorithm 3 we have

 \begin{equation}
  \label{eq:2}
C_q(\xgate^{\otimes_{i=1}^{n} b_{i}} \cdot \zgate^{\otimes_{i=1}^{n}})\ket{\psi}=C_q^\prime (\xgate^{\otimes_{i=1}^{n} b_{i}} \cdot \zgate^{\otimes_{i=1}^{n}})\ket{\psi}
\end{equation}
Let $s=a_1b_1,\ldots, a_nb_n$ and let

 \begin{equation}
 \label{eq:3}
F_{C_{q_1}}(s)=(a^\prime_1,b^\prime_1,\ldots, a^\prime_n,b^\prime_n) \mbox{ and } F_{C_{q_2}}(s)=(d_1^\prime,e_1^\prime,\ldots, d_n^\prime,e^\prime_n)
\end{equation}

We can rewrite equation \ref{eq:2} as (see gate teleportation \ref{protocol: gate-teleportation} or algorithm 3.
 \begin{equation}
  \label{eq:4}
(\zgate^{\otimes_{i=1}^{n} a_{i}^\prime} \cdot \xgate^{\otimes_{i=1}^{n} b_{i}^\prime})C_q(\ket{\psi})=(\zgate^{\otimes_{i=1}^{n} d_{i}^\prime} \cdot \xgate^{\otimes_{i=1}^{n} e_{i}^\prime})C_q^\prime(\ket{\psi})
\end{equation}

Since $C_q^\prime(\ket{\psi})=C_q^\prime(\ket{\psi})$ we can substitute $C_q^\prime$ for $C_q^\prime$ in equation \ref{eq:4}

\begin{equation}
  \label{eq:5}
(\zgate^{\otimes_{i=1}^{n} a_{i}^\prime} \cdot \xgate^{\otimes_{i=1}^{n} b_{i}^\prime})C_q(\ket{\psi})=(\zgate^{\otimes_{i=1}^{n} d_{i}^\prime} \cdot \xgate^{\otimes_{i=1}^{n} e_{i}^\prime})C_q(\ket{\psi})
\end{equation}

It follows from equation \ref{eq:5} that
\begin{equation}
  \label{eq:6}
(\zgate^{\otimes_{i=1}^{n} a_{i}^\prime} \cdot \xgate^{\otimes_{i=1}^{n} b_{i}^\prime})=(\zgate^{\otimes_{i=1}^{n} d_{i}^\prime} \cdot \xgate^{\otimes_{i=1}^{n} e_{i}^\prime})
\end{equation}

It follows from equation \ref{eq:6}  that $$\zgate^{a_i^\prime} \cdot \xgate^{b_i^\prime}=\zgate^{d_i^\prime} \cdot \xgate^{e_i^\prime} \mbox{ for all } i\in[n].$$ This means $a_i^\prime=d_i^\prime$ and  $b_i^\prime=e_i^\prime$ and therefore, if $C_{q_1}$ and $C_{q_2},$ are two equivalent circuits, then they have the same update function.
\end{flushleft}





%\subsection{Indistinguishability Obfuscation for the Clifford Update Functions}
%\label{sec: iO-clifford-functions}
%
%
%\anote{I am not sure how am I to understand that the $F_{C_q}$ are Cliffords.... I thought they were classical circuits?}
%
%In this section we present a (quantum secure classical) $i\mathcal{O}$ that can obfuscate the update function $F_{C_q},$ for any $n$-qubit Clifford circuit $C_q.$
%
%\begin{algorithm}[H]
%   \caption{$i\mathcal{O}$ for Clifford update Functions $F_{\tt Clifford}$}
%  \begin{itemize}
%  \item Input: A $n$-qubit Clifford Circuit $C_q,$ a security parameter.
%  \begin{enumerate}
%  \item Compute $C_{q_c}\leftarrow$ {\tt AG-Canonical-Unique}$(C_q).$
%  \item Compute the circuit $C_{F_{q_c}}$ for the update function $F_{C_{q_c}}.$
%  \item Output $C_{F_{q_c}}.$
%  \end{enumerate}
%  \end{itemize}
%\end{algorithm}

\section{Obfuscating Beyond Clifford Circuits}
\label{QiO:Clifford+T:family}
\Anne{I think we need to work on the structure of this section, currently when reading quickly, it is hard to understand how the various pieces fit together}
Recall from  section \ref{defn:Clifford+T:family} that adding a non-Clifford gate  such as $\tgate$-gate to Clifford gates gives us a generating set for all unitary quantum circuits. In this section we show how to build quantum indistinguishability obfuscation for any quantum circuit that has at most logarithmic  $\tgate$-gates in the size of the circuit.
%\anote{Is this the correct link to the Definition?}

%\subsubsection{Using a Canonical Form}
%Main idea: Each Clifford layer is obfuscated using the canonical form. The $\tgate$-gate layers are given in the clear. The result is a canonical form, since the family is selected such that the description just given will be a canonical form.
%\anote{I think this one is less interesting because the $T$ gates are in the clear. Maybe drop it?}


\subsubsection{Using Gate Teleportation}
\label{QiO:Clifford+T:family+GT}
In this section we show how we can obfuscate any 1-qubit unitary circuit using gate teleportation and a quantum secure iO; and then based on this idea we will show how we can obfuscate any $n$-qubit circuit $C_q$ as far as the number of $\tgate$-gates remain logarithmic in $|C_q|.$ The $\tgate$ relates to the $\xgate, \zgate$
$$\tgate \xgate^b \zgate^a=\xgate^b \zgate^{a\oplus b}\pgate^b \tgate$$
If $b=0,$ then we don't have to worry about $\pgate$ gate otherwise we have to perform a $\pgate$ correction. We can write $\pgate=(\frac{1+i}{2}) \igate + (\frac{1-i}{2})\zgate,$ therefore

\begin{equation}
\label{eq:pgate1}
\tgate \xgate^b \zgate^a=\xgate^b \zgate^{a\oplus b}\left[\left(\frac{1+i}{2}\right) \igate + \left(\frac{1-i}{2}\right)\zgate\right]^b \tgate
\end{equation}
Note  $\left[\left(\frac{1+i}{2}\right) \igate + \left(\frac{1-i}{2}\right)\zgate\right]^b=\left(\frac{1+i}{2}\right) \igate + \left(\frac{1-i}{2}\right)\zgate^b$ for $b \in \{0,1\},$ therefore we can rewrite the equation \ref{eq:pgate1} as
\begin{equation}
\label{eq:pgate2}
\begin{aligned}
\tgate \xgate^b \zgate^a=\xgate^b \zgate^{a\oplus b}\left[\left(\frac{1+i}{2}\right) \igate + \left(\frac{1-i}{2}\right)\zgate^b\right] \tgate \\
=\left[\left(\frac{1+i}{2}\right) \xgate^b \zgate^{a\oplus b} + \left(\frac{1-i}{2}\right)\xgate^b \zgate^a\right] \tgate. \\
\end{aligned}
\end{equation}

Since $a$ and $b$ are bits, the equation \ref{eq:pgate2} can be written as a linear combination of $\{\igate, \xgate, \zgate, \xgate \zgate\},$

\begin{equation}
\label{eq:pgate3}
\begin{aligned}
\tgate \xgate^b \zgate^a= (\alpha_0 \igate +  \alpha_1 \xgate + \alpha_2 \zgate + \alpha_3 \xgate\zgate)\tgate \\
\end{aligned}
\end{equation}
for   $\alpha_i \in\left\{0,1, \frac{1+i}{2}, \frac{1-i}{2}\right\},$  $i\in[4].$ The equation \ref{eq:pgate3} is important as we will see that the correction unitary or any 1-qubit circuit can be expressed in the form of \ref{eq:pgate3} .


Let $C_q$ be a $1$-qubit circuit we want to obfuscate and $\ket{\psi}$ is the quantum state on which we want to evaluate $C_q.$ Note we can write any $1$-qubit circuit as a sequence of gates from the set $\{\hgate, \tgate\}$ \footnote{The set $\{\hgate,  \tgate\}$ is universal for 1-qubit unitary.}

$$C_q=(g_{|C_q|},\ldots,g_2,g_1 ),\; g_i\in\{\hgate, \tgate\}$$
In gate teleportation algorithm we have an subsystem of the form $C_q \xgate^b\zgate^a (\ket{\psi})$ (section \ref{protocol: gate  teleportation})

\begin{equation}
\label{eq:pgate4}
C_q \xgate^b\zgate^a (\ket{\psi})=(g_{|C_q|},\ldots,g_2,g_1 )\xgate^b\zgate^a (\ket{\psi}),
\end{equation}
and to evaluate the circuit on the input we have to apply a correction unitary. If $g_1=\hgate$ then we can write the above system as

\begin{equation}
\begin{aligned}
\label{eq:pgate5}
C_q \xgate^b\zgate^a (\ket{\psi})=(g_{|C_q|},\ldots,g_2)  \xgate^a \zgate^b g_1 (\ket{\psi}) \\
=(g_{|C_q|},\ldots,g_2)(e_0 \igate +  e_1 \xgate + e_2 \zgate + e_3\xgate \zgate)g_1 (\ket{\psi})
\end{aligned}
\end{equation}
where  $e_i \in \{0,1\},$  $i\in[4].$ Similarly if $g_1=\tgate,$ then we can also write the expression \ref{eq:pgate4} as

\begin{equation}
\begin{aligned}
\label{eq:pgate6}
C_q \xgate^b\zgate^a (\ket{\psi})=(g_{|C_q|},\ldots,g_2)(e_0^\prime \igate +  e_1^\prime \xgate + e_2^\prime \zgate + e_3^\prime \xgate \zgate)g_1 (\ket{\psi})
\end{aligned}
\end{equation}
where $e_i^\prime \in\left\{0,1, \frac{1+i}{2}, \frac{1-i}{2}\right\}$  for $i\in[4].$ Note there are more possibilities for $e_i^\prime$'s in the expression \ref{eq:pgate6} than $e_i$'s in the expression \ref{eq:pgate5}. This is because $\tgate$-gates can affect the size of the coefficients and whereas $\hgate$-gates can only permute the coefficients in the summation. After applying $g_2$ to expressions \ref{eq:pgate5} or \ref{eq:pgate6} the number of terms in the summation cannot exceed to more than~4. Moreover, the size of the coefficients $e_i$'s may only change if $g_2=\tgate,$

\begin{equation}
\begin{aligned}
\label{eq:pgate7}
C_q \xgate^b\zgate^a (\ket{\psi})=(g_{|C_q|},\ldots,g_3)(\gamma_0 \igate +  \gamma_1 \xgate + \gamma_2 \zgate + \gamma_3 \xgate \zgate)g_2g_1 (\ket{\psi})
\end{aligned}
\end{equation}
where $\gamma_i \in \mathbb{C}$ if $g_2=\tgate$ and $\gamma_i  \in\left\{0,1, \frac{1+i}{2}, \frac{1-i}{2}\right\},$  $i\in[4].$

Hence, it follows  that by applying the remaining  sequence of gates $(g_{|C_q|},\ldots,g_3)$, the expression \ref{eq:pgate7} can be written as
\begin{equation}
\begin{aligned}
\label{eq:pgate8}
C_q \xgate^b\zgate^a (\ket{\psi})=(\beta_0 \igate +  \beta_1 \xgate + \beta_2 \zgate + \beta_3\xgate \zgate)(g_{|C_q|},\ldots,g_2,g_1 )(\ket{\psi})
\end{aligned}
\end{equation}
where $\beta_i  \in \mathbb{C},$ $i\in[4].$  Therefore, for any  $1$-qubit circuit $C_q,$ the correction unitary can be written as

\begin{equation}
\label{eq:pgate9}
 (\beta_0 \igate +   \beta_1 \xgate +  \beta_2 \zgate +  \beta_3 \xgate\zgate), \mbox{ for }  \beta_i \in \mathbb{C},\; i\in[n]
\end{equation}

Moreover, it can be shown that the size of $\beta_i$'s grows at most a polynomial in the number of $\tgate$-gates in the circuit (appendix \ref{coeff:size}). Therefore the update function for any $1$-qubit circuit $C_q$ can be defined as a following map,

\begin{equation*}
F_{C_q}:\{0,1\}^2\rightarrow \mathbb{C}^4,\; (a,b)\mapsto (\beta_0, \beta_1,\beta_2,\beta_3),
\end{equation*}
which corresponds to the correction unitary $(\beta_0 \igate +  \beta_1 \xgate + \beta_2 \zgate + \beta_3\xgate \zgate).$
Moreover, the update functions are for the any two equivalent 1-qubit circuit are equivalent (see Lemma \ref{lemma:1qubit}). Therefore, we can obfuscate any 1-qubit circuit using gate teleportation and a quantum secure iO as we did for the Clifford circuits (section \ref{sec:Clifford-iO-teleportaion}).



\begin{lemma}
\label{lemma:1qubit}
Let $C_q$ and $C_q^\prime$ be two equivalent 1-qubit circuits and
\begin{equation*}
F_{C_q}:\{0,1\}^2\rightarrow \mathbb{C}^4 \mbox{ and } F_{C_q^\prime}:\{0,1\}^2\rightarrow \mathbb{C}^4
\end{equation*}
are the corresponding update functions for $C_q$ and $C_q^\prime,$  then $F_{C_q}=F_{C_q^\prime}.$
\end{lemma}

{\bf Proof:}  Suppose  $C_q$ and $C_q^\prime$ be two equivalent and let  $F_{C_q}(a,b)=(\beta_0, \beta_1, \beta_2, \beta_3)$ and $F_{C_q^\prime}(a,b)=(\beta_0^\prime, \beta_1^\prime, \beta_2^\prime, \beta_3^\prime)$ for $a,b\in\{0,1\}$ be their corresponding update functions. Let $(\ket{\psi}$ be any  qubit.
\begin{equation}
\label{lemma:1qubit:eq1}
\begin{aligned}
 C_q \xgate^b \zgate^a(\ket{\psi})=(\beta_0 \igate+\beta_1 \xgate+\beta_2 \zgate+\beta_3 \xgate \zgate)  C_q (\ket{\psi}).\\
 \end{aligned}
\end{equation}

\begin{equation}
\label{lemma:1qubit:eq2}
\begin{aligned}
 C_q^\prime \xgate^b \zgate^a(\ket{\psi})=(\beta_0^\prime \igate+\beta_1^\prime \xgate+\beta_2^\prime \zgate+\beta_3^\prime \xgate \zgate)  C_q^\prime (\ket{\psi}).
 \end{aligned}
\end{equation}

Since,  $C_q (\rho)=C_q^\prime (\rho)$ for every input $\rho$ we can must have
\begin{equation}
C_q \xgate^b \zgate^a(\ket{\psi})= C_q^\prime \xgate^b \zgate^a(\ket{\psi}).
\end{equation}

Therefore we have

\begin{equation}
\label{lemma:1qubit:eq3}
\begin{aligned}
(\beta_0 \igate+\beta_1 \xgate+\beta_2 \zgate+\beta_3 \xgate \zgate)  C_q (\ket{\psi})=(\beta_0^\prime \igate+\beta_1^\prime \xgate+\beta_2^\prime \zgate+\beta_3^\prime \xgate \zgate)  C_q^\prime (\ket{\psi})\\
(\beta_0 \igate+\beta_1 \xgate+\beta_2 \zgate+\beta_3 \xgate \zgate)  C_q (\ket{\psi})=(\beta_0^\prime \igate+\beta_1^\prime \xgate+\beta_2^\prime \zgate+\beta_3^\prime \xgate \zgate)  C_q(\ket{\psi}).\\
\end{aligned}
\end{equation}


From \ref{lemma:1qubit:eq3} we must have,
\begin{equation}
\label{lemma:1qubit:eq4}
\begin{aligned}
(\beta_0 \igate+\beta_1 \xgate+\beta_2 \zgate+\beta_3 \xgate \zgate) =(\beta_0^\prime \igate+\beta_1^\prime \xgate+\beta_2^\prime \zgate+\beta_3^\prime \xgate \zgate).\\
\end{aligned}
\end{equation}

From equation \ref{lemma:1qubit:eq4}
\begin{equation}
\label{lemma:1qubit:eq5}
\begin{aligned}
(\beta_0-\beta_0^\prime) \igate+(\beta_1 -\beta_1^\prime)\xgate+(\beta_2-\beta_2^\prime) \zgate+(\beta_3-\beta^\prime_3) \xgate \zgate={\bf 0}.
\end{aligned}
\end{equation}

The equation \ref{lemma:1qubit:eq5},  can only be satisfied if $\beta_i = \beta_i^\prime$ for every $i\in[4],$ therefore,  $F_{C_q}=F_{C_q^\prime}.$


\subsection{Quantum Indistinguishability Obfuscation for $n$-qubit Circuits}
In the previous section we showed how we can construct a $Qi\mathcal{O}$  for any $1$-qubit unitary circuit. In this section we show that how to construct  $Qi\mathcal{O}$ for any $n$-qubit unitary circuit $C_q$ using iO gate teleportation and quantum secure iO provided the number of $\tgate$-gates are in $C_q$ are $O(\log(|C_q|).$  The reason for $O(\log(|C_q|)$ bound is that when $\cnot$ and $\tgate$-gates are mixed together then the
the second property ({\tt Polynomial Slowdown}) of $Qi\mathcal{O}$ is violated if $\tgate$-count $\in \omega(\log(|C_q|)).$ More precisely we will show that for any $n$-qubit circuit with $k$ number of $\tgate$-gates,  the correction unitary in the worst-case has $4^k$ terms is of form  \ref{exp:nqubit-correction}

\begin{equation}
\label{exp:nqubit-correction}
\sum_{i=0}^{4^k-1} \beta_i \xgate^{b{i_1}} \zgate^{a{i_1}}\otimes \cdots \otimes \xgate^{b{i_n}} \zgate^{a{i_n}},\; \beta_i\in\mathbb{C},\; a_{i,j},b_{i_j}\in\{0,1\}.
\end{equation}

Since, there are at most $2^{2n}$ bit strings of size $n,$ it follows from  \ref{exp:nqubit-correction} that for any $n$-qubit circuit the correction unitary can be written as a summation of at most $2^{2n}$ terms. This means the terms in the correction unitary can increase exponential in the number of $\tgate$-gates for $k\leq n.$ But if the correction unitary has $2^{2n}$ terms, then by applying more $\tgate$-gates or any gate cannot increase the number of terms in the expression \ref{exp:nqubit-correction}. However,  the size of size of coefficients $\beta_i$'s can be affected with each application of a $\tgate$-gate (see Remark \ref{remark:nqubit-correction2} for details). Note that for any uniform family of quantum circuits
$\mathcal{C}_q=\{C_{q_n}\mid n\in \mathbb{N}\}$ there exist a $poly(n)$ such that for every $C_{q_n}\in \mathcal{C}_q,$ imply $|C_{q_n}|\leq poly(n).$ Otherwise we can add $\igate$-gates to any circuit such that the $|C_q|\geq 2^{2n},$ and then obfuscate any circuit with polynomially many $\tgate$-gates using gate teleportation and classical iO.


Let $C_q$ be an $n$-qubit circuit and $\ket{\psi}$ be an $n$-qubit state. If we want to evaluate $C_q$ on the state $\ket{\psi}$ using gate teleportation, then we are required to compute the correction unitary $U^\prime$ corresponding to $C_q$ (section \ref{protocol: gate  teleportation}) satisfying equation \ref{nqubit:eq6}.

\begin{equation}
\label{nqubit:eq6}
C_qU(\ket{\psi})= U^\prime C_q (\xgate^{a_{1}}\zgate^{b_{1}}\otimes \xgate^{a_{2}}\zgate^{b_{2}}\cdots \otimes \xgate^{a_{n}}\zgate^{b_{n}})(\ket{\psi}).
\end{equation}

Let's construct $U^\prime$ and estimate how big it can grow. Suppose number of $\tgate$ gates in $C_q$ are $k$ for some non-negative integer $k\leq n.$ To simplify the analysis we assume that $\tgate$-gates are in the first $l$ wires of $C_q.$ \footnote{This restriction will not not affect our analysis see  remark \ref{remark:nqubit-correction2}.} Note we must have $l\leq k.$ Then for every $j$-th wire with $\tgate$-gate(s)

\begin{equation}
\begin{aligned}
\label{nqubit:eq7}
\tgate(\xgate^{a_{j}}\zgate^{b_{j}})\mapsto \left(\lambda_{j_0} \igate + \lambda_{j_1} \xgate + \lambda_{j_2} \zgate + \lambda_{j_3} \xgate \zgate \right)\tgate=\left(\sum_{i=1}^4 \lambda_{j_i} \xgate^{a_{i}^\prime}\zgate^{b_{i}^\prime}\right)\tgate\\
 \end{aligned}
\end{equation}
where $\lambda_{j_i}\in\mathbb{C}, $  $a_{i}^\prime, b_{i}^\prime \in\{0,1\}$ for $i\in[4]$ and $j\in[l].$ For now also assume $C_q$ has no $\cnot$ gates. Then we can rewrite relation \ref{nqubit:eq6} as


\begin{equation}
\label{nqubit:eq8}
\begin{aligned}
 C_q (\xgate^{a_{i}}\zgate^{b_{i}})^{\otimes_{i=1}^{n}}\ket{\psi}= \left(\sum_{i=1}^4 \lambda_{j_i} \xgate^{a_{i}^\prime}\zgate^{b_{i}^\prime}\right)^{\otimes_{i=1}^{l}} \otimes \left(\xgate^{a_{i}^\prime}\zgate^{b_{i}^\prime}\right)^{\otimes_{i=l+1}^{n}}\\
\end{aligned}
\end{equation}
Now suppose $C_q$ has $\cnot$-gates. Then in the worst-case we have to expand the relation \ref{nqubit:eq8} as using identity $(A+B)\otimes C=A\otimes C + B\otimes C.$


\begin{equation}
\label{nqubit:eq9}
\begin{aligned}
C_q (\xgate^{a_{i}}\zgate^{b_{i}})^{\otimes_{i=1}^{n}}=\left(\sum_{j=1}^{4^l} \beta_j \xgate^{b_{j1}}\zgate^{a_{j1}} \otimes \cdots \otimes\xgate^{b_{jl}}\zgate^{a_{jl}}\right) \otimes \left(\xgate^{a_{i}^\prime}\zgate^{b_{i}^\prime}\right)^{\otimes_{i=l+1}^{n}}, \\
 =\left(\sum_{j=1}^{4^{l}} \beta_j \xgate^{b_{j1}^\prime}\zgate^{a_{j1}^\prime} \otimes \cdots \otimes\xgate^{b_{jl}^\prime}\zgate^{a_{jl}^\prime} \otimes \xgate^{a_{l+1}^\prime}\zgate^{b_{l+1}^\prime}  \otimes \cdots   \otimes\xgate^{a_{n}^\prime}\zgate^{b_{n}^\prime}\right)
 \end{aligned}
\end{equation}
where $\beta_j\in \mathbb{C}$ for $j\in[4^l].$  Recall that $k$ denote the number of $\tgate$-gates and $l\leq k,$ therefore in the worst case the correction unitary can have $4^k$ terms.

\begin{remark}
\label{remark:nqubit-correction1}
Note the restriction that $\tgate$-gates are in the first $l$ wires of the circuit has no affect on our analysis. This because, any permutation of the wires in the circuit will give us $l$ terms of form $\sum_{i=1}^4 \lambda_{j_i} \xgate^{a_{j_i}^\prime} \zgate^{b_{j_i}^\prime}$ and $n-l$ terms of the form $\xgate^{a_{i}^\prime} \zgate^{b_{i}^\prime}.$  And the tensor product of these terms in any order will result into a unitary in the worst-case of form
\end{remark}

\begin{equation}
\label{nqubit-correction-final}
\sum_{i=1}^{4^k} \beta_i \xgate^{b{i_1}} \zgate^{a{i_1}}\otimes \cdots \otimes \xgate^{b{i_n}} \zgate^{a{i_n}},\; \beta_i\in\mathbb{C},\; a_{i_j},b_{i_j}\in\{0,1\}.
\end{equation}

Therefore, if $k\in O(\log(|C_q|)$ (where $k$ is the $\tgate$-count), then the number of terms in the correction unitary \ref{nqubit-correction-final} are in $O(|C_q|).$ Also, note that if $k\in \omega(\log(|C_q|)),$ then there are $4^{\omega(\log(|C_q|)}$  terms in \ref{nqubit-correction-final}, which is a super-polynomial in $|C_q|.$




\begin{remark}
\label{remark:nqubit-correction2}
Note for any $n$-qubit circuit the correction unitary (relation \ref{nqubit-correction-final} can have at most $2^{2n}$ terms, regardless how many $\tgate$-gates a circuit has. To realize this suppose we have
$$C_q (\xgate^{a_{i}}\zgate^{b_{i}})^{\otimes_{i=1}^{n}}=\left(\sum_{j=0}^{l} \beta_j \xgate^{b_{j1}}\zgate^{a_{j1}} \otimes \cdots \otimes\xgate^{b_{jl}}\zgate^{a_{jl}} \otimes \xgate^{a_{l+1}^\prime}\zgate^{b_{l+1}^\prime}  \otimes \cdots   \otimes\xgate^{a_{n}^\prime}\zgate^{b_{n}^\prime}\right)$$
for some integer $l>2^{2n}.$
But since, there are at most $2^{2n}$ bit strings of size $2n,$ by pigeon hole principle we must have at least $2n-l$ terms that have common binary strings and by adding these common terms we are guaranteed to have every unitary that has at most $2^{2n}$ terms. This means that if the number of terms in the correction unitary grows exponentially (in the worst-case) in the number of $\tgate$-gates $k$ as far as $k\geq n.$ Note the size of the coefficients $\beta_i$ is a function of $\tgate$-gates and will be affected as number of $\tgate$-gates increases. So if there are exponentially many $\tgate$-gates then the size of the $\beta_i$ can also also grow exponentially.
\end{remark}


\subsection{Update Function for an arbitrary $n$-Qubit circuit}
\label{nqubit:update-function}
In the previous section saw that the general form of a correction unitary for an arbitrary quantum circuit is $$\sum_{i=1}^{4^k} \beta_i \xgate^{b{i_1}} \zgate^{a{i_1}}\otimes \cdots \otimes \xgate^{b{i_n}} \zgate^{a{i_n}}.$$ But in order to construct $Qi\mathcal{O}$ we have to define precisely define update functions. The update function must satisfy the following two properties.

\begin{enumerate}
\item ({\bf Invariant}) Any two equivalent quantum circuits should have the update function.
\item ({\bf Efficiency}) The update function can be computed by a classical circuit of polynomial size.
\end{enumerate}

Let $C_q$ be an $n$-qubit circuit and $\mathbb{C}^*= \mathbb{C}-\{0\}$ (i.e. set of all complex numbers excluding 0). The update function $F_{C_q}$ for the circuit is define as

\begin{equation}
\label{nqubit:eq10}
\begin{aligned}
F_{C_q}: \{0,1\}^{2n}\longrightarrow {\mathbb{C}^*}^m\times \{0,1,\ldots,2^{2n}-1\}^m, \\
\end{aligned}
\end{equation}
$$(a_1,b_1,\ldots, a_n,b_n) \mapsto ((\beta_{k_1}, \beta_{k_2},\ldots, \beta_{k_m}),(k_1,k_2,\ldots, k_m))$$

where $m$ is some integer depend on $C_q$ and each integer $k_i$ encode a $2n$ bit string. Moreover  $k_i< k_j$ for all indices $1 \leq i<j \leq n.$ Note  the update function (relation \ref{nqubit:eq10} ) corresponds to the correction unitary (relation  \ref{nqubit:eq11} )

\begin{equation}
\label{nqubit:eq11}
\begin{aligned}
\sum_{i=1}^m \beta_{k_i} \xgate^{b_{k_{i1}}} \zgate^{a_{k_{i1}}} \otimes \cdots  \otimes \xgate^{b_{k_{in}}} \zgate^{a_{k_{in}}},
\end{aligned}
\end{equation}
where string $b_{k_{i1}} a_{k_{i1}}\ldots b_{k_{in}} a_{k_{in}}$ is the binary representation of integer $k_i$ for $i\in[m].$ The lemma \ref{lemma:nqubit} prove the invariant property of the update function (see \ref{nqubit:update-function}).


\begin{lemma}
\label{lemma:nqubit}
Let $C_q$ and $C_q^\prime$ be two equivalent $n$-qubit circuits and
\begin{equation*}
\begin{aligned}
F_{C_q}: \{0,1\}^{2n}\longrightarrow {\mathbb{C}^*}^m\times \{0,1,\ldots,2^{2n}-1\}^m,  \\
F_{C_q^\prime}: \{0,1\}^{2n}\longrightarrow {\mathbb{C}^*}^{t}\times \{0,1,\ldots,2^{2n}-1\}^{t}.
\end{aligned}
\end{equation*}
are the corresponding update functions for $C_q$ and $C_q^\prime,$  (as defined in relation relation \ref{nqubit:eq10}), then $F_{C_q}=F_{C_q^\prime}.$
\end{lemma}

\noindent{\bf Proof}: Suppose $C_q$ and $C_q^\prime$ are equivalent circuits and
\begin{equation*}
\begin{aligned}
F_{C_q}(a_1,b_1,\ldots,a_n,b_n)= ((\beta_{k_1}, \beta_{k_2},\ldots, \beta_{k_m}),(k_1,k_2,\ldots, k_m)) \\
F_{C_q}^\prime(a_1,b_1,\ldots,a_n,b_n)= ((\beta_{k_1}^\prime, \beta_{k_2}^\prime,\ldots, \beta_{k_m}^\prime),(k_1^\prime,k_2^\prime,\ldots, k_t^\prime))
\end{aligned}
\end{equation*}
For every quantum state $\ket{\psi}$ we have
\begin{equation}
\label{nqubit:eq12}
\begin{aligned}
C_q (\xgate^{b_i}\zgate^{a_i})^{\otimes_{i=1}^n}(\ket{\psi})=C_q^\prime (\xgate^{b_i}\zgate^{a_i})^{\otimes_{i=1}^n}(\ket{\psi})
\end{aligned}
\end{equation}

From relation \ref{nqubit:eq12} we have
\begin{equation}
\label{nqubit:eq13}
\begin{aligned}
\left(\sum_{i=1}^m \beta_{k_i} \xgate^{b_{k_{i1}}} \zgate^{a_{k_{i1}}} \otimes \cdots  \otimes \xgate^{b_{k_{in}}} \zgate^{a_{k_{in}}}\right) C_q (\ket{\rho})=\left(\sum_{i=1}^t \beta_{k_i}^\prime \xgate^{b_{k_{i1}^\prime}} \zgate^{a_{k_{i1}}^\prime} \otimes \cdots  \otimes \xgate^{b_{k_{in}}^\prime} \zgate^{a_{k_{in}}^\prime}\right)C_q^\prime (\ket{\rho})
\end{aligned}
\end{equation}

We assume WLOG we can assume that the terms (on R.H.S and L.H.S) in the relation \ref{nqubit:eq13} are written in the lexicographic order impose by  $k_1<k_2<\ldots <k_m$ and $k_1^\prime<k_2^\prime<\ldots <k_t^\prime.$ Note we can substitute  $C_q(\ket{\rho})$ for $C_q^\prime (\ket{\rho})$ in relation \ref{nqubit:eq13}.

\begin{equation}
\label{nqubit:eq14}
\begin{aligned}
\left(\sum_{i=1}^m \beta_{k_i} \xgate^{b_{k_{i1}}} \zgate^{a_{k_{i1}}} \otimes \cdots  \otimes \xgate^{b_{k_{in}}} \zgate^{a_{k_{in}}}\right) C_q (\ket{\rho})=\left(\sum_{i=1}^t \beta_{k_i}^\prime \xgate^{b_{k_{i1}^\prime}} \zgate^{a_{k_{i1}}^\prime} \otimes \cdots  \otimes \xgate^{b_{k_{in}}^\prime} \zgate^{a_{k_{in}}^\prime}\right)C_q (\ket{\rho}).
\end{aligned}
\end{equation}
From relation \ref{nqubit:eq14} it follows that

\begin{equation}
\label{nqubit:eq15}
\begin{aligned}
\left(\sum_{i=1}^m \beta_{k_i} \xgate^{b_{k_{i1}}} \zgate^{a_{k_{i1}}} \otimes \cdots  \otimes \xgate^{b_{k_{in}}} \zgate^{a_{k_{in}}}\right)=\left(\sum_{i=1}^t \beta_{k_i}^\prime \xgate^{b_{k_{i1}^\prime}} \zgate^{a_{k_{i1}}^\prime} \otimes \cdots  \otimes \xgate^{b_{k_{in}}^\prime} \zgate^{a_{k_{in}}^\prime}\right)
\end{aligned}
\end{equation}

From relation \ref{nqubit:eq15} we have
\begin{equation}
\label{nqubit:eq16}
\begin{aligned}
\left(\sum_{i=1}^m \beta_{k_i} \xgate^{b_{k_{i1}}} \zgate^{a_{k_{i1}}} \otimes \cdots  \otimes \xgate^{b_{k_{in}}} \zgate^{a_{k_{in}}}\right)-\left(\sum_{i=1}^t \beta_{k_i}^\prime \xgate^{b_{k_{i1}^\prime}} \zgate^{a_{k_{i1}}^\prime} \otimes \cdots  \otimes \xgate^{b_{k_{in}}^\prime} \zgate^{a_{k_{in}}^\prime}\right)={\bf 0}
\end{aligned}
\end{equation}

By definition of update functions (relation \ref{nqubit:eq10}) we have $m$ distinct binary strings $b_{k_{i1}} a_{k_{i1}}\ldots b_{k_{in}} a_{k_{in}}$ corresponding to the $m$ distinct integers $\{k_1,k_2,\cdots, k_m\}$ and similarly we have $t$ distinct binary strings $b_{k_{i1}^\prime} a_{k_{i1}^\prime}\ldots b_{k_{in}^\prime} a_{k_{in}^\prime}$ corresponding to corresponding to the $t$ distinct integers $\{k_1^\prime,k_2^\prime,\cdots, k_t^\prime\}.$  Moreover, by definition of the update functions (relation \ref{nqubit:eq10}) $\beta_i\neq 0$ for $i\in [m]$ and $\beta_i^\prime\neq 0$  for $i\in [t].$ Now suppose if  $m>t$ or $m<t$ then there is at least one non-zero term on the L.H.S
in the relation \ref{nqubit:eq16}, while the R.H.S is equal to a {\bf 0} matrix. Therefore $m=t$ and since the terms our in  lexicographic order we must have for all $i\in[m].$
\begin{equation}
\label{nqubit:eq17}
 \beta_{k_i}\xgate^{b_{k_{i1}}} \zgate^{a_{k_{i1}}} \otimes \cdots  \otimes \xgate^{b_{k_{in}}} \zgate^{a_{k_{in}}}=\beta_{k_i}^\prime\xgate^{b_{k_{i1}^\prime}} \zgate^{a_{k_{i1}}^\prime} \otimes \cdots  \otimes \xgate^{b_{k_{in}^\prime}} \zgate^{a_{k_{in}^\prime}}.
\end{equation}
 Otherwise, L.H.S cannot equal to {\bf 0}. Therefore for all $i\in[m],$  $\beta_{k_i}=\beta_{k_i}^\prime$ and $k_i=k_i^\prime$ and we have $F_{C_q}=F_{C_q^\prime}.$

\section*{Cost of Computing the Update Function}
We have to show that the update function can be computed by a polynomially size classical circuit (see \ref{nqubit:update-function}). Let $C_q $ be an $n$-qubit circuit with at most $O(\log(|C_q|)$ $\tgate$-gates and
                                                                                        $$F_{C_q}(s)=((\beta_{k_1}, \beta_{k_2},\ldots, \beta_{k_m}),(k_1,k_2,\ldots, k_m)),\; \mbox{for } s\in\{0,1\}^{2n}.$$
Since, $\tgate$-count in $O(\log(|C_q|),$ we have $m\in O(4^{\log(|C_q|}))= O(|C_q|).$ Moreover, each the size of $\beta_i \in poly(|C_q|)$ (appendix \ref{coeff:size}) and $k_i \in O(n),$ but $n\leq |C_q|,$ therefore  $k_i \in O(|C_q|).$ Now in order to compute $F_{C_q}$ we go gate by gate in $C_q$ and perform operations such as $\oplus$, swaps, $+$ and $\otimes$ on operand of size that are bounded by $poly(|C_q|).$ Therefore, $F_{C_q}$ can be computed in deterministic polynomial-time. From complexity theory we know that
if $F_{C_q}$ can be computed in time $O(t|C_q|)),$ then there is a classical circuit $C$ of size $O(t^2|C_q|)),$ that can compute $F_{C_q}$. %\cite{Sipser 3rd edition, page 407} (Sipser). 
Therefore $F_{C_q}$ can be computed by a classical circuit of $poly(|C_q|).$

\subsection{$Qi\mathcal{O}$ for Unitary Quantum Circuits with Low $\tgate$ gate Complexity}
The $Qi\mathcal{O}$ construction and proof of its security is identical to the to the construction for Clifford gates (section \ref{sec:Clifford-iO-teleportaion}).



\Anne{TODO: use package cleveref}
