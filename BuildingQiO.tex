
%%%%%%%%%%%%%%%%%%%%%%%%%%%%%%%%%%%%%%%%%%%%%%%%%%%%
%Section
\section{Quantum Indistinguishability Obfuscation}
\label{sec:QiO-Cliffords and more}
In this section we give a definition of equivalent quantum circuits and also define a notion of Quantum Indistinguishability Obfuscation $(Qi\mathcal{O})$. In  \Cref{QiO:Clifford-Circuits} we show how to construct $Qi\mathcal{O}$ for some families of quantum circuits. We give two alternative methods, the first based on canonical forms, and the second based on gate teleportation.

%\anote{Need some blah blah at the beginning of a section to say what we present in this section.}

\subsection{Definitions}
\label{def:equivalent-circuits}
\begin{definition} {\rm (Equivalent Quantum Circuits):}
Let $C_{q_0}$ and $C_{q_1}$ be two $n$-qubit quantum circuits. We say $C_{q_0}$ and $C_{q_1}$ are equivalent if for every $n$-qubit state $\ket{\psi}$
$$C_{q_1}(\ket{\psi}) =C_{q_2}(\ket{\psi}).$$
\end{definition}
%\anote{Need some blah blah before an important definition to describe what it is saying, but in words}

%\anote{why is computational in brackets below?}
\begin{definition}{\rm (Quantum Indistinguishability Obfuscation)}
\label{def:QiO}
A  polynomial-time quantum algorithm for a class of quantum circuits $\mathcal{C}_Q$ is a quantum computational indistinguishability obfuscator $Qi\mathcal{O}$  if the following conditions hold:

\begin{enumerate}
\item {\tt Functionality:}  For every $C_q\in \mathcal{C}_Q$ and every quantum state $\ket{\phi},$
$$(\rho, C_q^\prime)\leftarrow Qi\mathcal{O}(C_q) \;  \mbox{ and }\; C_q^\prime(\rho,\ket{\phi})=C_{q}(\ket{\phi})$$
%where $\rho \in Q(m)$ and $C_q^\prime$  is a $m+n$ qubit quantum circuit.
											
%\anote{above, we are mixing pure states such as $\ket{\phi}$ with mixed states, such as $\rho$. This gets confusing, but I don't know exacly how to fix it. Maybe $\rho$ could be assumed wlog to be pure, so we can use use $\ket{\psi}$ instead of $\rho?$. }

\item  {\tt Polynomial Slowdown:}  For every $C_{q}\in \mathcal{C}_Q,$
\begin{itemize}
\item  $\rho$ is at most a $poly(|C_{q}|)$ qubit state.
\item $|C_{q}^\prime| \in poly(|C_{q}|).$
\end{itemize}
%\anote{in below, did we define what we mean by size? If this is the number of gates, I would find the definition quite restrictive. Maybe you mean the same dimensions for input- output?}

\item {\tt Indistinguishability:} For any two equivalent quantum circuits $C_{q_1},C_{q_2}\in \mathcal{C}_Q,$ of the same size
 and for every polynomial-time quantum distinguisher $\mathcal{D}_q,$ there exists a negligible function {\rm negl} such that:
					$$\Big |{\rm Pr}[\mathcal{D}_q(Qi\mathcal{O}(C_{q_1}))=1]-{\rm Pr}[\mathcal{D}_q(Qi\mathcal{O}(C_{q_2}))=1] \Big |\leq  {\rm negl}(k).$$		
Where $k=|C_{q_1}|=|C_{q_2}|.$						
\end{enumerate}
\end{definition}


\subsection{Quantum Indistinguishability Obfuscation for Clifford Circuits}
\label{QiO:Clifford-Circuits}
 In this section we present two methods to obfuscate any Clifford circuit,  one using a canonical form and the other using gate teleportation.

\subsubsection{$Qi\mathcal{O}$ for Clifford via Canonical Form}
\label{sec:Clifford-iO-canonical}
Aaronson and Gottesman invented a polynomial-quantum algorithm (polynomial size circuit) that takes a Clifford circuit $C_q$ and output its canonical form~\cite{AG04} (section VI). This canonical form is invariant for any two equivalent circuits. Moreover the size of the canonical form remain $poly(|C_q|).$ Therefore, the $Qi\mathcal{O}$ in essentially the Aaronson and Gottesman algorithm that takes a Clifford circuit $C_q,$ as an input and output its canonical form $C_q^\prime$ and an empty register $\rho$ \footnote{Note we output the empty register to satisfy the definition of quantum indistinguishability obfuscation (\ref{def:QiO}).}
														$$(\rho, C_q^\prime)\leftarrow Qi\mathcal{O}(C_q).$$

 \begin{lemma}
The above construction (section \ref{sec:Clifford-iO-canonical}) is a Quantum Indistinguishability Obfuscation for Clifford Circuits.
\end{lemma}

\noindent {\bf Proof}: We have to show that the construction satisfies definition \ref{def:QiO} . Note $Qi\mathcal{O}$ is a polynomial-time quantum algorithm, since it is essentially the same a Aaronson and Gottesman a polynomial-quantum algorithm for computing a canonical form of a Clifford circuit ~\cite{AG04} (section VI).
To show the construction satisfies all 3 properties of the definition \ref{def:QiO}; let $C_q^\prime$ be the canonical form of some Clifford circuit $C_q.$  Since, $C_q^\prime$ and $C_q$ are equivalent circuits (definition \ref{def:equivalent-circuits}) they have the same function functionality. Note $\rho$ is an empty register and the size of $|C_q^\prime| \in poly(|C_q|)$ ~\cite{AG04}, hence the construction is efficient (polynomial slowdown). Moreover the canonical form reveals no knowledge about the input circuit except the unitary it computes, therefore this $Qi\mathcal{O}$ is perfectly indistinguishable against any quantum adversary. Therefore, also computationally indistinguishable against any quantum adversary.



\subsubsection{$Qi\mathcal{O}$ for Clifford via Gate Teleportation}
\label{sec:Clifford-iO-teleportaion}
In this section we  show that how gate teleportation can be used to construct a quantum indistinguishability obfuscation for Clifford circuits Our algorithm relies on the assumption that there exists a quantum secure $i\mathcal{O}$ for classical circuits, this seems problematic at first since, there is no provably quantum secure $i\mathcal{O}$ known for general classical circuits.
%\anote{Really? I thought there were some candidates.... }
However,  our construction relies on the assumption that a quantum secure $i\mathcal{O}$ exists for a very specific class of classical circuits the compute update functions (section \ref{update function}). And In fact, it is easy to construct an $i\mathcal{O}$ for classical circuits that compute update functions for Clifford circuits (appendix \ref{sec: iO-clifford-functions}). %section \ref{sec: iO-clifford-functions}). %\anote{Indeed, we're looking at some classical circuits that are probably quite easy to deal with. I'm not sure they are Cliffords, however, since I think they are classical functions.}


%\anote{below, $\rho$ is usually a density matrix, which does not require a $\ket{\cdot}$.}
%\anote{below, say something about what ``lower half'' means, or establish earlier on that another way to write the tensor product of $n$ Bell states is $\sum_i \ket{i}{i}$ (or something like this; need to check). Then it is clear which register we are talking about for applying $C_q$.}

\begin{algorithm}[H]
\label{QiO:Clifford-teleportation}
   \caption{$Qi\mathcal{O}$ using Gate Teleportation}
  \begin{itemize}
  \item Input: A $n$-qubit Clifford Circuit $C_q.$
  \begin{enumerate}
  \item Prepare $2n$ qubit Bell state $\ket{\beta^{ 2n}}=\ket{\beta_{00}}\otimes \cdots \otimes \ket{\beta_{00}}.$
  \item Apply the circuit $C_q$ on the right most $n$ qubits and obtain a system $\rho$
  										 $$\rho=(\igate_n\otimes C_q) \ket{\beta^{2n}}.$$
 \item Compute the classical circuit $C_{F_{C_q}}$ that computes the update function $F_{C_q}.$	
 \item  Set $C=i\mathcal{O}(C_{F_{C_q}}).$							
  \item Let $B_q$ denote the circuit that takes a $3n$ qubit state and perform a Bell measurement on the leftmost $2n$-qubits of the input state and obtain the bits (output of measurement), $(a_1,b_1\ldots,a_n,b_n)$
  \item Set $C_q^\prime=(B_q, C^\prime).$
  \item Output $\left(\rho,C_q^\prime \right).$
  \end{enumerate}
  \end{itemize}
\end{algorithm}

%\anote{For step 2 below, shall we add to preliminaries that classical circuits can be embedded into quantum ones? I'm not sure, actually, how you envisage this classical part of the corrections- maybe I haven't fully understood yet.}
%On receiving $\left(\rho,C_q^\prime \right),$ one can evaluate the original circuit $C_q$ on any $n$-qubit state $\ket{\psi}$ as follows

\begin{algorithm}[H]
\label{Eval:Clifford-teleportation}
\caption{Computing $C_q$ from $\left(\rho,C_q^\prime \right)$}
 \begin{itemize}
  \item Input $\left(\rho,C_q^\prime \right)$ and a quantum state $\ket{\psi}.$
  \begin{enumerate}
  \item   Write $C_q^\prime=(B_q, C)$ and the joint system $\rho_{AB}=\rho \otimes \ket{\psi}.$
  \item   Obtain the state $\rho_{AB}^\prime=B_q(\rho_{AB})$ and classical output $(a_1,b_1\ldots,a_n,b_n).$
  \item   Compute $C_q(\xgate^{\otimes_{i=1}^{n} b_{i}} \cdot \zgate^{\otimes_{i=1}^{n}})\ket{\psi}=Tr_A(\rho_{AB}^\prime)$ (see  gate teleportation protocol \ref{protocol: gate-teleportation}).
  \item   Compute the update function $$(a_1^\prime, b_1^\prime,\ldots, a_n^\prime, b_n^\prime)=C(a_1,b_1\ldots,a_n,b_n).$$
  \item    Compute the correction unitary $(\xgate^{\otimes_{i=1}^{n} b_{i}^\prime} \cdot \zgate^{\otimes_{i=1}^{n} a_{i}^\prime})$
  \item   Apply  $(\xgate^{\otimes_{i=1}^{n} b_{i}^\prime} \cdot \zgate^{\otimes_{i=1}^{n} a_{i}^\prime})$ to the system $C_q(\xgate^{\otimes_{i=1}^{n} b_{i}} \cdot \zgate^{\otimes_{i=1}^{n} a_{i}})\ket{\psi}$ to obtain the state $C_q(\ket{\psi})$ as follows,
  \begin{equation*}
  \begin{aligned}
 (\xgate^{\otimes_{i=1}^{n} b_{i}^\prime} \cdot \zgate^{\otimes_{i=1}^{n} a_{i}^\prime}) \cdot C_q(\xgate^{\otimes_{i=1}^{n} b_{i}} \cdot \zgate^{\otimes_{i=1}^{n}a_{i}})\ket{\psi}\\
 =(\xgate^{\otimes_{i=1}^{n} b_{i}^\prime} \cdot \zgate^{\otimes_{i=1}^{n} a_{i}^\prime}) \cdot (\zgate^{\otimes_{i=1}^{n} a_{i}^\prime} \cdot \xgate^{\otimes_{i=1}^{n}b_{i}^\prime})C_q(\ket{\psi})\\
 = C_q(\ket{\psi}).
 \end{aligned}
 \end{equation*}
  \end{enumerate}
  \end{itemize}
\end{algorithm}

\begin{theorem}
If $i\mathcal{O}$ is a quantum secure classical indistinguishability obfuscation, then $Qi\mathcal{O}$ constructed from the gate teleportation  a quantum Indistinguishability Obfuscation for all Clifford Circuits.
\end{theorem}

\anote{Note to self: I have not checked the below}

\noindent{\bf Proof}:
\begin{itemize}
\item  {\tt Functionality:} The functionality $Qi\mathcal{O}$ is followed from the gate teleportation (section \ref{protocol: gate-teleportation}).
\item {\tt Polynomial Slowdown:} Note $\rho$ is a $O(n)$-qubit state  and the size of $|C_q|=|B_q|+|C^\prime|.$ The circuit $B_q$ perform the Bell measurement of a $O(n)$ qubit system. Therefore, $B_q$ can be constructed with $O(n)$ gates. Note from complexity theory we know that if a function can be computed by a deterministic algorithm in time $t(n),$ then it can be computed by a circuit of size $O(t^2(n)).$ Therefore if we can show  that  ${F_{q_c}}$ can be computed in $poly(|C_q|),$ then we can conclude that the size of $|C|\in poly(|C_q|).$ Note the function $F_{C_q}$ takes  $2n$ bits as an input and perform $O(n)$ operations on the input bits for each gate in $C_{q}.$ There are at most $|C_q|$ gates, therefore the total cost of computing $F_{C_q}$ is $O(n)|C_q|.$  Note any $n$-qubit circuit $C_q,$ $n\leq |C_q|,$ the function $F_{C_q}$ can be computed by a circuit  of size at most $poly(|C_{q}|).$ Therefore,
\begin{equation*}
\rho\in poly(|C_q|)    \mbox{ and } |C_q^\prime |=|B_q|+|C| \in poly(|C_q|)
\end{equation*}

\item  {\tt Indistinguishability:} Let $C_{q_1}$ and $C_{q_2}$ be two $n$-qubit equivalent Clifford circuits
\begin{equation*}
\left(\rho_1 ,\left(B_q, i\mathcal{O}(C_{F_{C_{q_1}}})\right)\right)= Qi\mathcal{O}(C_{q_1}) \mbox{ and } \left(\rho_2 ,\left(B_q, i\mathcal{O}(C_{F_{C_{q_2}}})\right)\right)= Qi\mathcal{O}(C_{q_2})
\end{equation*}

Since $C_{q_1}(\ket{\tau})=C_{q_2}(\ket{\tau})$ for every quantum state $\ket{\tau}$ we have,

\begin{equation}
\label{eq:1}
\rho_1=(I\otimes C_{q_1}) \ket{\beta^{2n}}=(I\otimes C_{q_2}) \ket{\beta^{2n}}=\rho_2.
\end{equation}
And of course $B_q$ will always be the same circuit for circuits. So the only thing we need to show that $i\mathcal{O}(C_{F_{C_{q_1}}})$ and $i\mathcal{O}(C_{F_{C_{q_2}}})$ are computationally indistinguishable against any polynomial-time quantum adversary. Note if we can prove that $F_{C_{q_1}}$ and $F_{C_{q_2}}$ are equivalent functions then we can conclude that the classical circuits ($C_{q_1}$ and $C_{q_2}$) that computes these functions are also be equivalent. Then from the definition of classical quantum secure $i\mathcal{O}$ we can conclude that  $iO(C_{F_{C_{q_2}}})$ and $iO(C_{F_{C_{q_2}}})$ are computationally indistinguishable against any polynomial-time quantum adversary. From Theorem \ref{sec:Clifford Functions} we have $F_{C_{q_1}}$ and $F_{C_{q_2}}$ equivalent functions, therefore $Qi\mathcal{O}$ is a quantum secure indistinguishability obfuscation for all Clifford Circuits.\footnote{Note without loss of generality we can assume that $|C_{F_{C_{q_1}}}|=|C_{F_{C_{q_2}}}|,$ because we can always increase the size of a circuit by adding trivial gates to it. }
\end{itemize}


\begin{theorem}\label{sec:Clifford Functions}
Let $C_{q_1}$ and $C_{q_2}$ be two equivalent Clifford circuits and  $F_{C_{q_1}}$ and $F_{C_{q_2}}$  denote the update functions for the circuits $C_{q_1}$ and $C_{q_2},$ then $F_{C_{q_1}}({\bf s})=F_{C_{q_2}}({\bf s})$ for every ${\bf s}\in\{0,1\}^{2n}.$
\end{theorem}

\begin{flushleft}
{\bf Proof:} For any {\em n}-qubit state $\ket \psi$ we have $C_{q_1}(\ket \psi)=C_{q_2}(\ket \psi)$ and let $F_{C_{q_1}}$ and $F_{C_{q_2}}$ denote the update functions for the circuits $C_{q_1}$ and $C_{q_2}.$ Suppose $F_{C_{q_1}}\neq F_{C_{q_2}},$  then there must exist at least one binary string $a_1b_1,\ldots, a_nb_n$ such that
\begin{equation*}
F_{C_{q_1}}(a_1,b_1,\ldots,a_n,b_n)\neq F_{C_{q_2}}(a_1,b_1,\ldots,a_n,b_n)
\end{equation*}

From the above (see indistinguishability) we have (see equation \ref{eq:1})
\begin{equation*}
\left(\rho ,\left(B_q, i\mathcal{O}(C_{F_{C_{q_1}}})\right)\right)= Qi\mathcal{O}(C_{q_1}) \mbox{ and } \left(\rho ,\left(B_q, i\mathcal{O}(C_{F_{C_{q_2}}})\right)\right)= Qi\mathcal{O}(C_{q_2}).
\end{equation*}

Suppose we want to evaluate circuits on some state $\ket \psi$ follows that for both circuits $C_{q_1}$ and $C_{q_2}$ in algorithm 3 we have the same state $\rho_{AB}=\rho\otimes \ket \psi$ (see algorithm 3). Therefore, the classical output bits we obtain after performing the Bell measurement in algorithm 3 are independent of the circuits $C_{q_1}$ and $C_{q_2}$). Suppose for both circuits we obtain the same classical outputs $a_1b_1,\ldots, a_nb_n.$ From algorithm 3 we have

 \begin{equation}
  \label{eq:2}
C_q(\xgate^{\otimes_{i=1}^{n} b_{i}} \cdot \zgate^{\otimes_{i=1}^{n}})\ket{\psi}=C_q^\prime (\xgate^{\otimes_{i=1}^{n} b_{i}} \cdot \zgate^{\otimes_{i=1}^{n}})\ket{\psi}
\end{equation}
Let $s=a_1b_1,\ldots, a_nb_n$ and let

 \begin{equation}
 \label{eq:3}
F_{C_{q_1}}(s)=(a^\prime_1,b^\prime_1,\ldots, a^\prime_n,b^\prime_n) \mbox{ and } F_{C_{q_2}}(s)=(d_1^\prime,e_1^\prime,\ldots, d_n^\prime,e^\prime_n)
\end{equation}

We can rewrite equation \ref{eq:2} as (see gate teleportation \ref{protocol: gate-teleportation} or algorithm 3.
 \begin{equation}
  \label{eq:4}
(\zgate^{\otimes_{i=1}^{n} a_{i}^\prime} \cdot \xgate^{\otimes_{i=1}^{n} b_{i}^\prime})C_q(\ket{\psi})=(\zgate^{\otimes_{i=1}^{n} d_{i}^\prime} \cdot \xgate^{\otimes_{i=1}^{n} e_{i}^\prime})C_q^\prime(\ket{\psi})
\end{equation}

Since $C_q^\prime(\ket{\psi})=C_q^\prime(\ket{\psi})$ we can substitute $C_q^\prime$ for $C_q^\prime$ in equation \ref{eq:4}

\begin{equation}
  \label{eq:5}
(\zgate^{\otimes_{i=1}^{n} a_{i}^\prime} \cdot \xgate^{\otimes_{i=1}^{n} b_{i}^\prime})C_q(\ket{\psi})=(\zgate^{\otimes_{i=1}^{n} d_{i}^\prime} \cdot \xgate^{\otimes_{i=1}^{n} e_{i}^\prime})C_q(\ket{\psi})
\end{equation}

It follows from equation \ref{eq:5} that
\begin{equation}
  \label{eq:6}
(\zgate^{\otimes_{i=1}^{n} a_{i}^\prime} \cdot \xgate^{\otimes_{i=1}^{n} b_{i}^\prime})=(\zgate^{\otimes_{i=1}^{n} d_{i}^\prime} \cdot \xgate^{\otimes_{i=1}^{n} e_{i}^\prime})
\end{equation}

It follows from equation \ref{eq:6}  that $$\zgate^{a_i^\prime} \cdot \xgate^{b_i^\prime}=\zgate^{d_i^\prime} \cdot \xgate^{e_i^\prime} \mbox{ for all } i\in[n].$$ This means $a_i^\prime=d_i^\prime$ and  $b_i^\prime=e_i^\prime$ and therefore, if $C_{q_1}$ and $C_{q_2},$ are two equivalent circuits, then they have the same update function.
\end{flushleft}





%\subsection{Indistinguishability Obfuscation for the Clifford Update Functions}
%\label{sec: iO-clifford-functions}
%
%
%\anote{I am not sure how am I to understand that the $F_{C_q}$ are Cliffords.... I thought they were classical circuits?}
%
%In this section we present a (quantum secure classical) $i\mathcal{O}$ that can obfuscate the update function $F_{C_q},$ for any $n$-qubit Clifford circuit $C_q.$
%
%\begin{algorithm}[H]
%   \caption{$i\mathcal{O}$ for Clifford update Functions $F_{\tt Clifford}$}
%  \begin{itemize}
%  \item Input: A $n$-qubit Clifford Circuit $C_q,$ a security parameter.
%  \begin{enumerate}
%  \item Compute $C_{q_c}\leftarrow$ {\tt AG-Canonical-Unique}$(C_q).$
%  \item Compute the circuit $C_{F_{q_c}}$ for the update function $F_{C_{q_c}}.$
%  \item Output $C_{F_{q_c}}.$
%  \end{enumerate}
%  \end{itemize}
%\end{algorithm}
