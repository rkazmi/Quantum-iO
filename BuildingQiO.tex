
%%%%%%%%%%%%%%%%%%%%%%%%%%%%%%%%%%%%%%%%%%%%%%%%%%%%
%Section
\section{Quantum Indistinguishability Obfuscation}
\label{sec:QiO-Cliffords and more}
 In this section we 
\anote{Need some blah blah at the beginning of a section to say what we present in this section.}
 %Here, we show how to achieve iO for some families of quantum circuits.
%We give two alternative methods, the first based on canonical forms, and the second based on gate teleportation.
%% can be obfuscated using a quantum-secure indistinguishability obfuscator.

\subsection{Definitions}
\label{sec:iO-Cliffords and more}
\begin{definition} {\rm (Equivalent Quantum Circuits):}
\label{def:eqcircuit}
Let $C_{q_0}$ and $C_{q_1}$ be two $n$-qubit quantum circuits. We say $C_{q_0}$ and $C_{q_1}$ are equivalent if for every state $\ket{\psi}\in\mathit{Q}(n)$
$$C_{q_1}(\ket{\psi}) =C_{q_2}(\ket{\psi}).$$
\end{definition}
\anote{Need some blah blah before an important definition to describe what it is saying, but in words}

\anote{why is computational in brackets below?}
\begin{definition}\label{def:QiO} {\rm (Quantum Indistinguishability Obfuscation)}
A  polynomial-time quantum algorithm for a class of quantum circuits $\mathcal{C}_Q$ is a quantum {\rm(}computational{\rm)} indistinguishability obfuscator $Qi\mathcal{O}$  if the following conditions hold:

\begin{enumerate}
\item {\tt Functionality:}  For every $C_q\in \mathcal{C}_Q$ and every quantum state $\ket{\phi},$
$$(\rho, C_q^\prime)\leftarrow Qi\mathcal{O}(C_q) \;  \mbox{ and }\; C_q^\prime(\rho,\ket{\phi})=C_{q}(\ket{\phi})$$
%where $\rho \in Q(m)$ and $C_q^\prime$  is a $m+n$ qubit quantum circuit.
											
\anote{above, we are mixing pure states such as $\ket{\phi}$ with mixed states, such as $\rho$. This gets confusing, but I don't know exacly how to fix it. Maybe $\rho$ could be assumed wlog to be pure, so we can use use $\ket{\psi}$ instead of $\rho?$. }

\item  {\tt Polynomial Slowdown:}  For every $C_{q}\in \mathcal{C}_Q,$
\begin{itemize}
\item  $\rho$ is at most a $poly(|C_{q}|)$ qubit state.
\item $|C_{q}^\prime| \in poly(|C_{q}|).$
\end{itemize}
\anote{in below, did we define what we mean by size? If this is the number of gates, I would find the definition quite restrictive. Maybe you mean the same dimensions for input- output?}

\item {\tt Indistinguishability:} For any two equivalent quantum circuits $C_{q_1},C_{q_2}\in \mathcal{C}_Q,$ of the same size
 and for every polynomial-time quantum distinguisher $\mathcal{D}_q,$ there exists a negligible function {\rm negl} such that:

	
					$$\Big |{\rm Pr}[\mathcal{D}_q(Qi\mathcal{O}(C_{q_1}))=1]-{\rm Pr}[\mathcal{D}_q(Qi\mathcal{O}(C_{q_2}))=1] \Big |\leq  {\rm negl}(k).$$		
Where $k=|C_{q_1}|=|C_{q_2}|.$						
\end{enumerate}

\end{definition}


\subsection{Quantum Indistinguishability Obfuscation for Clifford Circuits}
 In this section we present two methods to obfuscate any Clifford circuit,  one using a canonical form and the other using gate teleportation.

\subsubsection{Obfuscating Clifford Using the Canonical Form}
\label{sec:Clifford-iO-canonical}

\anote{This entire section is very cryptic. We need to have a good intuitive understanding of why their canonical form works, and then explain this to the reader. Currently, it seems like magic. I'm also unsure about this part taking up so much space- with essentially zero contribution from us, but I don't have an alternative. I am thinking about it.}
\anote{For instance, is there a relationship between the $f$ update functions in the preliminaries, and the update functions here for the Cliffords?}

Aaronson and Gottesman invented a polynomial-time algorithm that computed a canonical form of any Clifford circuit~\cite{AG04}. Their algorithm (we denote as {\tt AG-Canonical}, page 8 - 9 \cite{AG04})\anote{pages 8-9 of the AG reference? I think it is more standard to cite the Section numbers, and make sure you check the published version, not the arXiv version.}takes a tableau (binary matrix) related to a Clifford circuit as an input and output its canonical form.  However, the {\tt AG-Canonical} lacks two properties that our important for obtaining $Qi\mathcal{O}.$ First, it does not take a circuit as an input, but rather a $2n \times 2n$ binary matrix representing some quantum state and second, the canonical form computed by it may not always be unique. In this section we show that how one can easily tweak {\tt AG-Canonical}, so that it satisfies both of the properties and allow us to construct a  $Qi\mathcal{O}$ for all Clifford Circuits. We start by briefly describing  {\tt AG-Canonical}.

\anote{excellent lead-in paragraph above!}

The algorithm  represents an $n$-qubit state by tableau consisting of $2n$ rows and $2n+1$ columns, where $x_{i,j} ,z_{i,j}\in \{0,1\}.$


\[
\left(
\begin{tabular}
[c]{ccc|ccc|c}%
$z_{11}$ & $\cdots$ & $z_{1n}$ & $x_{11}$ & $\cdots$ & $x_{1n}$ & $r_{1}$\\
$\vdots$ & $\ddots$ & $\vdots$ & $\vdots$ & $\ddots$ & $\vdots$ & $\vdots$\\
$z_{n1}$ & $\cdots$ & $z_{nn}$ & $x_{n1}$ & $\cdots$ & $x_{nn}$ & $r_{n}%
$\\\hline
$z_{\left(  n+1\right)  1}$ & $\cdots$ & $z_{\left(  n+1\right)  n}$ &
$x_{\left(  n+1\right)  1}$ & $\cdots$ & $x_{\left(  n+1\right)  n}$ &
$r_{n+1}$\\
$\vdots$ & $\ddots$ & $\vdots$ & $\vdots$ & $\ddots$ & $\vdots$ & $\vdots$\\
$z_{\left(  2n\right)  1}$ & $\cdots$ & $z_{\left(  2n\right)  n}$ &
$x_{\left(  2n\right)  1}$ & $\cdots$ & $x_{\left(  2n\right)  n}$ & $r_{2n}$%
\end{tabular}
\ \ \ \ \right)
\]

Each row  of the tableau $R_{i}=\pm (P_{1}\otimes \cdots \otimes P_{n})\in \mathcal{P}_n$, for $1\leq i\leq 2n,$ where bits $(x_{ij},z_{ij})$ determine the $j^{th}$ Pauli gate $P_{j}$: $00$
means $\igate$, $01$ means $\xgate$, $11$ means~$\ygate$, and $10$ means~$\zgate$ and $r_{i}$ is $1$ if $R_{i}$ has negative phase and $0$ if $r_{i}$\ has
positive phase. Rows $1$ to $n$ of the tableau represent the destabilizer generators $R_{1},\ldots,R_{n}$ and rows $n+1$ to $2n$ represent the stabilizer
generators $R_{n+1},\ldots,R_{2n}$ of the state it represents. A state is uniquely determined by the group of Pauli operators that stabilize it.

For example for the $2$-qubit state $\ket{00}$ a possible tableau is
\[
\left(
\begin{tabular}
[c]{cc|cc|c}%
$~1~$ & $~0~$ & $~0~$ & $~0~$ & $~0~$\\
$~0~$ & $~1~$ & $~0~$ & $~0~$ & $~0~$\\\hline
$~0~$ & $~0~$ & $~1~$ & $~0~$ & $~0~$\\
$~0~$ & $~0~$ & $~0~$ & $~1~$ & $~0~$%
\end{tabular}
\right)
\]

and the stabilizer generator for the state are $R_3=+(\zgate \otimes \igate)$ and $R_4=+(\igate\otimes \zgate).$ Note if we swap row 3 with row 4, then we get another tableau for the state $\ket{00},$ therefore a quantum state can be represented by more than one tableau. Now suppose we want to compute a canonical form of an $n$-qubit Clifford circuit $C_q.$ Let $\mathcal{T}_i$ be a initial tableau representing the state $\ket{0}^{\otimes n}.$ As we proceed through the gates in $C_q;$ we update $\mathcal{T}_i$  according to the following rules \cite{AG04}.\\


\noindent\textbf{CNOT from control }$a$\textbf{ to target }$b$\textbf{.} \ For
all $i\in\left\{  1,\ldots,2n\right\}  $, set $r_{i}:=r_{i}\oplus x_{ia}%
z_{ib}\left(  x_{ib}\oplus z_{ia}\oplus1\right)  $, $x_{ib}:=x_{ib}\oplus
x_{ia}$, and $z_{ia}:=z_{ia}\oplus z_{ib}$.\medskip

\noindent\textbf{Hadamard on qubit }$a$\textbf{.} \ For all $i\in\left\{
1,\ldots,2n\right\}  $, set $r_{i}:=r_{i}\oplus x_{ia}z_{ia}$\ and swap
$x_{ia}$\ with $z_{ia}$.\medskip

\anote{todo: make sure we say somewhere (in prelims) that Phase gate is $\pgate$.    }
\noindent\textbf{Phase on qubit }$a$\textbf{.} \ For all $i\in\left\{
1,\ldots,2n\right\}  $, set $r_{i}:=r_{i}\oplus x_{ia}z_{ia}$ and then set
$z_{ia}:=z_{ia}\oplus x_{ia}$.\medskip



Let $\mathcal{T}_f$ be the final tableau representing the state $C_q(\ket{0}^{\otimes n}).$ The algorithm  {\tt AG-Canonical}  takes the $2n\times 2n$ entries of $\mathcal{T}_f$ (ignore the phase bits $r_i$) as  input and computes a canonical form as follows: First write the tableau into four $n\times n$ matrices $A, B, C$ and $D$
\[
\left[
\begin{tabular}
[c]{c|c}
A&B \\\hline
C&D
\end{tabular}
\right]
\]

\anote{really, 11 rounds? What is so special about this number?}


apply 11 rounds  $\hgate-\cnot-\pgate-\cnot-\pgate-\cnot-\hgate-\pgate-\cnot-\pgate-\cnot$ to the above block matrix to obtain standard initial tableau $[{\bf I}_{2n}|{\bf 0}].$ Let $(g_i)_{i=1}^k,\, k\in\mathbb{N}$ be the sequence of gates corresponding to these rounds. Then the canonical form of the circuit $C_q$ is given be the reverse of the sequence $(g_i)_{i=1}^k.$ Now suppose $C_q^\prime$  is another circuit that is equivalent to
$C_q$ and $\mathcal{T}_i^\prime \neq \mathcal{T}_i$ be a initial tableau representing the state $\ket{0}^{\otimes n}.$ Let $\mathcal{T}_f^\prime$ be the final tableau representing the state $C_q^\prime(\ket{0}^{\otimes n}).$ Then it is possible that $\mathcal{T}_f\neq \mathcal{T}_f^\prime$ and in this case the sequence $(g_i^\prime )_{i=1}^{k^\prime} \leftarrow {\tt AG-Canonical}(\mathcal{T}_f^\prime)$ is different from $(g_i)_{i=1}^k,\, k\in\mathbb{N}.$ Therefore, in general the output of {\tt AG-Canonical}  can be different for two equivalent circuits.
\anote{The above needs to be emphasized earlier on- the cannonical form of AG is not actually unique}
\anote{I think there is another reason why this form is not unique. There are possibly many choices of the gates that would lead to the standard initial tableau (think: you can chose in what order you do CNOTs, if the qubits are disjoint, you can decide to do a P now, or later, depending on the intermediate configuration, etc. So, I don't think we have the full proof that this is a unique canonical form. \cite{AG04} only claim that you can put your stabilizer in the form of 11 rounds of $H-C-P-C-P-C-H-P-C-P-C$, they do not claim that the specific decomposition is unique. }
%This can also be demonstrated by a counter example, Let $C_q= \igate \otimes  \igate$ and $C_q^\prime= (\igate \otimes  \hgate)\cdot (\igate \otimes  \hgate)$ are equivalent circuits.  Let $\mathcal{T}_i$ and $\mathcal{T}_i^\prime$ denote initial tableaus for the state $\ket{0}^{\otimes 2}.$
%
% Then it is possible that $\mathcal{T}_f\neq \mathcal{T}_f^\prime.$ Moreover, the algorithm  {\tt AG-Canonical} takes $\mathcal{T}_f$ as a input and compute a canonical form as follows: First apply $\cnot, \hgate$ and $\pgate$  to $\mathcal{T}_f$ in a sequence (denoted by $S_G$) to obtain standard initial tableau  $[{\bf I}_{2n}|{\bf 0}].$ Then the canonical form of $C_q$ is given by the sequence $S_G.$ This means, the output of {\tt AG-Canonical}  may be different for $C_q$ from $C_q^\prime$ if $\mathcal{T}_f\neq \mathcal{T}_f^\prime.$  For example, $C_q= \igate \otimes  \igate$ and $C_q^\prime= (\igate \otimes  \hgate)\cdot (\igate \otimes  \hgate)$ are equivalent circuits. Let Let $\mathcal{T}_i$ and $\mathcal{T}_i^\prime$ denote initial tableaus for the state $\ket{0}^{\otimes 2}.$
%
% \[
%\mathcal{T}_i=\left(
%\begin{tabular}
%[c]{cc|cc|c}%
%$~1~$ & $~0~$ & $~0~$ & $~0~$ & $~0~$\\
%$~0~$ & $~1~$ & $~0~$ & $~0~$ & $~0~$\\\hline
%$~0~$ & $~0~$ & $~1~$ & $~0~$ & $~0~$\\
%$~0~$ & $~0~$ & $~0~$ & $~1~$ & $~0~$%
%\end{tabular}
%\right) \; \mbox{ and }\;  \mathcal{T}_i^\prime=\left(
%\begin{tabular}
%[c]{cc|cc|c}%
%$~1~$ & $~0~$ & $~0~$ & $~0~$ & $~0~$\\
%$~0~$ & $~1~$ & $~0~$ & $~0~$ & $~0~$\\\hline
%$~0~$ & $~0~$ & $~0~$ & $~1~$ & $~0~$\\
%$~0~$ & $~0~$ & $~1~$ & $~0~$ & $~0~$
%\end{tabular}
%\right)
%\]
%
%After running  $C_q$ and $C_q^\prime$ on the state $\ket{0}^{\otimes 2}$ and updating their initial tableaus according to the above rules, we have $\mathcal{T}_f=\mathcal{T}_i \neq \mathcal{T}_f^\prime=\mathcal{T}_i^\prime.$ Then we calculate the canonical form of  using {\tt AG-Canonical} on input  $\mathcal{T}_f$ and $\mathcal{T}_f^\prime,$ then we cannot possibly apply to both final tableau the same sequence of $\cnot, \hgate$ and $\pgate$ to obtain the initial tableau, therefore the output of {\tt AG-Canonical} is a different canonical form for $C_q$ from $C_q^\prime.$

However, one can trivially modify the algorithm ({\tt AG-Canonical}) \cite{AG04} so that it take a Clifford circuit as an input and outputs a unique canonical form. We add the following initial lines of code to {\tt AG-Canonical} and called this new algorithm {\tt AG-Canonical-Unique}.




%Then  In fact, if
%. If we let
% \[
%\mathcal{T}_i=\left(
%\begin{tabular}
%[c]{cc|cc|c}%
%$~1~$ & $~0~$ & $~0~$ & $~0~$ & $~0~$\\
%$~0~$ & $~1~$ & $~0~$ & $~0~$ & $~0~$\\\hline
%$~0~$ & $~0~$ & $~1~$ & $~0~$ & $~0~$\\
%$~0~$ & $~0~$ & $~0~$ & $~1~$ & $~0~$%
%\end{tabular}
%\right) \; \mbox{ and }\;  \mathcal{T}_i^\prime=\left(
%\begin{tabular}
%[c]{cc|cc|c}%
%$~1~$ & $~0~$ & $~0~$ & $~0~$ & $~0~$\\
%$~0~$ & $~1~$ & $~0~$ & $~0~$ & $~0~$\\\hline
%$~0~$ & $~0~$ & $~0~$ & $~1~$ & $~0~$\\
%$~0~$ & $~0~$ & $~1~$ & $~0~$ & $~0~$
%\end{tabular}
%\right)
%\]
% $$\igate \otimes  \igate\leftarrow{\tt AG-Canonical}(T_f) \; \mbox{  and  } \; \igate \otimes  \igate\leftarrow{\tt AG-Canonical}(T_f^\prime) $$
%  we have
%  above the canonical form of  $\igate \otimes  \igate$


\begin{framed}
\begin{flushleft}
({\tt AG-Canonical-Unique})
\begin{itemize}
\item[] Input: A $n$-qubit Clifford Circuit $C_q.$
\begin{enumerate}
\item Set $\mathcal{T}_i=[{\bf I}_{2n}|{\bf 0}]\in \{0,1\}^{2n \times 2n+1}$ be the standard  initial tableau  representing the state $\ket{0}^{\otimes n}.$
\item Compute the final tableau $\mathcal{T}_f$ by running $\mathcal{T}_i$ on $C_q.$
\item Computes  the circuit $C_q^\prime \leftarrow$ {\tt AG-Canonical}$(\mathcal{T}_f).$
\item Output $C_q^\prime.$
\end{enumerate}
\end{itemize}
\end{flushleft}
\end{framed}

\begin{lemma}
The {\tt AG-Canonical-Unique} computes a unique canonical form for Clifford Circuits in polynomial-time.
\end{lemma}

\begin{proof}
The cost of computing the final tableau $\mathcal{T}_f$ is at most $O(n)$ gates (\cite{AG04}) and $|C_q^\prime |\in O(poly(|n|)$ (\cite{AG04}).  Note, for any $n$-qubit circuit $n\leq |C_q|,$ therefore the size of the algorithm {\tt AG-Canonical-Unique} is at most $poly(|C_q|)$ gates. Moreover, Aaronson and Gottesman proved in \cite{AG04} that if $C_{q_1}$ and $C_{q_2}$ are two Clifford circuits and $\mathcal{T}_{f_1}$ and $\mathcal{T}_{f_2}$ are their final tableaus obtained by running them on the standard initial tableau
$\mathcal{T}_{i_1}=\mathcal{T}_{i_2}=[{\bf I}_{2n}|{\bf 0}]$, then $C_{q_1}$ and $C_{q_2}$ are equivalent if and only if $\mathcal{T}_{f_1}=\mathcal{T}_{f_2}.$ This means the final tableau $\mathcal{T}_f$ obtain in {\tt AG-Canonical-Unique} is the same for every equivalent Clifford circuit. Therefore {\tt AG-Canonical-Unique} computes a unique canonical form for Clifford circuits.
\end{proof}

\anote{Regarding the above proof, as mentioned earlier, I am not convinced, because although we have idential tableaus $\mathcal{T}_{f_1}=\mathcal{T}_{f_2}$, the canonical form algorithm of \cite{AG04} does not necessarily give a unique decomposition into the 11 rounds of $H-C-P-C-P-C-H-P-C-P-C$.
I think that, instead of trying to fix the AG04 method, it might be easier and safer to go with Seiliger's method (see the .pdf under the DropBox folder References/Selinger-notesAnne.pdf). I think he proves that what he calls the "normal form" is unique for Cliffords.  }

\anote{should the following have a section number?}
\noindent{\bf Quantum Indistinguishability Obfuscator from Canonical Form}:  On input a Clifford circuit $C_q,$ the $Qi\mathcal{O}$ first computes the canonical form  {\tt AG-Canonical-Unique}$(C_q)$ and then outputs an empty register $\rho$ along with {\tt AG-Canonical-Unique}$(C_q).$

\begin{center}
$(\rho, \mbox{{\tt AG-Canonical-Unique}}(C_q))\leftarrow Qi\mathcal{O}(C_q).$
\end{center}
Note we output an empty register $\rho$ to be consistent with our definition of  $Qi\mathcal{O}.$ Clearly, $Qi\mathcal{O}$ satisfy the first two properties (functionality and polynomial slow down). Moreover the canonical form reveals no knowledge about the input circuit, therefore this $Qi\mathcal{O}$ is perfectly indistinguishable against any quantum adversary.

\anote{``reveals no knowledge'' is misleading. Actually, iO can reveal alot of knowledge. The point is that the output of the obfuscators are indistinguishable, which is the case.}

%The cost computing tableau $\mathcal{T}_f$ is bounded by a polynomial  $n$ (section III, ~\cite{AG04}). Moreover, if $C_{q_1}$ and $C_{q_2}$ are two equivalents circuits then, there final tableau as computed in step 3 are also equivalent $\mathcal{T}_{f_1}=\mathcal{T}_{f_2}.$ (Lemma 5 ~\cite{AG04}). This means the final tableau obtain in step 3 is independent of the input circuit. The output of the algorithm 1 is computed using the theorem 8 ~\cite{AG04} which only take the final tableau $\mathcal{T}_f$  and output an equivalent circuit $C_q^\prime$ which is completely determined by $\mathcal{T}_f.$ Moreover, the procedure describe in theorem 8 ~\cite{AG04} is a deterministic polynomial-time algorithm. Therefore, the algorithm 1 is guaranteed to output in polynomial-time a unique canonical form for any Clifford circuit.





\subsubsection{Quantum Indistinguishability Obfuscation using Gate Teleportation}
\label{sec:Clifford-iO-teleportaion}
In this section we  show that how gate teleportation can be used to obfuscate Clifford circuits. Our algorithm relies on the assumption that there exists a quantum secure $i\mathcal{O}$ for classical circuits, this seems problematic at first since, there is no provably quantum secure $i\mathcal{O}$ known for general classical circuits. 
\anote{Really? I thought there were some candidates.... }
However,  our construction relies on the assumption that a quantum secure $i\mathcal{O}$ exists for a very specific class of circuits that are associated with the update functions (section \ref{correction function}). In fact,  to construct a quantum secure $i\mathcal{O}$ for the family circuits (update functions) corresponding to Clifford circuits is quite easy (section \ref{sec: iO-clifford-functions}). \anote{Indeed, we're looking at some classical circuits that are probably quite easy to deal with. I'm not sure they are Cliffords, however, since I think they are classical functions.}



\anote{below, $\rho$ is usually a density matrix, which does not require a $\ket{\cdot}$.}
\anote{below, say something about what ``lower half'' means, or establish earlier on that another way to write the tensor product of $n$ Bell states is $\sum_i \ket{i}{i}$ (or something like this; need to check). Then it is clear which register we are talking about for applying $C_q$.}

\begin{algorithm}[H]
   \caption{$Qi\mathcal{O}$ using Gate Teleportation}
  \begin{itemize}
  \item Input: A $n$-qubit Clifford Circuit $C_q.$
  \begin{enumerate}
  \item Prepare $2n$ qubit Bell state $\ket{\beta^{ 2n}}=\ket{\beta_{00}}\otimes \cdots \otimes \ket{\beta_{00}}.$
  \item Apply the circuit $C_q$ on the lower half of the qubits $\ket{\rho}=(I\otimes C_q) \ket{\beta^{2n}}.$
  \item Compute the circuit $C_q^\prime$ (defined below).
  \item Output $\left(\ket{\rho},C_q^\prime \right).$
  \end{enumerate}
  %\item[-] Note we denote the classical circuit that computes the function $F_{C_q}$  also by $C.$
  \end{itemize}
\end{algorithm}


\anote{For step 2 below, shall we add to preliminaries that classical circuits can be embedded into quantum ones? I'm not sure, actually, how you envisage this classical part of the corrections- maybe I haven't fully understood yet.}
\begin{algorithm}[H]
\caption{Circuit $C_q^\prime$}
  \begin{itemize}
  \item Input  $\left(\ket{\rho}, \ket{\psi} \right)$
  \begin{enumerate}
  \item  Measure the first $2n$ qubits of the system $\ket{\rho}\otimes \ket{\psi}$ in the Bell basis and obtain the classical output $(a_1,b_1,\ldots,a_n,b_n).$ ({\em Note last $n$-qubits are now in the state} $C_q(\xgate^{\otimes_{i=1}^{n} b_{i}} \cdot \zgate^{\otimes_{i=1}^{n} a_{i}})(\ket{\phi})=(\xgate^{\otimes_{i=1}^{n} b_{i}^\prime} \cdot \zgate^{\otimes_{i=1}^{n} a_{i}^\prime})C_q(\ket{\phi})$ see section \ref{protocol: gate  teleportation}).
  \item Compute the correction bits $(a_1^\prime,b_1^\prime,\ldots,a_n^\prime,b_n^\prime)=i\mathcal{O}(C_{F_{C_q}})(a_1,b_1,\ldots,a_n,b_n).$ Where $C_{F_{C_q}}$ is the circuit that computes $F_{C_q}.$
  \item Apply $(\xgate^{\otimes_{i=1}^{n} b_{i}^\prime} \cdot \zgate^{\otimes_{i=1}^{n} a_{i}^\prime})$ to the system $C_q(\xgate^{\otimes_{i=1}^{n} b_{i}} \cdot \zgate^{\otimes_{i=1}^{n} a_{i}})(\ket{\phi})$  and output the state $C_q(\ket{\phi}).$
  \end{enumerate}
  \end{itemize}
\end{algorithm}

\begin{theorem}
If $i\mathcal{O}$ is a quantum secure classical indistinguishability obfuscation, then $Qi\mathcal{O}$ constructed from the gate teleportation protocol is a quantum Indistinguishability Obfuscation for all Clifford Circuits.
\end{theorem}

\anote{Note to self: I have not checked the below}

\noindent{\bf Proof}:
\begin{itemize}
\item  {\tt Functionality:} The functionality $Qi\mathcal{O}$ is followed from the gate teleportation.
\item {\tt Polynomial Slowdown:} Note $\rho$ is a $2n$-qubit state  and the size of $C_q^\prime$ is equal to the size of the circuits that computes the Bell measurement,  $i\mathcal{O}(F_{C_q})$ and $(\xgate^{\otimes_{i=1}^{n} b_{i}^\prime} \cdot \zgate^{\otimes_{i=1}^{n} a_{i}^\prime}.$ The Bell measurement of an $O(n)$ qubit system can be perform by a circuit of size $O(n)$ and clearly the size of the circuit  $(\xgate^{\otimes_{i=1}^{n} b_{i}^\prime} \cdot \zgate^{\otimes_{i=1}^{n} a_{i}^\prime})$ is in $O(n).$ The function $F_{C_q}$ takes  $2n$ bits as an input and perform {\bf XOR} or {\bf Swap} operations on it as its passes through each layer of the gates in $C_{q}.$ There are at most $|C_q|$ layers of gates, therefore the total number gates  required to compute $F_{C_q}$ is $poly(n)|C_q|.$ Since, for any $n$-qubit circuit $C_q,$ $n\leq |C_q|,$ the function $F_{C_q}$ can be computed by a circuit  of size at most $poly(|C_{q}|).$  Hence, the size of $|C_q^\prime | \in poly(|C_q|).$


\item  {\tt Indistinguishability:} Let $C_{q_1}$ and $C_{q_2}$ be two $n$-qubit equivalent Clifford circuits
$$\left(\ket{\rho_1},C_{q_1}^\prime \right)= Qi\mathcal{O}(C_{q_1})
\;\mbox{ and }\;\left(\ket{\rho_2},C_{q_2}^\prime \right)= Qi\mathcal{O}(C_{q_2}) $$
\end{itemize}

Since $C_{q_1}(\ket{\tau})=C_{q_2}(\ket{\tau})$ for every quantum state $\ket{\tau}$ we have,

$$\ket{\rho_1}=(I\otimes C_{q_1}) \ket{\beta^{2n}}=(I\otimes C_{q_2}) \ket{\beta^{2n}}=\ket{\rho_2}.$$

so the first part of the $Qi\mathcal{O}$ output is perfectly indistinguishable. \anote{Yes, the above passage is correct.} Now for the second  part note that he only difference between  $C_{q_1}^\prime$ and $C_{q_2}^\prime$ are circuits that compute the update functions $F_{C_{q_1}}$ and  $F_{C_{q_2}}.$ If we can prove that $F_{C_{q_1}}({\bf s})=F_{C_{q_2}}({\bf s})$ for every  ${\bf s}\in\{0,1\}^{2n},$ then from definition of classical indistinguishability obfuscation we can conclude that  $iO(C_{F_{C_{q_2}}})$ and $iO(C_{F_{C_{q_2}}})$
are computationally indistinguishable against any quantum adversary \footnote{Note without loss of generality we can assume that $|C_{F_{C_{q_1}}}|=|C_{F_{C_{q_2}}}|,$ because we can always increase the size of a circuit by adding trivial gates to it. }

From Theorem \ref{sec:Clifford Functions} we have $F_{C_{q_1}}$ and $F_{C_{q_2}}$ equivalent functions, therefore $Qi\mathcal{O}$ is a quantum indistinguishability obfuscation for all Clifford Circuits.

\begin{theorem}\label{sec:Clifford Functions}
Let $C_{q_1}$ and $C_{q_2}$ be two equivalent Clifford circuits and  $F_{C_{q_1}}$ and $F_{C_{q_2}}$  denote the update functions for the circuits $C_{q_1}$ and $C_{q_2},$ then $F_{C_{q_1}}({\bf s})=F_{C_{q_2}}({\bf s})$ for every ${\bf s}\in\{0,1\}^{2n}.$
\end{theorem}


\begin{flushleft}
{\bf Proof:} For any {\em n}-qubit state $\ket \psi$ we have $C_{q_1}(\ket \psi)=C_{q_2}(\ket \psi).$ If we use Circuit teleportation protocol to compute $C_1(\ket \psi)$ and $C_{q_2}(\ket \psi),$ After measurement and tracing out the Bell states \anote{I'm not sure which part you are tracing out?} the system corresponding to the circuit $C_{q_1}$ and $C_{q_2}$ are in the states.

$$C_{q_1} ({\rm {\rm U}}\ket \psi)={\rm {\rm U}}_{C_{q_1}}C_{q_1}(\ket \psi)  \;\mbox{ and }\; C_{q_2} ({\rm {\rm U}}\ket \psi)={\rm {\rm U}}_{C_{q_2}}(C_{q_2}(\ket \psi))$$
Where

 $${\rm {\rm U}}=( X^{\otimes_{i=1}^{n} b_{i}} \cdot Z^{\otimes_{i=1}^{n} a_{i}}),$$
$${\rm {\rm U}}_{C_{q_1}}=(X^{\otimes_{i=1}^{n} b^\prime_{i}} \cdot Z^{\otimes_{i=1}^{n} a^\prime_{i}}),$$
$${\rm {\rm U}}_{C_{q_2}}=(X^{\otimes_{i=1}^{n} e^\prime_{i}} \cdot Z^{\otimes_{i=1}^{n} d^\prime_{i}}).$$
$$F_{C_{q_1}}(a_1,b_1,\ldots, a_n,b_n)=(a^\prime_1,b^\prime_1,\ldots, a^\prime_n,b^\prime_n) \;\mbox{ and }\; F_{C_{q_2}}(d_1,e_1,\ldots, d_n,e_n)=(d_1^\prime,e_1^\prime,\ldots, d_n^\prime,e^\prime_n)$$

We have

$$C_{q_1}(\ket \psi)=({\rm {\rm U}}_{C_{q_1}}^\dagger {\rm {\rm U}}_{C_{q_1}})C_{q_1}(\ket \psi)={\rm {\rm U}}_{C_{q_1}}^\dagger ( C_{q_1}({\rm {\rm U}}(\ket \psi))$$
$$={\rm {\rm U}}_{C_{q_1}}^\dagger ( C_{q_2}({\rm {\rm U}}(\ket \psi)),  {\color{blue}[C_{q_1} \mbox{ and } C_{q_2} \mbox{ are equivalent } ]}$$
$$={\rm {\rm U}}_{C_{q_1}}^\dagger {\rm {\rm U}}_{C_{q_2}}(C_{q_2}(\ket \psi))=C_{q_2}(\ket \psi)$$
$$\Longrightarrow {\rm {\rm U}}_{C_{q_1}}^\dagger {\rm {\rm U}}_{C_{q_2}}={\bf I}\Longleftrightarrow {\rm {\rm U}}_{C_{q_1}}={\rm {\rm U}}_{C_{q_2}}$$


By Lemma \ref{sec:Clifford Functions Equivalence}  we have that $F_{C_{q_1}}=F_{C_{q_2}}.$
\end{flushleft}

\begin{lemma}
\label{sec:Clifford Functions Equivalence}
Let $C_{q_1}$ and $C_{q_2}$ be two equivalent Clifford circuits, then $F_{C_{q_1}}=F_{C_{q_2}}$ if and only if ${\rm U}_{C_{q_1}}={\rm U}_{C_{q_2}}.$
\end{lemma}
{\bf Proof:} ($p\Rightarrow q)$ Suppose $F_{C_{q_1}}=F_{C_{q_2}}$ then clearly ${\rm U}_{C_{q_1}}={\rm U}_{C_{q_2}}.$\\

		Now to proof $(q\Rightarrow p)$ Suppose ${\rm U}_{C_{q_1}}={\rm U}_{C_{q_2}}$ and $F_{C_{q_1}}\neq F_{C_{q_2}},$ then there exists a binary string $(a_1,b_1,\ldots, a_n,b_n)$ such that
		  $$ F_{C_{q_1}}(a_1,b_1,\ldots, a_n,b_n)\neq F_{C_{q_2}}(a_1,b_1,\ldots, a_n,b_n)$$
		  Since ${\rm U}_{C_{q_1}}({\ket \psi})={\rm U}_{C_{q_2}}({\ket \psi})$ for every {\em n}-qubit state ${\ket \psi}.$ Let  $${\ket \psi}={\ket {\phi_1}} \otimes {\ket {\phi_2}}\otimes \cdots \otimes {\ket {\phi_n}}$$
		   where ${\ket {\phi_i}} =\alpha_i {\ket 0}+ \beta_i{\ket 1},$  $\alpha_i\neq 0$ and $\beta_i\neq 0,$ for every integer $1\leq i\leq n.$
		
		  $${\rm U}_{C_{q_1}}({\ket \psi})=\left(\alpha_1 {\ket {0+a^\prime_1}}+(-1)^{b^\prime_1} (i)^{c^\prime_1} \beta_1{\ket 1}\right) \otimes \dots \otimes  \left(\alpha_n {\ket {0+a^\prime_n}}+(-1)^{b^\prime_n} (i)^{c^\prime_i} \beta_n{\ket 1}\right)$$
		
		  $${\rm U}_{C_{q_2}}({\ket \psi})=\left(\alpha_1 {\ket {0+d^\prime_1}}+(-1)^{e^\prime_1} (i)^{f^\prime_1} \beta_1{\ket 1}\right) \otimes \dots  \otimes   \left(\alpha_n {\ket {0+d^\prime_n}}+(-1)^{e^\prime_n} (i)^{f^\prime_i} \beta_n{\ket 1}\right)$$
		
		  For any $i$ if  $a^\prime_i\neq d^\prime_i$ or $b^\prime_i\neq e^\prime_i$ or $c^\prime_i\neq f^\prime_i$ then the output ${\rm U}_{C_{q_1}}({\ket \psi})$ will differ from ${\rm U}_{C_{q_2}}({\ket \psi})$ on the {\em i}-th qubit. Which is a contraction to the assumption that ${\rm U}_{C_{q_1}}={\rm U}_{C_{q_2}}.$  Therefore $F_{C_{q_1}}=F_{C_{q_2}}.$



\subsection{Indistinguishability Obfuscation for the Clifford Update Functions}
\label{sec: iO-clifford-functions}


\anote{I am not sure how am I to understand that the $F_{C_q}$ are Cliffords.... I thought they were classical circuits?}

In this section we present a (quantum secure classical) $i\mathcal{O}$ that can obfuscate the update function $F_{C_q},$ for any $n$-qubit Clifford circuit $C_q.$

\begin{algorithm}[H]
   \caption{$i\mathcal{O}$ for Clifford update Functions $F_{\tt Clifford}$}
  \begin{itemize}
  \item Input: A $n$-qubit Clifford Circuit $C_q,$ a security parameter.
  \begin{enumerate}
  \item Compute $C_{q_c}\leftarrow$ {\tt AG-Canonical-Unique}$(C_q).$
  \item Compute the circuit $C_{F_{q_c}}$ for the update function $F_{C_{q_c}}.$
  \item Output $C_{F_{q_c}}.$
  \end{enumerate}
  \end{itemize}
\end{algorithm}

\section{Obfuscating Beyond Clifford Circuits}
Here, we show how to build iO for quantum circuit families that satisfy Definition \ref{defn:Clifford+T:family}.
\anote{Is this the correct link to the Definition?}

\subsubsection{Using a Canonical Form}
Main idea: Each Clifford layer is obfuscated using the canonical form. The $\tgate$ gate layers are given in the clear. The result is a canonical form, since the family is selected such that the description just given will be a canonical form.
\anote{I think this one is less interesting because the $T$ gates are in the clear. Maybe drop it?}


\subsubsection{Using Gate Teleportation}
As mentioned in  Section~\ref{defn:Clifford+T:family}, adding a non-Clifford gates  such as $\tgate$ to Clifford gates gives us a generating set for all quantum circuits. The $\tgate$ relates to the $\xgate, \zgate$
$$\tgate \xgate^b \zgate^a=\xgate^b \zgate^{a\oplus b}\pgate^b \tgate$$
Note  $\pgate=(\frac{1+i}{2}) \igate + (\frac{1-i}{2})\zgate,$
$$\tgate \xgate^b \zgate^a$$
$$=\xgate^b \zgate^{a\oplus b}\left[\left(\frac{1+i}{2}\right) \igate + \left(\frac{1-i}{2}\right)\zgate\right]^b \tgate$$
$$=\xgate^b \zgate^{a\oplus b}\left[\left(\frac{1+i}{2}\right) \igate + \left(\frac{1-i}{2}\right)\zgate^b\right] \tgate$$
\anote{For the last line, I would add some explanation, otherwise it looks like a mistake}
Now suppose we  want evaluate the circuit $\hgate \tgate$ on some qubit $\ket{\phi}$ using gate teleportation, then
 $$\hgate \tgate \xgate^b \zgate^a (\ket{\phi}) = \xgate^{b\oplus a} \zgate^b \hgate \pgate^b \tgate(\ket{\phi})$$
 If $b=0,$ then we don't have to worry about the $\pgate$ correction, otherwise we have to perform a $\pgate$ correction which is a problem since  $\hgate$ and $\pgate$ does not compute.\anote{commute?}
 $$\hgate \tgate \xgate^b \zgate^a (\ket{\phi}=\xgate^{a\oplus b} \zgate^b  \left[\left(\frac{1+i}{2}\right) \hgate + \left(\frac{1-i}{2}\right)\hgate\zgate^b\right] \tgate(\ket{\phi})$$
$$= \xgate^{a\oplus b} \zgate^b  \left[\left(\frac{1+i}{2}\right) \igate + \left(\frac{1-i}{2}\right)\xgate^b\right] \hgate\tgate(\ket{\phi})$$
$$=\left[\left(\frac{1+i}{2}\right)\xgate^{a\oplus b} \zgate^b  + (-1)^b\left(\frac{1-i}{2}\right)  \xgate^{a} \zgate^b\right]\hgate\tgate(\ket{\phi})$$
$$=\left[\alpha_1\xgate^{a\oplus b} \zgate^b  + (-1)^b\alpha_2  \xgate^{a} \zgate^b\right]\hgate\tgate(\ket{\phi})$$
 where $\alpha_1=\left(\frac{1+i}{2}\right)$ and $\alpha_2=\left(\frac{1-i}{2}\right).$ For this particular circuit, the update function requires at most two complex numbers and four bits.
 In general for any 1-qubit  unitary circuit we need to store at most four complex numbers and eights bits to represent the corresponding update function. To realize this, note that the set $\{\hgate,  \tgate\}$ is universal for 1-qubit unitary and any 1-qubit (unitary circuit) can be written as an alternating sequence of $\hgate,$ and $\tgate$ as follows,
\anote{I don't think the decomposition below is universal (see my email). Also, need to describe in preliminaries how to interpret this - from right-to-left or the other way?}

 $$C_q=\textsf{G}\mbox{ --- }\tgate \mbox{ --- } \hgate \mbox{ --- } \tgate \mbox{ --- } \hgate \mbox{ --- } \cdots \mbox{ --- } \tgate \mbox{ --- } \hgate \mbox{ --- } \tgate \mbox{ --- } \textsf{G}$$
  where $\textsf{G}=\hgate$ or $\textsf{G}=\igate.$

Lets compute this circuit gate by gate on some qubit $\ket{\phi}$ using gate teleportation
 \begin{enumerate}
\item[1)] $\textsf{G} \xgate^{b}\zgate^{a}(\ket{\phi})=\xgate^{b_1}\zgate^{a_1}\textsf{G}(\ket{\phi}),$
where $b_1=a,\, a_1=b,$ if $\textsf{G} =\hgate,$  otherwise   $b_1=b,\, a_1=a.$\\
\item[2)] $\tgate\xgate^{b_1}\zgate^{a_1}\textsf{G}= (\alpha_1\xgate^{a_1\oplus b_1 }\zgate^{b_1} +(-1)^{b_1} \alpha_2\xgate^{a_1}\zgate^{b_1})\tgate \textsf{G}.$
\item[3)]   $\hgate (\alpha_1\xgate^{a_1\oplus b_1 }\zgate^{b_1} + (-1)^{b_1} \alpha_2\alpha_2\xgate^{a_1}\zgate^{b_1})\tgate \textsf{G}=(\alpha_1\xgate^{b_1}\zgate^{a_1\oplus b_1 } + (-1)^{b_1} \alpha_2\xgate^{b_1}\zgate^{a_1}) \hgate \tgate \textsf{G}.$ \anote{typo: $\alpha_2\alpha_2$?}
\item[4)] $\tgate(\alpha_1\xgate^{b_1}\zgate^{a_1\oplus b_1 } + \alpha_2\xgate^{b_1}\zgate^{a_1}) \hgate \tgate \textsf{G}=\left(\frac{i}{2}\xgate^{a_1}\zgate^{a_1\oplus b_1 } +  \frac{(-1)^{b_1}}{2}\xgate^{a_1\oplus b_1}\zgate^{b_1}+
   \frac{(-1)^{b_1}}{2}\xgate^{a_1\oplus b_1}\zgate^{a_1}  - \frac{i}{2}\xgate^{a_1}\zgate^{b_1} \right) \tgate\hgate \tgate \textsf{G}.$

\item[5)]$\hgate  \left(\frac{i}{2}\xgate^{a_1}\zgate^{a_1\oplus b_1 } +  \frac{(-1)^{b_1}}{2}\xgate^{a_1\oplus b_1}\zgate^{b_1}+
   \frac{(-1)^{b_1}}{2}\xgate^{a_1\oplus b_1}\zgate^{a_1}  - \frac{i}{2}\xgate^{a_1}\zgate^{b_1} \right)\tgate\hgate \tgate \textsf{G}\\=  \left(\frac{i}{2}\xgate^{a_1\oplus b_1}\zgate^{a_1} +  \frac{(-1)^{b_1}}{2}\xgate^{b_1}\zgate^{a_1\oplus b_1}+
   \frac{(-1)^{b_1}}{2}\xgate^{a_1}\zgate^{a_1\oplus b_1}  - \frac{i}{2}\xgate^{b_1}\zgate^{a_1} \right)\hgate \tgate\hgate \tgate \textsf{G}$

   $=\left(c_1\igate+ c_2\xgate+c_3\zgate+c_4\xgate\zgate\right)\hgate\tgate\hgate \tgate \textsf{G},$  for $c_i\in \mathbb{C}.$
\item[6)]  $\tgate\left(c_1\igate+ c_2\xgate+c_3\zgate+c_4\xgate\zgate\right)\hgate\tgate\hgate \tgate \textsf{G}=\left(c_1\igate+ c_2\xgate \zgate\pgate+c_3\zgate +c_4\xgate\pgate\right)\tgate \xgate\zgate\hgate\tgate\hgate \tgate \textsf{G}$

 \hspace{2.3in}   $=\left(c_1\igate+ c_2^\prime\xgate+c_3\zgate+c_4^\prime\xgate\zgate\right)\tgate \xgate\zgate\hgate\tgate\hgate \tgate \textsf{G},$ \\
  where $c_2^\prime=(c_2\alpha_2+c_4\alpha_1)$ and $c_4^\prime=(c_2\alpha_1+c_4\alpha_2).$
  %$\tgate(\left(c_1\igate+ c_2\xgate+c_3\zgate+c_4\xgate\zgate\right)\hgate\tgate\hgate \tgate \textsf{G},\right)$
\item[-] Note applying $\tgate$ gate in the line 6 only affected the coefficients but didn't increase the number of terms in the summation. It is easy to see that after applying the remain gates in $C_q$ we  have the quantity
$$\left(\beta_1 \igate+\beta_2 \xgate+\beta_3 \zgate+\beta_4 \xgate \zgate\right) C_q(\ket{\phi}), \; \mbox{ where } \beta_i\in \mathbb{C}.$$
\anote{because the Paulis form a basis?}
\end{enumerate}
Furthermore, the correction matrix $\left(\beta_1 \igate+\beta_2 \xgate+\beta_3 \zgate+\beta_4 \xgate \zgate\right)^{-1}$  is completely determine by $(\beta_1, \beta_2, \beta_3, \beta_4).$ Therefore, for any arbitrary 1-qubit unitary circuit $C_q,$ the update function is $$F_{C_q}: \mathbb{F}_2^2 \longrightarrow \mathbb{C}^4,\hspace{0.5cm} (a,b)\longrightarrow (\beta_1,\beta_2,\beta_3, \beta_4).$$

% $$F_{C_q}: \mathbb{F}_2^2 \longrightarrow \mathbb{C}^4 \times  \mathbb{F}_2^8,\hspace{0.5cm} (a,b)\longrightarrow ((\alpha_1,\alpha_2,\alpha_3, \alpha_4), (a_1,b_1,a_2,b_2,a_3,b_3,a_4,b_4)).$$
% and corresponds to the correction unitary $\sum_{i=1}^{4} \alpha_i\xgate^{b_i}\zgate^{a_i}.$
% To realize this note that the set $\{\hgate,  \tgate\}$ is universal for 1-qubit unitary, therefore any 1-qubit (unitary circuit) can be written as an alternating sequence of $\hgate,$ and $\tgate$ as follows, where $\textsf{G}=\hgate$ or $\textsf{G}=\igate$
%
% $$\textsf{G}\mbox{ --- }\tgate \mbox{ --- } \hgate \mbox{ --- } \tgate \mbox{ --- } \hgate \mbox{ --- } \cdots \mbox{ --- } \tgate \mbox{ --- } \hgate \mbox{ --- } \tgate \mbox{ --- } \textsf{G}$$
%
% Moreover, the number terms in the summation $\sum_{i=1}^{4} c_i(\xgate^{a_i\oplus b_i }\zgate^{b_i} + \xgate^{a_i}\zgate^{b_i}),$ remain at most 4 when acted upon by $\tgate$ gate


How big  are the $\beta_i's $? Note each $\beta_i's$ are generated by multiplying and adding elements from the set $\{\left(\frac{1+i}{2}\right), \left(\frac{1-i}{2}\right)\}.$ Moreover, $\left(\frac{1\pm i}{2}\right)^k=\pm \frac{i}{2}$ and
$\left(\frac{1+ i}{2}\right)\left(\frac{1- i}{2}\right)=\frac{1}{2}$ and only $\tgate$ can have impact on the size of the coefficients in the summation. Therefore, the size of complex numbers grows polynomially in the number of $\tgate$ gates.
So we obfuscate any 1-qubit unitary circuit using gate teleportation protocol. \anote{no, mistake in the math above (see my email).}

\anote{For above, I don't think necessarily that the size of the $\beta$'s is a determining factor, but more the number of such $\beta$'s. Presumably, the $\tgate$ and $\cnot$ play together is a way that makes the number of different $\beta$s increase-- this is what we need to figure out.}

\anote{Did we define in preliminaries what the notation $\cnot_{2,1}$ means? Also, if we end up keeping this example, a picture of the circuit would be good}
\anote{For step 1, to which qubit is the $\tgate$ applied? I would have guessed the first, but looking at the update, I am not sure, it looks like it is applied to the second}

What about $n$-qubit circuits for $n>1$? As we will see that if a circuit has $k$ number of $\tgate$ gates, then in the worst-case the number of complex numbers required to specify grows super-polynomial in $k.$ The problem arises when $\cnot$ and $\tgate$ gates are mixed. To explain this let's consider the following 2-qubit circuit.
$$\cnot_{2,1}(\igate\otimes \tgate)\cnot_{1,2}(\tgate\otimes \igate)$$
Let compute this circuit gate by gate using gate-teleportation

 \begin{enumerate}
\item[1)]  $\tgate\otimes \igate \left[\xgate^{b_1}\zgate^{a_1}\otimes \xgate^{b_1^\prime}\zgate^{a_1^\prime}\right]=\left[\alpha_1 \xgate^{b_1}\zgate^{a_1} \otimes \xgate^{b_1^\prime}\zgate^{a_1^\prime \oplus b_1^\prime} + \alpha_2 \xgate^{b_1}\zgate^{a_1} \otimes \xgate^{b_1^\prime}\zgate^{a_1^\prime})\right]\tgate\otimes \igate.$
\item[2)] $\cnot_{1,2}\left[\alpha_1 \xgate^{b_1}\zgate^{a_1} \otimes \xgate^{b_1^\prime}\zgate^{a_1^\prime \oplus b_1^\prime} + \alpha_2 \xgate^{b_1}\zgate^{a_1} \otimes \xgate^{b_1^\prime}\zgate^{a_1^\prime}\right] (\tgate\otimes \igate)$

$\hspace{3.9cm}=\left[(\alpha_1 \xgate^{b_1}\zgate^{a_2} \otimes \xgate^{b_2}\zgate^{a_1^\prime \oplus b_1^\prime} + \alpha_2 \xgate^{b_1}\zgate^{a_2^\prime} \otimes \xgate^{b_2^\prime}\zgate^{a_1^\prime})\right]\cnot_{1,2}(\tgate\otimes \igate)$\\
where $a_2=a_1 \oplus a_1^\prime \oplus b_1^\prime,\, b_2=b_1 \oplus b_1^\prime$  and  $a_2^\prime=a_1 \oplus a_1^\prime,\, b_2^\prime=b_1 \oplus b_1^\prime.$
\item[3)] $\igate\otimes \tgate \left[\alpha_1 \xgate^{b_1}\zgate^{a_2} \otimes \xgate^{b_2}\zgate^{a_1^\prime \oplus b_1^\prime} + \alpha_2 \xgate^{b_1}\zgate^{a_2^\prime} \otimes \xgate^{b_2^\prime}\zgate^{a_1^\prime}\right]\cnot_{1,2}(\tgate\otimes \igate).$

$=\left[(\alpha_1^2 \xgate^{b_1}\zgate^{a_2} \otimes \xgate^{b_2}\zgate^{a_1^\prime \oplus b_1^\prime})+(\alpha_1\alpha_2 \xgate^{b_1}\zgate^{a_2\oplus b_1} \otimes \xgate^{b_2}\zgate^{a_1^\prime \oplus b_1^\prime})+ (\alpha_1\alpha_2 \xgate^{b_1}\zgate^{a_2^\prime} \otimes \xgate^{b_2^\prime}\zgate^{a_1^\prime})\right.$

$+ \left.(\alpha_2^2 \xgate^{b_1}\zgate^{a_2^\prime \oplus b_1} \otimes \xgate^{b_2^\prime}\zgate^{a_1^\prime})\right] (\igate\otimes \tgate)\cnot_{1,2}(\tgate\otimes \igate).$
\item[4)] Applying $\cnot_{2,1}$ to the above system.

$\left[(\alpha_1^2 \xgate^{b_1\oplus b_2}\zgate^{a_2} \otimes \xgate^{b_2}\zgate^{a_1^\prime \oplus b_1^\prime})+(\alpha_1\alpha_2 \xgate^{b_1\oplus b_2}\zgate^{a_2\oplus b_1} \otimes \xgate^{b_2}\zgate^{a_1^\prime \oplus b_1^\prime})+ (\alpha_1\alpha_2 \xgate^{b_1^\prime}\zgate^{a_2^\prime} \otimes \xgate^{b_2^\prime}\zgate^{a_1})+\right.$

$\left.(\alpha_2^2 \xgate^{b_1^\prime}\zgate^{a_2^\prime \oplus b_1} \otimes \xgate^{b_2}\zgate^{a_1})\right] \cnot_{2,1}(\igate\otimes \tgate)\cnot_{1,2}(\tgate\otimes \igate).$
$$=\left(\sum_{i=1}^{4} \beta_i \xgate^{e_i}\zgate^{d_i}\otimes \xgate^{e_i^\prime}\zgate^{d_i^\prime}\right)\cnot_{2,1}(\igate\otimes \tgate)\cnot_{1,2}(\tgate\otimes \igate);\; \beta_i\in \mathbb{C}, e_i,d_i,e_i^\prime,d_i^\prime \in\{0,1\}$$
\end{enumerate}

We observe the following 2 things from the above example:
\begin{enumerate}
\item[1)] We need 4 complex  numbers and 16 bits to represent the correction matrix.
\anote{For the above, I suspect we will be able to say that the $\beta$'s appear in a fixed pattern, e.g. variations on $e^{i\pi/4}$, so the ``only'' thing we will need to obfuscate is the exponents of the Paulis. For instance, we give a program that tells us if a fixed $\beta$ appears (yes/no), and next, if it appears, the program gives us the exponents for the Paulis (and also a sign). Thus, we'll have a bunch of such functions, and this is what we need to obfuscate using classical iO.}
\item[2)] $\cnot$ gates when combine with $\tgate$ gates increases the number of terms in the expression.
\end{enumerate}

For any 2-qubit quantum circuit  $C_q$ that has $k$ number of $\tgate$ gates, the correction unitary in the worst-case has $2^k$ terms;
         $$\sum_{i=1}^{2^k} \left(\beta_i \xgate^{e_i}\zgate^{d_i}\otimes \xgate^{e_i^\prime}\zgate^{d_i^\prime}\right),\; \mbox{ for } \beta_i\in \mathbb{C},\, e_i,d_i,e_i^\prime,d_i^\prime \in\{0,1\}.$$
This means we need $2^k$ complex numbers and $2^k\cdot 4$ bits to represent the correction unitary and the update function $f_{C_q}:\mathbb{C}^{2^k}\times \{0,1\}^{2^{k+2}}.$ Therefore, we can obfuscate any 2-qubit quantum circuit $C_q$ using gate-teleportation as far as  the number of $\tgate$  gates are at most $O(\log(|C_q|).$

\anote{These statements (above and below) need to be justified. It looks to me like they would be amenable to a proof by induction.  }

In, general, for any $n$-qubit circuit $C_q,$  the correction unitary can be written as $$\sum_{j=1}^{2^k} \left(\beta_j \xgate^{e_{1j}}\zgate^{d_{1j}}\otimes \xgate^{e_{2j}}\zgate^{d_{2j}}\otimes \cdots  \otimes  \xgate^{e_{2n}}\zgate^{d_{2n}}\right),$$
where $\beta_i\in \mathbb{C}$ and  $ e_i,j,d_i,j \in \{0,1\}.$  This means we need $2^k$ complex numbers and $2^k\cdot n$ bits to represent the correction unitary and the update function $f_{C_q}:\{0,1\}^n\longrightarrow \mathbb{C}^{2^k}\times \{0,1\}^{n2^{k}}.$
%$$\cnot_{2,1}\cnot_{1,2}(\tgate \otimes \tgate)((\xgate^{b_1} \zgate^{a_1})_1\otimes (\xgate^{b_2} \zgate^{a_2})_2)$$
%$$\cnot_{2,1}\cnot_{1,2}[(\alpha_1(\xgate^{a_1\oplus b_1} \zgate^{b_1})+\alpha_2(\xgate^{a_1\oplus b_1})]_1 \otimes [\alpha_2(\xgate^{a_2\oplus b_2}\zgate^{b_2})+ (\xgate^{b_2} \zgate^{a_2})]_2(\tgate \otimes \tgate)$$
%%$$\left(\frac{1+i}{2}\right)^{k_1} \left(\frac{1-i}{2}\right)^{k_2}=\left[\left(\frac{1+i}{2}\right) \left(\frac{1-i}{2}\right)\right]^k \left(\frac{1+i}{2}\right)^{k_1-k} \left(\frac{1-i}{2}\right)^{k_2-k},\; k=min\{k_1, k_2\} $$
%%$$=\frac{1}{2^k} \left(\frac{1+i}{2}\right)^{k_1-k} \left(\frac{1-i}{2}\right)^{k_2-k}$$
%%Suppose $k= k_1,$ and $k_2-k=2l$ then $\frac{1}{2^k} \left(\frac{1+i}{2}\right)^{k_1-k} \left(\frac{1-i}{2}\right)^{k_2-k}=(1-i)^{k_2-k}/2^{k_2}$ otherwise $\frac{1}{2^k} \left(\frac{1+i}{2}\right)^{k_1-k} \left(\frac{1-i}{2}\right)^{k_2-k}=(1+i)^{k_1-k}/2^{k_1}$
%%
%%If ${k_1-k}=2l$ is even then $(1-i)^{2l}/2^{k_2}=.$
%
%
%
% In general, to evaluate a $n$-qubit quantum circuit $C_q$ with $k$ number of $\tgate,$ the update function is
% $$F_{C_q}: \mathbb{F}_2^n \longrightarrow \mathbb{C}^{l_1} \times  \mathbb{F}_2^{l_2}, \mbox{ where } l_1=n\cdot 4^k, \; l_2=2n\cdot 4^k,$$
%
% $$\sum_{j=1}^{4^k} \left(\alpha_{1j}\xgate^{b_{1j}}\zgate^{a_{1j}} \otimes \alpha_{2j}\xgate^{b_{2j}}\zgate^{a_{2j}} \otimes \cdots \otimes \alpha_{nj}\xgate^{b_{nj}}\zgate^{a_{nj}}\right)$$
% where $a_{ij},b_{ij}\in \mathbb{F}_2, \alpha_{ij}\in \mathbb{C}$ and  $\frac{1}{2^k}\leq|\alpha_{ij}| \leq k.$ Therefore, we can efficiently obfuscate any quantum circuit as far as  the number of $\tgate$ gates in it are at most $O(\log(n)).$
%
%{\color{red}Here we use an idea related to \cite{BFK09,BJ15}. See Figure 8 in \cite{BJ15}. We leave EPR pairs at each T-gate layer, and after the gate teleportation, use iO on a program that will decide whether or not the $\pgate$-gate correction needs to be applied. Have to ask Anne about the relevance of \cite{BFK09} }

\anote{In general, the specific example discussed earlier should come out of the more general case. So, once we have the general case, we can provide (possibly in appendix, depending on what the paper looks like at that point), a worked out example as currently appears in the main text (which is helpful right now, since we're trying to sense a pattern).}


%\Anne{TODO: use package cleveref (ie. use \cref instead of \ref)} 
