
%%%%%%%%%%%%%%%%%%%%%%%%%%%%%%%%%%%%%%%%%%%%%%%%%%%%
%Section
\section{Quantum Indistinguishability Obfuscation}
\label{sec:QiO-Cliffords and more}
In this section, we provide definitions of equivalent quantum circuits and  Quantum Indistinguishability Obfuscation ($Qi\mathcal{O})$. In  \Cref{QiO:Clifford-Circuits}, we show how to construct $Qi\mathcal{O}$ for Clifford circuits. The first construction is based on a canonical form, and the second is based on the principle of gate teleportation.


%\anote{Need some blah blah at the beginning of a section to say what we present in this section.}

\subsection{Definitions}
\label{def:equivalent-circuits}
\begin{definition} {\rm (Equivalent Quantum Circuits):}
Let $C_{q_0}$ and $C_{q_1}$ be two $n$-qubit quantum circuits. We say $C_{q_0}$ and $C_{q_1}$ are \emph{equivalent} if for every $n$-qubit state $\ket{\psi}$
$$C_{q_1}(\ket{\psi}) =C_{q_2}(\ket{\psi}).$$
\end{definition}
%\anote{Need some blah blah before an important definition to describe what it is saying, but in words}

%\anote{why is computational in brackets below?}
\begin{definition}{\rm (Quantum Indistinguishability Obfuscation)}
\label{def:QiO}
A  polynomial-time quantum algorithm for a class of quantum circuits $\mathcal{C}_Q$ is a \emph{quantum computational indistinguishability obfuscator} ($Qi\mathcal{O}$)  if the following conditions hold:
%\Anne{as already noted, I am wondering if we should use a pure state $\ket{\phi}$ instead of a mixed state $\rho$, mostly because $ C_q^\prime(\ket{\phi},\ket{\psi})$ is funny notation. }

\begin{enumerate}
\item {\tt Functionality:}  For every $C_q\in \mathcal{C}_Q$ and every quantum state $\ket{\psi},$
$$(\ket{\phi}, C_q^\prime)\leftarrow Qi\mathcal{O}(C_q) \;  \mbox{ and }\; C_q^\prime(\ket{\phi},\ket{\psi})=C_{q}(\ket{\psi})$$
%where $\rho \in Q(m)$ and $C_q^\prime$  is a $m+n$ qubit quantum circuit.
											
%\anote{above, we are mixing pure states such as $\ket{\phi}$ with mixed states, such as $\rho$. This gets confusing, but I don't know exactly how to fix it. Maybe $\rho$ could be assumed wlog to be pure, so we can use use $\ket{\psi}$ instead of $\rho?$. }

\item  {\tt Polynomial Slowdown:}  For every $C_{q}\in \mathcal{C}_Q,$
\begin{itemize}
\item  $\ket{\phi}$ is at most a $poly(|C_{q}|)$ qubit state.
\item $|C_{q}^\prime| \in poly(|C_{q}|).$
\end{itemize}
%\anote{in below, did we define what we mean by size? If this is the number of gates, I would find the definition quite restrictive. Maybe you mean the same dimensions for input- output?}

\item {\tt Indistinguishability:} For any two equivalent quantum circuits $C_{q_1},C_{q_2}\in \mathcal{C}_Q,$ of the same size
 and for every polynomial-time quantum distinguisher $\mathcal{D}_q,$ there exists a negligible function {\rm negl} such that:
					$$\Big |{\rm Pr}[\mathcal{D}_q(Qi\mathcal{O}(C_{q_1}))=1]-{\rm Pr}[\mathcal{D}_q(Qi\mathcal{O}(C_{q_2}))=1] \Big |\leq  {\rm negl}(k).$$		
Where $k=|C_{q_1}|=|C_{q_2}|.$						
\end{enumerate}
\end{definition}


\subsection{Quantum Indistinguishability Obfuscation for Clifford Circuits}
\label{QiO:Clifford-Circuits}
 In this section we present two methods to construct $Qi\mathcal{O}$ for any Clifford circuit,  one using a canonical form and the other using gate teleportation.

\subsubsection{$Qi\mathcal{O}$ for Clifford Circuits via Canonical Form}
\label{sec:Clifford-iO-canonical}
%\Anne{what happened to all the subtlelties we had to fix in Aaronson-Gottesman?}
Aaronson and Gottesman invented a polynomial-time quantum algorithm (polynomial size circuit) that takes a Clifford circuit $C_q$ and outputs its canonical form (see~\cite{AG04}, section~VI). This canonical form is invariant for any two equivalent circuits. Moreover the size of the canonical form remains polynomial in the size of the input circuit.
\begin{algorithm}[]
\label{QiO:Canonical-Clifford}
   \caption{$Qi\mathcal{O}$-Canonical}
  \begin{itemize}
\item Input: A $n$-qubit Clifford Circuit $C_q.$
 \begin{enumerate}
  \item   Using  the Aaronson and Gottesman algorithm~\cite{AG04} compute the canonical form of $C_q$
   		\begin{equation*}
 		 C_q^\prime \xleftarrow{\mbox{canonical form}}C_q
 		 \end{equation*}
  \item Let $\ket{\phi}$ be an empty register.
  \item Output $\left(\ket{\phi},C_q^\prime \right)$.
  \end{enumerate}
  \end{itemize}
\end{algorithm}



 \begin{lemma}
\Cref{QiO:Canonical-Clifford} is a Quantum Indistinguishability Obfuscation for all Clifford Circuits.
\end{lemma}

\noindent {\bf Proof}: We have to show that \Cref{QiO:Canonical-Clifford} satisfies the definition of quantum indistinguishability obfuscation \ref{def:QiO}. Note that the construction is essentially the Aaronson and Gottesman  algorithm~\cite{AG04}. Therefore, it runs in polynomial-time.
Since,  $C_q^\prime$ is the canonical form of $C_q,$ they have the same functionality. Note $\ket{\phi}$ is an empty register and the size of $|C_q^\prime| \in poly(|C_q|)$ ~\cite{AG04}, therefore the construction is efficient (polynomial slowdown). Finally note the canonical form reveals no information about the input circuit $C_q$ excepts its functionality, therefore this $Qi\mathcal{O}$ is perfectly indistinguishable against any quantum adversary. Hence, computationally indistinguishable against any quantum adversary.



\subsubsection{$Qi\mathcal{O}$ for Clifford Circuits via Gate Teleportation}
\label{sec:Clifford-iO-teleportaion}
In this section, we  show how gate teleportation (\Cref{protocol: gate-teleportation}) can be used to construct a quantum indistinguishability obfuscation for Clifford circuits. Our algorithm (see \Cref{QiO:Clifford-teleportation}) relies on the assumption that there exists a quantum secure $i\mathcal{O}$ for classical circuits; this seems problematic at first, since there is no provably quantum secure $i\mathcal{O}$ known for general classical circuits.
%\anote{Really? I thought there were some candidates.... }
However,  our construction relies on the assumption that a quantum secure $i\mathcal{O}$ exists for a very specific class of boolean circuits.\footnote{Circuits that compute update functions for Clifford circuits, see \Cref{update function}}.  And in fact, it is easy to construct an $i\mathcal{O}$ for this class of circuits (see appendix \ref{sec: iO-clifford-functions}). %section \ref{sec: iO-clifford-functions}). %\anote{Indeed, we're looking at some classical circuits that are probably quite easy to deal with. I'm not sure they are Cliffords, however, since I think they are classical functions.}


%\anote{below, $\rho$ is usually a density matrix, which does not require a $\ket{\cdot}$.}
%\anote{below, say something about what ``lower half'' means, or establish earlier on that another way to write the tensor product of $n$ Bell states is $\sum_i \ket{i}{i}$ (or something like this; need to check). Then it is clear which register we are talking about for applying $C_q$.}

\begin{algorithm}[]
\label{QiO:Clifford-teleportation}
   \caption{$Qi\mathcal{O}$ from Gate Teleportation}
  \begin{itemize}
  \item Input: A $n$-qubit Clifford Circuit $C_q.$
  \begin{enumerate}
  \item Prepare $2n$ qubit Bell state $\ket{\beta^{ 2n}}=\ket{\beta_{00}}\otimes \cdots \otimes \ket{\beta_{00}}.$
  \item Apply the circuit $C_q$ on the right most $n$ qubits and obtain a system $\ket{\phi}$
  										 $$\ket{\phi}=(\igate_n\otimes C_q) \ket{\beta^{2n}}.$$
 \item Compute the classical circuit $C$ that computes the update function $F_{C_q}.$	
 \item Set $C^\prime\leftarrow i\mathcal{O}(C).$							
  \item Description of the circuit  $C_q^\prime:$
    \begin{enumerate}
     \item  Perform the Bell measurement on the leftmost $2n$-qubits on the system $\ket{\phi}\otimes \ket{\psi}),$ where $\ket{\phi}$ is an auxiliary state and $\ket{\psi}$ is an input state. Obtain classical bits $(a_1,b_1\ldots,a_n,b_n)$ and the state
     \begin{equation}
     \label{eq:last}
      C_q(\xgate^{\otimes_{i=1}^{n} b_{i}} \cdot \zgate^{\otimes_{i=1}^{n} a_{i}})\ket{\psi}.
      \end{equation}
     \item Compute the correction bits
      \begin{equation}
     \label{eq:last1}
  (a_1^\prime, b_1^\prime,\ldots, a_n^\prime, b_n^\prime)=C^\prime(a_1,b_1\ldots,a_n,b_n).
  \end{equation}
     \item Compute the correction unitary $(\xgate^{\otimes_{i=1}^{n} b_{i}^\prime} \cdot \zgate^{\otimes_{i=1}^{n} a_{i}^\prime}).$
     \item Apply  $(\xgate^{\otimes_{i=1}^{n} b_{i}^\prime} \cdot \zgate^{\otimes_{i=1}^{n} a_{i}^\prime})$ to the system $C_q(\xgate^{\otimes_{i=1}^{n} b_{i}} \cdot \zgate^{\otimes_{i=1}^{n} a_{i}})\ket{\psi}$ to obtain the state $C_q(\ket{\psi}).$
     \begin{equation*}
  \begin{aligned}
 (\xgate^{\otimes_{i=1}^{n} b_{i}^\prime} \cdot \zgate^{\otimes_{i=1}^{n} a_{i}^\prime}) \cdot C_q(\xgate^{\otimes_{i=1}^{n} b_{i}} \cdot \zgate^{\otimes_{i=1}^{n}a_{i}})\ket{\psi}\\
 =(\xgate^{\otimes_{i=1}^{n} b_{i}^\prime} \cdot \zgate^{\otimes_{i=1}^{n} a_{i}^\prime}) \cdot (\zgate^{\otimes_{i=1}^{n} a_{i}^\prime} \cdot \xgate^{\otimes_{i=1}^{n}b_{i}^\prime})C_q(\ket{\psi})\\
 = C_q(\ket{\psi}).
 \end{aligned}
 \end{equation*}
     \end{enumerate}
  \item Output $\left(\ket{\phi},C_q^\prime \right).$
  \end{enumerate}
  \end{itemize}
\end{algorithm}

%\anote{For step 2 below, shall we add to preliminaries that classical circuits can be embedded into quantum ones? I'm not sure, actually, how you envisage this classical part of the corrections- maybe I haven't fully understood yet.}
%On receiving $\left(\rho,C_q^\prime \right),$ one can evaluate the original circuit $C_q$ on any $n$-qubit state $\ket{\psi}$ as follows

%\begin{algorithm}[H]
%\label{Eval:Clifford-teleportation}
%\caption{Computing $C_q$ from $\left(\rho,C_q^\prime \right)$}
% \begin{itemize}
%  \item Input $\left(\rho,C_q^\prime \right)$ and a quantum state $\ket{\psi}.$
%  \begin{enumerate}
%  \item   Write $C_q^\prime=(B_q, C^\prime)$ and the joint system
%  \begin{equation*}
%  \label{eq1:GT:Clifford}
%  \rho_{AB}=\rho \otimes \ket{\psi}.
%  \end{equation*}
%  \item   Using circuit $B_q$ perform the Bell measurement on the leftmost $2n$-qubits on the system $\rho_{AB},$ and obtain classical bits $(a_1,b_1\ldots,a_n,b_n)$ and  a state $\rho_{AB}^\prime=B_q(\rho_{AB}).$
%  \item   Compute $C_q(\xgate^{\otimes_{i=1}^{n} b_{i}} \cdot \zgate^{\otimes_{i=1}^{n}a_{i}})\ket{\psi}=Tr_A(\rho_{AB}^\prime)$ (see  gate teleportation protocol \ref{protocol: gate-teleportation}).
%  \item   Compute the update function $$(a_1^\prime, b_1^\prime,\ldots, a_n^\prime, b_n^\prime)=C^\prime(a_1,b_1\ldots,a_n,b_n).$$
%  \item    Compute the correction unitary $(\xgate^{\otimes_{i=1}^{n} b_{i}^\prime} \cdot \zgate^{\otimes_{i=1}^{n} a_{i}^\prime})$
%  \item   Apply  $(\xgate^{\otimes_{i=1}^{n} b_{i}^\prime} \cdot \zgate^{\otimes_{i=1}^{n} a_{i}^\prime})$ to the system $C_q(\xgate^{\otimes_{i=1}^{n} b_{i}} \cdot \zgate^{\otimes_{i=1}^{n} a_{i}})\ket{\psi}$ to obtain the state $C_q(\ket{\psi})$ as follows,
%  \begin{equation*}
%  \begin{aligned}
% (\xgate^{\otimes_{i=1}^{n} b_{i}^\prime} \cdot \zgate^{\otimes_{i=1}^{n} a_{i}^\prime}) \cdot C_q(\xgate^{\otimes_{i=1}^{n} b_{i}} \cdot \zgate^{\otimes_{i=1}^{n}a_{i}})\ket{\psi}\\
% =(\xgate^{\otimes_{i=1}^{n} b_{i}^\prime} \cdot \zgate^{\otimes_{i=1}^{n} a_{i}^\prime}) \cdot (\zgate^{\otimes_{i=1}^{n} a_{i}^\prime} \cdot \xgate^{\otimes_{i=1}^{n}b_{i}^\prime})C_q(\ket{\psi})\\
% = C_q(\ket{\psi}).
% \end{aligned}
% \end{equation*}
%  \end{enumerate}
%  \end{itemize}
%\end{algorithm}

\begin{theorem}
If $i\mathcal{O}$ is a Quantum Secure Classical Indistinguishability Obfuscation, then construction  \ref{sec:Clifford-iO-teleportaion} is a Quantum Indistinguishability Obfuscation for all Clifford Circuits.
\end{theorem}

\noindent{\bf Proof}: We have to show that \cref{QiO:Clifford-teleportation} satisfies the definition of $Qi\mathcal{O}$ (\Cref{def:QiO}).
\begin{itemize}
\item  {\tt Functionality:} The functionality $Qi\mathcal{O}$ is followed from the gate teleportation (\Cref{protocol: gate-teleportation}).
\item {\tt Polynomial Slowdown:} Note $\ket{\phi}$ is a $O(n)$ qubit state, therefore $\ket{\phi}\in poly(|C_q|.$ and The cost  Bell measurement of an $O(n)$ qubit system cost is $O(n)$ gates and since $C^\prime\leftarrow i\mathcal{O}(C),$ the $i\mathcal{O}$ guarantees that $|C^\prime|\in poly(|C|).$ So all we have to show is that $|C|\in poly(|C_q|).$  Let $g_{|C_q|},\ldots, g_2,g_1$ be a topological ordering of the gates in $C_q.$ For each gate $g_i$ in $C_q$ the classical circuit $C$ has to implement one of the following operations (section \ref{update function}):
 \begin{itemize}
 \item[] $(a,b)\xrightarrow{\xgate, \zgate}(a,b).$
 \item[]  $(a,b)\xrightarrow{\hgate}(b,a)$         (one swap).
 \item[]  $(a,b)\xrightarrow{\pgate}(a,a\oplus b)$      (one $\oplus$ operation).
 \item[]  $(a_1,b_1,a_2,b_2)\xrightarrow{\mbox{\cnot}}(a_1\oplus a_2,b_1,a_2, b_1\oplus b_2)$ (two $\oplus$ operations).
 \end{itemize}
\begin{equation*}
\ket{\phi}\in poly(|C_q|)    \mbox{ and } |C_q^\prime |= \in poly(|C_q|).
\end{equation*}

\item  {\tt Indistinguishability:} Let $C_{q_1}$ and $C_{q_2}$ be two $n$-qubit equivalent Clifford circuits of the same size. We have to show that $Qi\mathcal{O}(C_{q_1})\sim^c Qi\mathcal{O}(C_{q_2})$ are computationally indistinguishable. Let
\begin{equation*}
\begin{aligned}
(\ket{\phi_1} ,C_{q_1}^\prime) \leftarrow Qi\mathcal{O}(C_{q_1}) \mbox{ and }
(\ket{\phi_2} ,C_{q_2}^\prime)(C_2))\leftarrow Qi\mathcal{O}(C_{q_2})
 \end{aligned}
\end{equation*}


Since $C_{q_1}(\ket{\tau})=C_{q_2}(\ket{\tau})$ for every quantum state $\ket{\tau}$ we have,

\begin{equation}
\label{eq:1}
\ket{\phi}_1=(I\otimes C_{q_1}) \ket{\beta^{2n}}=(I\otimes C_{q_2}) \ket{\beta^{2n}}=\ket{\phi}_2.
\end{equation}
Now all we need to show is that $i\mathcal{O}(C_1)\sim^ci\mathcal{O}(C_2)$ are computationally indistinguishable against any polynomial-time quantum adversary. Note if we can  prove the following two properties concerning $C_1$ and~$C_2$

\begin{enumerate}
\item[1)] $C_1$ and $C_2$ are equivalent circuits, \emph{i.e.}, they compute the same functionality
\item[2)]  $|C_1|=|C_2|,$
\end{enumerate}
then using the $i\mathcal{O}$ definition we can conclude that $i\mathcal{O}(C_1)\sim^ci\mathcal{O}(C_2).$ Note $C_1$ computes the update function $F_{C_{q_1}}$ and $C_2$ computes the update function $F_{C_{q_2}}.$ From Theorem \ref{sec:Clifford Functions} we have $F_{C_{q_1}}=F_{C_{q_2}},$ therefore, $C_1$ and $C_2$ must be equivalent circuits. Moreover if we use the $i\mathcal{O}$ defined in \Cref{sec: iO-clifford-functions}, then we are guaranteed to have $|C_1|=|C_2|.$ Therefore we have $i\mathcal{O}(C_1)\sim^c\mathcal{O}(C_2)$ and this completes the proof. But what if we are using any other quantum secure $i\mathcal{O},$ then it might be the classical circuit $C_1$ and $C_2$ are of different sizes, even if  $|C_{q_1}|=|C_{q_2}|.$  But this is not a problem, suppose we want to obfuscate a quantum circuit $C_q$. The classical circuit $C$ is constructed by going through each gate in $C_{q}.$ Some gates are more costly than others (for \emph{e.g.}, $\cnot$ vs. $\zgate,$ see above or section \ref{update function}). Note that we can obtain an upper bound on all the classical circuits by replacing each gate in $C_q$ with the most costly gate and then computing the classical circuit for the resulting quantum gate. Now suppose $m$ is the upper bound on the size of classical circuits, then for any circuit $C_q,$ we first calculate the circuit $C$ that computes $F_{C_q}$ and then pads $C$ with $m-|C|$ identity gates. This will ensure that if $|C_{q_1}|=|C_{q_2}|,$ then $|C_1|=|C_2|.$
\end{itemize}


\begin{theorem}\label{sec:Clifford Functions}
Let $C_{q_1}$ and $C_{q_2}$ be two equivalent $n$-qubit Clifford circuits and  $F_{C_{q_1}}$ and $F_{C_{q_2}}$  denote the update functions for the circuits $C_{q_1}$ and $C_{q_2},$ then $F_{C_{q_1}}({\bf s})=F_{C_{q_2}}({\bf s})$ for every ${\bf s}\in\{0,1\}^{2n}.$
\end{theorem}

\begin{flushleft}
{\bf Proof:} Let $F_{C_{q_1}}$ and $F_{C_{q_2}}$ denote the update functions for the circuits $C_{q_1}$ and $C_{q_2}.$ Suppose $F_{C_{q_1}}\neq F_{C_{q_2}},$  then there must exist at least one binary string $a_1b_1,\ldots, a_nb_n$ such that
\begin{equation}
\label{fun:assumption1}
F_{C_{q_1}}(a_1,b_1,\ldots,a_n,b_n)\neq F_{C_{q_2}}(a_1,b_1,\ldots,a_n,b_n)
\end{equation}
Let $C_1$ and $C_2$ be circuits that compute $F_{C_{q_1}}$ and $F_{C_{q_2}}$ respectively, then
\begin{equation}
\label{ineq:1}
C_1(a_1,b_1,\ldots,a_n,b_n)\neq C_2(a_1,b_1,\ldots,a_n,b_n)
\end{equation}


%\begin{equation*}
%\left(\rho ,\left(B_q, i\mathcal{O}(C_1})\right)\right)= Qi\mathcal{O}(C_2}) \mbox{ and } \left(\rho ,\left(B_q, i\mathcal{O}(C_{F_{C_{q_2}}})\right)\right)= Qi\mathcal{O}(C_{q_2}).
%\end{equation*}

Suppose we evaluated circuits $C_{q_1}$ and $C_{q_2}$ on some state $\ket \psi$ using construction based on gate teleportation (\Cref{sec:Clifford-iO-teleportaion}). Then we are left with systems  (see \Cref{eq:last})%$\rho_{AB}=\rho\otimes \ket \psi$ (see algorithm 3). Therefore, the classical output bits we obtain after performing the Bell measurement in algorithm 3 are independent of the circuits $C_{q_1}$ and $C_{q_2}$). Suppose for both circuits we obtain the same classical outputs $a_1b_1,\ldots, a_nb_n.$ From algorithm 3 we have

 \begin{equation}
  \label{eq:2}
C_{q_1} (\xgate^{\otimes_{i=1}^{n} b_{i}^\prime} \cdot \zgate^{\otimes_{i=1}^{n}a_{i}^\prime})\ket{\psi} \;\mbox{ and }\;
C_{q_2} (\xgate^{\otimes_{i=1}^{n} \beta_{i}^\prime} \cdot \zgate^{\otimes_{i=1}^{n}\alpha_{i}^\prime})\ket{\psi}
\end{equation}
where $a_1^\prime b_1^\prime,\ldots, a_n^\prime b_n^\prime$ and $\alpha_1^\prime \beta_1^\prime,\ldots, \alpha_n^\prime \beta_n^\prime$ are random bits obtained after Bell measurement (\Cref{eq:last1}).   And further assume that the classical output of the Bell measurement for both circuits is the string  $s=a_1,b_1,\ldots,a_n,b_n,$ i.e  $s=a_1b_1,\ldots, a_nb_n$ and  $C_1(s)\neq C_2(s)$ (\ref{ineq:1}). Let
 \begin{equation}
 \label{eq:3}
C_1(s)=(a^\prime_1,b^\prime_1,\ldots, a^\prime_n,b^\prime_n)\neq(d_1^\prime,e_1^\prime,\ldots, d_n^\prime,e^\prime_n)=C_2(s)
\end{equation}



Since, $C_{q_1}$ and $C_{q_2}$ are equivalent circuits,  the relationships in \Cref{eq:2} are also equivalent, \emph{i.e.},
 \begin{equation}
  \label{eq:4}
C_{q_1} (\xgate^{\otimes_{i=1}^{n} b_{i}} \cdot \zgate^{\otimes_{i=1}^{n}a_{i}})\ket{\psi}=C_{q_2} (\xgate^{\otimes_{i=1}^{n} b_{i}} \cdot \zgate^{\otimes_{i=1}^{n}a_{i}})\ket{\psi}
\end{equation}

We can rewrite \Cref{eq:4} by using update functions from \Cref{eq:3}
 \begin{equation}
  \label{eq:5:last}
(\zgate^{\otimes_{i=1}^{n} a_{i}^\prime} \cdot \xgate^{\otimes_{i=1}^{n} b_{i}^\prime})C_{q_1}(\ket{\psi})=(\zgate^{\otimes_{i=1}^{n} d_{i}^\prime} \cdot \xgate^{\otimes_{i=1}^{n} e_{i}^\prime})C_{q_2}(\ket{\psi})
\end{equation}

Since $C_{q_1}(\ket{\psi})=C_{q_2}(\ket{\psi})$ we can substitute $C_{q_1}$ for $C_{q_2}$ in \Cref{eq:5:last}

\begin{equation}
  \label{eq:6:last}
(\zgate^{\otimes_{i=1}^{n} a_{i}^\prime} \cdot \xgate^{\otimes_{i=1}^{n} b_{i}^\prime})C_{q_1}(\ket{\psi})=(\zgate^{\otimes_{i=1}^{n} d_{i}^\prime} \cdot \xgate^{\otimes_{i=1}^{n} e_{i}^\prime})C_{q_1}(\ket{\psi})
\end{equation}

It follows from \Cref{eq:6:last} that
\begin{equation}
  \label{eq:7:last}
(\zgate^{\otimes_{i=1}^{n} a_{i}^\prime} \cdot \xgate^{\otimes_{i=1}^{n} b_{i}^\prime})=(\zgate^{\otimes_{i=1}^{n} d_{i}^\prime} \cdot \xgate^{\otimes_{i=1}^{n} e_{i}^\prime})
\end{equation}

It follows from equation \ref{eq:7:last}  that $$\zgate^{a_i^\prime} \cdot \xgate^{b_i^\prime}=\zgate^{d_i^\prime} \cdot \xgate^{e_i^\prime} \mbox{ for all } i\in[n].$$ Which is only possible if $a_i^\prime=d_i^\prime$ and  $b_i^\prime=e_i^\prime$ and we have a contradiction, therefore, if $C_{q_1}$ and $C_{q_2},$ are two equivalent circuits, then $F_{C_{q_1}}=F_{C_{q_2}}.$
\end{flushleft}





%\subsection{Indistinguishability Obfuscation for the Clifford Update Functions}
%\label{sec: iO-clifford-functions}
%
%
%\anote{I am not sure how am I to understand that the $F_{C_q}$ are Cliffords.... I thought they were classical circuits?}
%
%In this section we present a (quantum secure classical) $i\mathcal{O}$ that can obfuscate the update function $F_{C_q},$ for any $n$-qubit Clifford circuit $C_q.$
%
%\begin{algorithm}[H]
%   \caption{$i\mathcal{O}$ for Clifford update Functions $F_{\tt Clifford}$}
%  \begin{itemize}
%  \item Input: A $n$-qubit Clifford Circuit $C_q,$ a security parameter.
%  \begin{enumerate}
%  \item Compute $C_{q_c}\leftarrow$ {\tt AG-Canonical-Unique}$(C_q).$
%  \item Compute the circuit $C_{F_{q_c}}$ for the update function $F_{C_{q_c}}.$
%  \item Output $C_{F_{q_c}}.$
%  \end{enumerate}
%  \end{itemize}
%\end{algorithm}
