
%%%%%%%%%%%%%%%%%%%%%%%%%%%%%%%%%%%%%%%%%%%%%%%5
%Section: Obfuscating Beyond Clifford Circuits
%%%%%%%%%%%%%%%%%%%%%%%%%%%%%%%%%%%%%%%%%%%%%
\section{Obfuscating Beyond Clifford Circuits}
\label{QiO:Clifford+T:family}
In the previous section, we showed how we can construct a $Qi\mathcal{O}$ for Clifford circuits from gate-teleportation and $i\mathcal{O}.$ In this section, we extend the gate teleportation technique to show how we can construct $Qi\mathcal{O}$ for any quantum circuit that has at most logarithmic non-Clifford-gates in the circuit size.  We start by constructing a $Qi\mathcal{O}$ for any 1-qubit quantum circuit (\cref{QiO:Clifford+T:family+GT}), then in \Cref{sec:n-qubit-circuits}  we extend the 1-qubit construction to any $n$-qubit quantum circuit that has at most a logarithmic number of  $\tgate$-gates in the circuit size.\footnote{Recall from  \Cref{defn:Clifford+T:family} that adding a non-Clifford gate  such as $\tgate$-gate to Clifford gates gives us a generating set for all unitary quantum circuits.}



\subsection{$Qi\mathcal{O}$ for 1-Qubit Quantum Circuits}
\label{QiO:Clifford+T:family+GT}
In this section we show how we can obfuscate any 1-qubit unitary circuit using gate teleportation and a quantum secure $i\mathcal{O}.$  The $\tgate$ relates to the $\xgate, \zgate$ as
\begin{equation}
\label{eq:pgate}
\tgate \xgate^b \zgate^a=\xgate^b \zgate^{a\oplus b}\pgate^b \tgate.
\end{equation}
If $b=0,$ then we don't have to worry about $\pgate$ gates; otherwise we have to perform a $\pgate$ correction. This is undesirable as $\pgate$ does not commute with $\xgate,$ making the update function complicated. Note we can write $\pgate=(\frac{1+i}{2}) \igate + (\frac{1-i}{2})\zgate,$ therefore  \Cref{eq:pgate} can be rewritten as,

\begin{equation}
\label{eq:pgate1}
\tgate \xgate^b \zgate^a=\xgate^b \zgate^{a\oplus b}\left[\left(\frac{1+i}{2}\right) \igate + \left(\frac{1-i}{2}\right)\zgate\right]^b \tgate
\end{equation}
Note  $\left[\left(\frac{1+i}{2}\right) \igate + \left(\frac{1-i}{2}\right)\zgate\right]^b=\left(\frac{1+i}{2}\right) \igate + \left(\frac{1-i}{2}\right)\zgate^b$ for $b \in \{0,1\},$ therefore we can rewrite  \Cref{eq:pgate1} as,
\begin{equation}
\label{eq:pgate2}
\begin{aligned}
\tgate \xgate^b \zgate^a=\xgate^b \zgate^{a\oplus b}\left[\left(\frac{1+i}{2}\right) \igate + \left(\frac{1-i}{2}\right)\zgate^b\right] \tgate \\
=\left[\left(\frac{1+i}{2}\right) \xgate^b \zgate^{a\oplus b} + \left(\frac{1-i}{2}\right)\xgate^b \zgate^a\right] \tgate. \\
\end{aligned}
\end{equation}

Since $a,b\in\{0,1\}$ there are only four possible combinations. This mean that we can be express \Cref{eq:pgate2} as a linear combination of $\{\igate, \xgate, \zgate, \xgate \zgate\},$

\begin{equation}
\label{eq:pgate3}
\begin{aligned}
\tgate \xgate^b \zgate^a= (\alpha_0 \igate +  \alpha_1 \xgate + \alpha_2 \zgate + \alpha_3 \xgate\zgate)\tgate \\
\end{aligned}
\end{equation}
where $\alpha_j \in\left\{0,1, \frac{1+i}{2}, \frac{1-i}{2}\right\},$ for  $j\in[4]$. \Cref{eq:pgate3} is important as we will see that the correction unitary for any 1-qubit circuit can be expressed in this form.

\subsection{Single-Qubit Circuits}
\label{sec:1-qubit}

Let $C_q$ be a $1$-qubit circuit we want to obfuscate and $\ket{\psi}$ be the quantum state on which we want to evaluate $C_q.$ Note we can write any $1$-qubit circuit as a sequence of gates from the set $\{\hgate, \tgate\}$ \footnote{The set $\{\hgate,  \tgate\}$ is universal for 1-qubit unitary \cite{KLM07}.}

$$C_q=(g_{|C_q|},\ldots,g_2,g_1 ),\; g_i\in\{\hgate, \tgate\}$$
Recall that in the gate-teleportation construction for Clifford circuits, we are left with a subsystem of the form $C_q \xgate^b\zgate^a (\ket{\psi})$ (\Cref{Eval:Clifford-teleportation}, algorithm 4, step 3)

\begin{equation}
\label{eq:pgate4}
C_q \xgate^b\zgate^a (\ket{\psi})=(g_{|C_q|},\ldots,g_2,g_1 )\xgate^b\zgate^a (\ket{\psi}),
\end{equation}
and to evaluate the circuit on $\ket{\psi}$, we have to apply a correction unitary. We can write the system in \Cref{eq:pgate4} as

\begin{equation}
\begin{aligned}
\label{eq:pgate5}
C_q \xgate^b\zgate^a (\ket{\psi})=(g_{|C_q|},\ldots,g_2)(\alpha_0 \igate +  \alpha_1 \xgate + \alpha_2 \zgate + \alpha_3 \xgate \zgate)g_1 (\ket{\psi})
\end{aligned}
\end{equation}
 where $\alpha_i \in\left\{0,1, \frac{1+i}{2}, \frac{1-i}{2}\right\}.$ We can write  \Cref{eq:pgate5}  as a conditional equation based on if $g_2=\tgate$ or $g_2=\hgate.$

 \begin{equation}
  C_q \xgate^b\zgate^a (\ket{\psi})=
\begin{cases}
  (g_{|C_q|},\ldots,g_3)(\alpha_0 \igate +  \alpha_2 \xgate + \alpha_1 \zgate + \alpha_3 \xgate \zgate)g_2g_1 (\ket{\psi}), & \text{if } g_2=\tgate\\
 (g_{|C_q|},\ldots,g_3)(\alpha_0 \igate +  \alpha_1 \zgate + \alpha_2^\prime \xgate + \alpha_3^\prime \xgate \zgate)g_2g_1 (\ket{\psi}), & \text{if } g_2=\hgate\\
\end{cases}
\end{equation}
where $\alpha_2^\prime=\alpha_2(1-i)/2+\alpha_3(1+i)/2$ and $\alpha_3^\prime=\alpha_3(1-i)/2+\alpha_2(1+i)/2$ for $i\in[4].$ Note that in the expression above, the coefficients may change if $g_2$ happens to be a $\tgate$-gate;  whereas if $g_2$ is an $\hgate$-gate then the size of the coefficients remains unaffected. Moreover, applying any number of gates to a system as in  \Cref{eq:pgate3} cannot increase the number of terms, \emph{i.e.}, the number of terms in the summation remains at most~4. However, the size of coefficients $\alpha_i$ can be changed with each application of a $\tgate$-gate.

Hence, it follows that by applying the remaining  sequence of gates $(g_{|C_q|},\ldots,g_3)$, the \Cref{eq:pgate5} can be written as
\begin{equation}
\begin{aligned}
\label{eq:pgate8}
C_q \xgate^b\zgate^a (\ket{\psi})=(\beta_0 \igate +  \beta_1 \xgate + \beta_2 \zgate + \beta_3\xgate \zgate)(g_{|C_q|},\ldots,g_2,g_1 )(\ket{\psi})
\end{aligned}
\end{equation}
where $\beta_i  \in \mathbb{C},$ $i\in[4].$  Therefore, for any  $1$-qubit circuit $C_q,$ the correction unitary can be written as

\Anne{above is a lot of fluff to say that the Paulis form a basis!}
\begin{equation}
\label{eq:pgate9}
 (\beta_0 \igate +   \beta_1 \xgate +  \beta_2 \zgate +  \beta_3 \xgate\zgate), \mbox{ for }  \beta_i \in \mathbb{C},\; i\in[n]
\end{equation}

Moreover, the size of the coefficients $\beta_i$'s grows at most a polynomial in the number of $\tgate$-gates. This is discussed in (\Cref{coeff:size}). Therefore the update function for any $1$-qubit circuit $C_q$ can be defined as a following map,

\begin{equation*}
F_{C_q}:\{0,1\}^2\rightarrow \mathbb{C}^4,\; (a,b)\mapsto (\beta_0, \beta_1,\beta_2,\beta_3),
\end{equation*}
which corresponds to the correction unitary $(\beta_0 \igate +  \beta_1 \xgate + \beta_2 \zgate + \beta_3\xgate \zgate).$
Moreover, the update functions for any two equivalent 1-qubit circuit are equivalent (see \Cref{lemma:1qubit}). Therefore, we can obfuscate any 1-qubit circuit using gate teleportation and a quantum secure $i\mathcal{O}$ as we did for the Clifford circuits (\Cref{sec:Clifford-iO-teleportaion}). The proof is nearly identical to the gate teleportation based Clifford construction (\Cref{sec:Clifford-iO-teleportaion}).
\Anne{Need to introduce next Lemma. Is the above sentence doing this? If so, be more direct.}


\begin{lemma}
\label{lemma:1qubit}
Let $C_{q_1}$ and $C_{q_2}$ be two equivalent 1-qubit circuits and suppose
\begin{equation*}
F_{C_q}:\{0,1\}^2\rightarrow \mathbb{C}^4 \mbox{ and } F_{C_q^\prime}:\{0,1\}^2\rightarrow \mathbb{C}^4
\end{equation*}
are the corresponding update functions for $C_{q_1}$  and $C_{q_2},$   then $F_{C_q}=F_{C_q^\prime}.$
\end{lemma}

{\bf Proof:}  Suppose $C_{q_1}$ and $C_{q_2}$ be two equivalent 1-qubit circuits and let $F_{C_{q_1}}$ and $F_{C_{q_2}}$ be the corresponding update functions. Suppose $F_{C_{q_1}}\neq F_{C_{q_2}}$ Then there must exist at least one input $(a,b)$ such that
\begin{equation}
\label{fun:assumption1}
F_{C_{q_1}}(a,b)\neq F_{C_{q_2}}(a,b)
\end{equation}
Let $C_1$ and $C_2$ be circuits that compute $F_{C_{q_1}}$ and $F_{C_{q_2}},$ then it follows from the relation \ref{fun:assumption1} that
\begin{equation*}
C_1(a,b)=(\beta_0, \beta_1, \beta_2 , \beta_3)\neq (\beta_0^\prime, \beta_1^\prime, \beta_2^\prime , \beta_3^\prime)=C_2(a,b)
\end{equation*}

Suppose we want to evaluate circuits $C_{q_1}$ and $C_{q_2}$ on some state $\ket \psi$ using gate teleportaton (\Cref{protocol: gate-teleportation}). And further assume that the classical output of the Bell measurement for both circuits is  $(a,b)$ (\Cref{protocol: gate-teleportation}, step~4). After step~5 of  gate teleportation protocol (\Cref{protocol: gate-teleportation}) we are left with  \Cref{lemma:1qubit:eq1} and \Cref{lemma:1qubit:eq2}
\begin{equation}
\label{lemma:1qubit:eq1}
\begin{aligned}
 C_{q_1} \xgate^b \zgate^a(\ket{\psi})=(\beta_0 \igate+\beta_1 \xgate+\beta_2 \zgate+\beta_3 \xgate \zgate)  C_{q_1} (\ket{\psi}).\\
 \end{aligned}
\end{equation}

\begin{equation}
\label{lemma:1qubit:eq2}
\begin{aligned}
 C_{q_2}\xgate^b \zgate^a(\ket{\psi})=(\beta_0^\prime \igate+\beta_1^\prime \xgate+\beta_2^\prime \zgate+\beta_3^\prime \xgate \zgate)  C_{q_2}(\ket{\psi}).
 \end{aligned}
\end{equation}

Since, $C_{q_1}$ and $C_{q_2}$ are equivalent circuits we must have $C_{q_1} \xgate^b \zgate^a(\ket{\psi})=C_{q_2} \xgate^b \zgate^a(\ket{\psi}),$ therefore

\begin{equation}
\label{lemma:1qubit:eq3}
\begin{aligned}
(\beta_0 \igate+\beta_1 \xgate+\beta_2 \zgate+\beta_3 \xgate \zgate)  C_{q_1} (\ket{\psi})=(\beta_0^\prime \igate+\beta_1^\prime \xgate+\beta_2^\prime \zgate+\beta_3^\prime \xgate \zgate)  C_{q_2} (\ket{\psi})\\
=(\beta_0^\prime \igate+\beta_1^\prime \xgate+\beta_2^\prime \zgate+\beta_3^\prime \xgate \zgate)  C_{q_1}(\ket{\psi}).\\
\end{aligned}
\end{equation}


From \Cref{lemma:1qubit:eq3} we must have,
\begin{equation}
\label{lemma:1qubit:eq4}
\begin{aligned}
(\beta_0 \igate+\beta_1 \xgate+\beta_2 \zgate+\beta_3 \xgate \zgate) =(\beta_0^\prime \igate+\beta_1^\prime \xgate+\beta_2^\prime \zgate+\beta_3^\prime \xgate \zgate).\\
\end{aligned}
\end{equation}

From equation \Cref{lemma:1qubit:eq4}
\begin{equation}
\label{lemma:1qubit:eq5}
\begin{aligned}
(\beta_0-\beta_0^\prime) \igate+(\beta_1 -\beta_1^\prime)\xgate+(\beta_2-\beta_2^\prime) \zgate+(\beta_3-\beta^\prime_3) \xgate \zgate={\bf 0}.
\end{aligned}
\end{equation}

The equation \Cref{lemma:1qubit:eq5},  can only be satisfied if $\beta_i = \beta_i^\prime$ for every $i\in[4],$ therefore we have a contradiction and $F_{C_{q_1}}=F_{C_{q_2}}.$



\subsection{$Qi\mathcal{O}$ via Gate Teleportation for all Quantum Circuits with Low $\tgate$-gate Complexity }
\label{sec:n-qubit-circuits}
In this section, we construct a $Qi\mathcal{O}$ for all quantum circuits where the number of $\tgate$-gates is at most logarithmic in the circuit size. The construction is very similar to the gate teleportation for Clifford circuits (\Cref{sec:Clifford-iO-teleportaion}). The reason for this bound is that the update function blows up once the number of $\tgate$-gates is greater than logarithmic in the circuit size. More, precisely the update function for any $n$-qubit quantum circuit $C_q$ with $k$ $\tgate$-gates can be defined as the following map. \footnote{It is discussed below why this map is defined conditionally on $k\leq n$ and $k>n$.}

\Anne{lots of notation going on here... not sure about $\bf s$, among other things. }

\begin{equation}
\label{nqubit:eq0}
\begin{aligned}
 F_{C_q}:
\begin{cases}
    \{0,1\}^{2n}\longrightarrow  (\mathbb{C} \times {\bf B})^{4^k},& \text{if } k\leq n\\
     \{0,1\}^{2n}\longrightarrow  (\mathbb{C} \times {\bf B})^{4^n}, & \text{if } k> n\\
\end{cases}
\end{aligned}
\end{equation}
\begin{equation}
\begin{aligned}
 (a_1,b_1,\ldots, a_n,b_n) \mapsto \left((\beta_1, {\bf s}_1),\dots, (\beta_{4^k}, {\bf s}_{4^k}))\right).
\end{aligned}
\end{equation}
where  ${\bf B}=\{0,1\}^{2n}.$

Therefore, we must have $k\in O(\log(|C_q|)).$ Note that the above map corresponds to the correction unitary \Cref{exp:nqubit-correction}



 \begin{equation}
\label{exp:nqubit-correction}
\sum_{i=1}^{4^k} \beta_i \xgate^{b_{i_1}} \zgate^{a_{i_1}}\otimes \cdots \otimes \xgate^{b{i_n}} \zgate^{a_{i_n}},\;  \; k\leq n.
\end{equation}
where $\beta_i\in\mathbb{C}$ and ${\bf s}_i=a_{i_1}b_{i_1}, \ldots, a_{i_n}b_{i_n} \in\{0,1\}^{2n}.$  Note that there are at most $2^{2n}$ bit strings of size $2n,$ therefore it follows from \Cref{exp:nqubit-correction} that for any $n$-qubit circuit the correction unitary can be written as a summation of at most $2^{2n}$ terms. Note that the range of the update function increases exponentially in the number of $\tgate$-gates for $k\leq n$ but if $k>n$ then the range of update function becomes fixed and no longer depends on $k$ (see \Cref{nqubit:eq0}). However, it is important to realize that the sizes of coefficients $\beta_i$'s (\Cref{nqubit:eq0}) can be increased with each application of a $\tgate$-gate.

\subsection{Worst-Case Correction Unitary}
Let $C_q$ be an $n$-qubit circuit and $\ket{\psi}$ be an $n$-qubit state. If we want to evaluate $C_q$ on the state $\ket{\psi}$ using gate teleportation, then we are required to compute a correction unitary $U^\prime$ corresponding to $C_q$ (\Cref{protocol: gate-teleportation}) satisfying \Cref{nqubit:eq6}. We will show that in the worst-case the correction unitary is of the form \Cref{exp:nqubit-correction}. \Anne{not sure what is the purpose of the Equation below.}
\begin{equation}
\label{nqubit:eq6}
C_qU(\ket{\psi})= U^\prime C_q (\xgate^{a_{1}}\zgate^{b_{1}}\otimes \xgate^{a_{2}}\zgate^{b_{2}}\otimes \cdots \otimes \xgate^{a_{n}}\zgate^{b_{n}})(\ket{\psi}).
\end{equation}

Let us construct $U^\prime$ and estimate how big it can grow. Suppose there are $k\leq n$ number of $\tgate$ gates in $C_q$ for some non-negative integer $k\leq n.$ To simplify the analysis we assume that $\tgate$-gates are in the first $l$ wires of $C_q,$ and all of $\cnot$ gates are acted after all the $\tgate$-gates \footnote{This restriction will not not affect our analysis see remark \cref{remark:nqubit-correction1}.} Note we must have $l\leq k.$ Then for every $j$-th wire with $\tgate$-gate(s)

\begin{equation}
\begin{aligned}
\label{nqubit:eq7}
\tgate(\xgate^{a_{j}}\zgate^{b_{j}})\mapsto \left(\lambda_{j_0} \igate + \lambda_{j_1} \xgate + \lambda_{j_2} \zgate + \lambda_{j_3} \xgate \zgate \right)\tgate=\left(\sum_{i=1}^4 \lambda_{j_i} \xgate^{a_{i}^\prime}\zgate^{b_{i}^\prime}\right)\tgate\\
 \end{aligned}
\end{equation}
where $\lambda_{j_i}\in\mathbb{C}, $  $a_{i}^\prime, b_{i}^\prime \in\{0,1\}$ for $i\in[4]$ and $j\in[l].$ Then we can rewrite relation \Cref{nqubit:eq6} as


\begin{equation}
\label{nqubit:eq8}
\begin{aligned}
 C_q (\xgate^{a_{i}}\zgate^{b_{i}})^{\otimes_{i=1}^{n}}\ket{\psi}= \left(\sum_{i=1}^4 \lambda_{j_i} \xgate^{a_{i}^\prime}\zgate^{b_{i}^\prime}\right)^{\otimes_{i=1}^{l}} \otimes \left(\xgate^{a_{i}^\prime}\zgate^{b_{i}^\prime}\right)^{\otimes_{i=l+1}^{n}}\\
\end{aligned}
\end{equation}
Due to $\cnot$ gates we have to expand the relation \Cref{nqubit:eq8}  using the identity $(A+B)\otimes C=A\otimes C + B\otimes C.$


\begin{equation}
\label{nqubit:eq9}
\begin{aligned}
C_q (\xgate^{a_{i}}\zgate^{b_{i}})^{\otimes_{i=1}^{n}}=\left(\sum_{j=1}^{4^l} \beta_j \xgate^{b_{j1}}\zgate^{a_{j1}} \otimes \cdots \otimes\xgate^{b_{jl}}\zgate^{a_{jl}}\right) \otimes \left(\xgate^{a_{i}^\prime}\zgate^{b_{i}^\prime}\right)^{\otimes_{i=l+1}^{n}}, \\
 =\left(\sum_{j=1}^{4^{l}} \beta_j \xgate^{b_{j1}^\prime}\zgate^{a_{j1}^\prime} \otimes \cdots \otimes\xgate^{b_{jl}^\prime}\zgate^{a_{jl}^\prime} \otimes \xgate^{a_{l+1}^\prime}\zgate^{b_{l+1}^\prime}  \otimes \cdots   \otimes\xgate^{a_{n}^\prime}\zgate^{b_{n}^\prime}\right)
 \end{aligned}
\end{equation}
where $\beta_j\in \mathbb{C}$ for $j\in[4^l].$  Recall that $k$ denotes the number of $\tgate$-gates and $l\leq k\leq n,$ therefore in the worst case the correction unitary can have $4^k$ terms.



Therefore, we define an update function for any $n$-quantum circuit $C_q$ with $k\leq n\leq |C_q|$ number of $\tgate$-gates as a map.
\begin{equation}
\label{nqubit:eq10}
\begin{aligned}
 F_{C_q}:\{0,1\}^{2n}\longrightarrow  (\mathbb{C} \times {\bf B})^{4^k}, \mbox{ where } {\bf B}=\{0,1\}^{2n}\\
  (a_1,b_1,\ldots, a_n,b_n) \mapsto \left((\beta_1, {\bf s}_1),\dots, (\beta_{4^k}, {\bf s}_{4^k})\right).
\end{aligned}
\end{equation}
Which corresponds to the correction unitary defined in \Cref{exp:nqubit-correction}.
\subsection{Remark}
\label{remark:nqubit-correction1}
Note the restriction that $\tgate$-gates are in the first $l$ wires of the circuit has no affect on our analysis. This is because, any permutation of the wires in the circuit will give us $\ell$ terms of form $\sum_{i=1}^4 \lambda_{j_i} \xgate^{a_{j_i}^\prime} \zgate^{b_{j_i}^\prime}$ and $n-l$ terms of the form $\xgate^{a_{i}^\prime} \zgate^{b_{i}^\prime}.$  And the tensor product of these terms in any order will result into a unitary in the worst-case of form. Moreover, $\cnot$ does not increase the number of terms in the update function or affect the size of the coefficients. So our worst-case analysis is not affected by this restriction.


\subsection{$Qi\mathcal{O}$ for Unitary Quantum Circuits with Low $\tgate$ gate Complexity}
The $Qi\mathcal{O}$ construction and proof of its security is identical to the  construction for Clifford gates (section \Cref{sec:Clifford-iO-teleportaion}). We only need to prove the following two properties concerning the update functions.
\begin{enumerate}
\item For any two equivalent quantum circuits $C_{q_1}$ and $C_{q_2}$ the corresponding update functions are equal $F_{C_{q_1}}=F_{C_{q_2}}.$
\item The update function can be computed by classical circuit of size $poly(|C_q|).$
\end{enumerate}

%
%\begin{equation}
%\label{nqubit-correction-final}
%\sum_{i=1}^{4^k} \beta_i \xgate^{b{i_1}} \zgate^{a{i_1}}\otimes \cdots \otimes \xgate^{b{i_n}} \zgate^{a{i_n}},\; \beta_i\in\mathbb{C},\; a_{i_j},b_{i_j}\in\{0,1\}.
%\end{equation}
%Therefore, if $k\in O(\log(|C_q|)$ (where $k$ is the $\tgate$-count), then the number of terms in the correction unitary \Cref{nqubit-correction-final} are in $O(|C_q|).$



%\begin{remark}
%\label{remark:nqubit-correction2}
%Note for any $n$-qubit circuit the correction unitary (relation \Cref{nqubit-correction-final} can have at most $2^{2n}$ terms, regardless how many $\tgate$-gates a circuit has. To realize this suppose we have
%$$C_q (\xgate^{a_{i}}\zgate^{b_{i}})^{\otimes_{i=1}^{n}}=\left(\sum_{j=0}^{l} \beta_j \xgate^{b_{j1}}\zgate^{a_{j1}} \otimes \cdots \otimes\xgate^{b_{jl}}\zgate^{a_{jl}} \otimes \xgate^{a_{l+1}^\prime}\zgate^{b_{l+1}^\prime}  \otimes \cdots   \otimes\xgate^{a_{n}^\prime}\zgate^{b_{n}^\prime}\right)$$
%for some integer $l>2^{2n}.$
%But since, there are at most $2^{2n}$ bit strings of size $2n,$ by pigeon hole principle we must have at least $2n-l$ terms that have common binary strings and by adding these common terms we are guaranteed to have every unitary that has at most $2^{2n}$ terms. This means that if the number of terms in the correction unitary grows exponentially (in the worst-case) in the number of $\tgate$-gates $k$ as far as $k\geq n.$ Note the size of the coefficients $\beta_i$ is a function of $\tgate$-gates and will be affected as number of $\tgate$-gates increases. So if there are exponentially many $\tgate$-gates then the size of the $\beta_i$ can also also grow exponentially.
%\end{remark}

%
%\subsection{Update Function for an arbitrary $n$-Qubit circuit}
%\label{nqubit:update-function}
%In the previous section saw that the general form of a correction unitary for an arbitrary quantum circuit is $$\sum_{i=1}^{4^k} \beta_i \xgate^{b{i_1}} \zgate^{a{i_1}}\otimes \cdots \otimes \xgate^{b{i_n}} \zgate^{a{i_n}}.$$ But in order to construct $Qi\mathcal{O}$ we have to define precisely define update functions. The update function must satisfy the following two properties.
%
%\begin{enumerate}
%\item ({\bf Invariant}) Any two equivalent quantum circuits should have the update function.
%\item ({\bf Efficiency}) The update function can be computed by a classical circuit of polynomial size.
%\end{enumerate}
%
%Let $C_q$ be an $n$-qubit circuit and $\mathbb{C}^*= \mathbb{C}-\{0\}$ (i.e. set of all complex numbers excluding 0). The update function $F_{C_q}$ for the circuit is define as
%
%\begin{equation}
%\label{nqubit:eq10}
%\begin{aligned}
%F_{C_q}: \{0,1\}^{2n}\longrightarrow {\mathbb{C}^*}^m\times \{0,1,\ldots,2^{2n}-1\}^m, \\
%\end{aligned}
%\end{equation}
%$$(a_1,b_1,\ldots, a_n,b_n) \mapsto ((\beta_{k_1}, \beta_{k_2},\ldots, \beta_{k_m}),(k_1,k_2,\ldots, k_m))$$
%
%where $m$ is some integer depend on $C_q$ and each integer $k_i$ encode a $2n$ bit string. Moreover  $k_i< k_j$ for all indices $1 \leq i<j \leq n.$ Note  the update function (relation \Cref{nqubit:eq10} ) corresponds to the correction unitary (relation  \Cref{nqubit:eq11} )
%
%\begin{equation}
%\label{nqubit:eq11}
%\begin{aligned}
%\sum_{i=1}^m \beta_{k_i} \xgate^{b_{k_{i1}}} \zgate^{a_{k_{i1}}} \otimes \cdots  \otimes \xgate^{b_{k_{in}}} \zgate^{a_{k_{in}}},
%\end{aligned}
%\end{equation}
%where string $b_{k_{i1}} a_{k_{i1}}\ldots b_{k_{in}} a_{k_{in}}$ is the binary representation of integer $k_i$ for $i\in[m].$ The lemma \Cref{lemma:nqubit} prove the invariant property of the update function (see \Cref{nqubit:update-function}).
%

\begin{lemma}
\label{lemma:nqubit}
Let $C_{q_1}$ and $C_{q_2}$ be two equivalent $n$-qubit circuits  and $k\leq n$
\begin{equation*}
 F_{C_{q_1}}:\{0,1\}^{2n}\longrightarrow  (\mathbb{C} \times {\bf B})^{4^k} \mbox{ and }
  F_{C_{q_2}}:\{0,1\}^{2n}\longrightarrow  (\mathbb{C} \times {\bf B})^{4^k}.
\end{equation*}
are the corresponding update functions for $C_{q_1}$ and $C_{q_2}$  then $F_{C_{q_1}}=F_{C_{q_2}},$ where ${\bf B}=\{0,1\}^{2n}$
\end{lemma}

\noindent{\bf Proof}: Suppose $C_{q_1}$ and $C_{q_2}$ are equivalent circuits and Suppose $F_{C_{q_1}}\neq F_{C_{q_2}},$  then there must exist at least one input string ${\bf s}=a_1b_1,\ldots, a_nb_n$ such that
\begin{equation}
\label{assumption}
F_{C_{q_1}}({\bf s})\neq F_{C_{q_2}}({\bf s})
\end{equation}

Let $C_1$ and $C_2$ be circuits that compute $F_{C_{q_1}}$ and $F_{C_{q_2}}$ respectively, then it follows from relation \ref{assumption} that
\begin{equation}
\label{fun:assumption2}
C_1({\bf s)}=((\beta_1, {\bf s}_1),\ldots, (\beta_n, {\bf s}_n))\neq ((\beta_1^\prime, {\bf s}_1^\prime),\ldots, (\beta_n^\prime, {\bf s}_n^\prime))=C_2({\bf s})
\end{equation}

Suppose we evaluated circuits $C_{q_1}$ and $C_{q_2}$ on some state $\ket \psi$ using gate teleportaton (\Cref{protocol: gate-teleportation}), and further assume that the classical output of the Bell measurement (\Cref{protocol: gate-teleportation}, step 4) for both circuits is the same string ${\bf s}.$ After step 5 of gate teleportation protocol (\Cref{protocol: gate-teleportation}) we are left with the following systems
 \begin{equation*}
C_{q_1} (\xgate^{a_{i}}\zgate^{b_{i}})^{\otimes_{i=1}^{n}}\ket{\psi} \;\mbox{ and }\;
C_{q_2} (\xgate^{a_{i}}\zgate^{b_{i}})^{\otimes_{i=1}^{n}}\ket{\psi}
\end{equation*}

Since $C_{q_1}$ and $C_{q_2}$ are equivalent quantum circuits we have
 \begin{equation}
 \label{nqubit:eq10}
C_{q_1} (\xgate^{a_{i}}\zgate^{b_{i}})^{\otimes_{i=1}^{n}}\ket{\psi}=C_{q_2} (\xgate^{a_{i}}\zgate^{b_{i}})^{\otimes_{i=1}^{n}}\ket{\psi}.
\end{equation}
Using the circuits for the update functions \ref{fun:assumption2} we can rewrite \Cref{nqubit:eq10} as
\begin{equation}
 \label{nqubit:eq11}
\sum_{i=1}^{4^k} \beta_i \xgate^{b_{i_1}} \zgate^{a_{i_1}}\otimes \cdots \otimes \xgate^{b{i_n}} \zgate^{a_{i_n}}C_{q_1}\ket{\psi}=\sum_{i=1}^{4^k} \beta_i \xgate^{b_{i_1}^\prime} \zgate^{a_{i_1}^\prime}\otimes \cdots \otimes \xgate^{b{i_n}^\prime}
\end{equation}

Therefore, we have
\begin{equation}
\label{nqubit:eq12}
\begin{aligned}
\left(\sum_{i=1}^{4^k} \beta_{k_i} \xgate^{b_{k_{i1}}} \zgate^{a_{k_{i1}}} \otimes \cdots  \otimes \xgate^{b_{k_{in}}} \zgate^{a_{k_{in}}}\right)=\left(\sum_{i=1}^{4^k} \beta_{k_i}^\prime \xgate^{b_{k_{i1}^\prime}} \zgate^{a_{k_{i1}}^\prime} \otimes \cdots  \otimes \xgate^{b_{k_{in}}^\prime} \zgate^{a_{k_{in}}^\prime}\right)
\end{aligned}
\end{equation}

\begin{equation}
\label{nqubit:eq13}
\begin{aligned}
\left(\sum_{i=1}^{4^k} \beta_{k_i} \xgate^{b_{k_{i1}}} \zgate^{a_{k_{i1}}} \otimes \cdots  \otimes \xgate^{b_{k_{in}}} \zgate^{a_{k_{in}}}\right)-\left(\sum_{i=1}^{4^k} \beta_{k_i}^\prime \xgate^{b_{k_{i1}^\prime}} \zgate^{a_{k_{i1}}^\prime} \otimes \cdots  \otimes \xgate^{b_{k_{in}}^\prime} \zgate^{a_{k_{in}}^\prime}\right)={\bf 0}
\end{aligned}
\end{equation}
From \cref{nqubit:eq13} it follows that for every $i\in [4^k]$
\begin{equation}
\label{nqubit:eq14}
\beta_{k_i} \xgate^{b_{k_{i1}}} \zgate^{a_{k_{i1}}} \otimes \cdots  \otimes \xgate^{b_{k_{in}}} \zgate^{a_{k_{in}}}=\beta_{k_i}^\prime \xgate^{b_{k_{i1}^\prime}} \zgate^{a_{k_{i1}}^\prime} \otimes \cdots  \otimes \xgate^{b_{k_{in}}^\prime} \zgate^{a_{k_{in}}^\prime}
\end{equation}
From \Cref{nqubit:eq14} we can conclude that $(\beta_i, {\bf s}_i)=(\beta_i^\prime, {\bf s}_i^\prime)$ for every $i\in [4^k],$ therefore we have a contradiction and $F_{C_{q_1}}=F_{C_{q_2}}.$


\subsection{Cost of Computing the Update Function}
We have to show that the update function can be computed by a polynomially size classical circuit. Let $C_q $ be an $n$-qubit circuit  with $k\in O(\log(|C_q|)).$
\begin{equation*}
F_{C_q}(s)=((\beta_1, {\bf s}_1),\ldots, (\beta_k, {\bf s}_k)),\; \mbox{for } s\in\{0,1\}^{2n}.
 \end{equation*}

Each $\beta_i \in poly(|C_q|)$ (appendix \Cref{coeff:size}) and $|{\bf s}_i|=2n,$ but $n\leq |C_q|,$ therefore ${\bf s}_i \in O(|C_q|).$  Therefore the size of each pair $(\beta_i, {\bf s}_i)$ is in $O(poly|C_q|)$ and there are only $\log(|C_q|)$ pairs.
 Moreover,  the size of $\beta_i \in poly(|C_q|)$ (appendix \Cref{coeff:size}). So the size of the function is polynomial. To compute $F_{C_q}$ we go gate by gate in $C_q$ and perform operations such as $\oplus$, swaps, $+$ and $\times$ on operand of size that are bounded by $poly(|C_q|).$ There are $|C_q|$ gates we have to go through and each require at most $poly(|C_q|)$
operations.Therefore, $F_{C_q}$ can be computed with $poly(|C_q|)$ size classical circuit.












