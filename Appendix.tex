\section{Appendix}
\label{sec:appendix}
 \subsection{Indistinguishability Obfuscator for Update Functions}
 \label{sec: iO-clifford-functions}
 In this section we show that how one can easily construct a quantum secure $i\mathcal{O}$ for update function corresponding to Clifford Circuit. Suppose $C_q$ is an $n$-qubit circuit.
\begin{algorithm}[H]
  \caption{$i\mathcal{O}$ for Clifford update Functions $F_{\tt Clifford}$}
  \begin{enumerate}
  \item Compute the canonical form $C_q$ using the algorithm presented in ~\cite{AG04} (section VI). Denote the canonical form as $C_q^\prime.$
  \item Let $g_k, \ldots,g_2,g_1$ be the topological ordering of the gates in $C_q^\prime,$ where $k=|C_q^\prime|.$
  \item Construct the classical circuit $C^\prime$ that computes update function $F_{C_q^\prime}$ as follows. For $i=1$ to $k$ implement update rule (section \ref{update function})  for each gate $g_i$
  \item Output the circuit $C^\prime.$
  \end{enumerate}
\end{algorithm}
%
\noindent Note $C_q^\prime$ reveal no information about $C_q,$ except the functionality. Therefore, $C^\prime$ can reveal no information about about $C_q,$ excepts its functionality. Note $|C^\prime|$ is at most $poly|C_q|.$ Moreover, each step of the $i\mathcal{O}$ can be done in the $poly(|C_q|),$ therefore $i\mathcal{O}$ runs in polynomial-time.



 \subsection{Size of Coefficients in Update Functions}
 \label{coeff:size}
The following map is an isomorphism between $\mathbb{C}$ and $\mathbb{R}^2$
 \begin{equation}
  \label{size:map:real-complex)}
  f:\mathbb{C}\leftarrow \mathbb{R}^2, \; (a+b_i)\mapsto (a,b)
\end{equation}

Therefore, the size of a complex number is simply the size $\tgate$$|a|+|b|.$\footnote{Note the notation $|a|+|b|$ means, sum of the number of bits in $a$ and $b.$}. So we can estimate the size of complex numbers by treating them as an ordered pairs of real numbers. Let $C_q$ be an $n$-qubit circuit and let $k\in O(\log(|C_q|)$ be the number of $\tgate$ gates in $C_q.$ Recall  that the update function for $C_q$ is given by the following map (\Cref{nqubit:eq0}),
\begin{equation}
\begin{aligned}
 F_{C_q}: \{0,1\}^{2n}\longrightarrow  (\mathbb{C} \times {\bf B})^{4^k}
 \end{aligned}
\end{equation}
\begin{equation}
\begin{aligned}
 (a_1,b_1,\ldots, a_n,b_n) \mapsto \left((\beta_1, {\bf s}_1),\dots, (\beta_{4^k}, {\bf s}_{4^k}))\right).
\end{aligned}
\end{equation}
where ${\bf B}=\{0,1\}^{2n}.$ The above map corresponds to the following correction unitary (\Cref{exp:nqubit-correction})

 \begin{equation*}
\sum_{i=1}^{4^k} \beta_i \xgate^{b_{i_1}} \zgate^{a_{i_1}}\otimes \cdots \otimes \xgate^{b{i_n}} \zgate^{a_{i_n}},\;  \; k\leq n.
\end{equation*}

Moreover, the complex coefficients $\beta_i$ are constructed adding and multiplying numbers from the set $\left\{\frac{1+i}{2},\frac{1-i}{2}\right\}.$ We note the following properties these numbers

 $$\left(\frac{1+i}{2}\right)\pm \left(\frac{1-i}{2}\right)=\pm1.$$
$$ \hspace{4cm}\left(\frac{1\pm i}{2}\right)^m=\left(\frac{a}{2}\right)^l, a\in\{\pm 1,\pm i\} \mbox{ and $m$ odd postive integer}$$
 $$\hspace{4cm}\left(\frac{1\pm i}{2}\right)^m=\left(\frac{a}{2}\right)^l \left(\frac{1\pm i}{2}\right), a\in\{\pm 1,\pm i\}\mbox{ and  $m$ even postive integer}$$



 Therefore, we can represent $\left(\frac{1\pm i}{2}\right)^m$ in $O(m)$ bits  and $\left(\frac{1+ i}{2}\right)^{s} \left(\frac{1- i}{2}\right)^{t}$ in $O(s+t)$ bits, for $s,t\in\mathbb{N}.$
 Note that each $\beta_i$ can be written as
 $$\beta_i=\sum_{i=1}^{poly(k)}a_{i_1}a_{i_2}\cdots a_{i_{poly(k)}},\; \mbox{ for } a_{i_j}\in\left\{\frac{1+i}{2},\frac{1-i}{2}\right\}$$
The size of the each term $a_{i_1}a_{i_2}\cdots a_{i_{poly(k)}}$ is at most $O(poly(k))$ and there are $O(poly(k))$ terms. Therefore, the size of each $|\beta_i|\in O(poly(|C_q)|).$


