\section{Introduction}
%\anote{start more smoothly. What is iO, how did it come about, why do we care, what is the state-of-the-art? Also, mention somewhere that a canonical form is an iO}
%\anote{Add more background about the state-of-the art in iO: what are the current candidates, and is there a post-quantum one?}
%\anote{introduce quantum cryptography. Introduce Quantum iO as per \cite{AF16arxiv}.}
At the intuitive level, an \emph{obfuscator} is a probabilistic polynomial-time algorithm that  transforms a circuit $C$ into another circuit $C’$ that has the same functionality as $C$ but does not reveal anything about $C$, except its functionality \emph{i.e.}, anything that can be learned from $C’$ about $C$ can also be learned from  black-box access to the input-output functionality of~$C$. This concept is formalized in terms of \emph{virtual black-box obfuscation}, and was shown~\cite{BGI+12} to be unachievable in general.
%\footnote{The argument consists in a construction of a family of Boolean functions which is inherently unobfuscatable.}.
Motivated by this impossibility result, the same work~\cite{BGI+12} proposed a weaker notion called \emph{indistinguishability obfuscation} 

An \emph{indistinguishability obfuscator} ($i\mathcal{O}$) is a probabilistic polynomial-time algorithm that takes a circuit~$C$ as  input and outputs a circuit $i\mathcal{O}(C)$ such that $i\mathcal{O}(C)(x)=C(x)$ for all inputs~$x$ and the size of $i\mathcal{O}(C)$ is at most polynomial in the size $C$. Moreover, it must be that for any two circuits $C_1$ and $C_2$ of the same size and that compute the same function, the obfuscations of both circuits are computationally indistinguishable. An $i\mathcal{O}$ achieves the notion of \emph{best possible obfuscation}, which states that any information that is not hidden by the obfuscated circuit is also not hidden by any  circuit of similar size computing the same functionality~\cite{GR14}. Indistinguishability obfuscation is a very powerful cryptographic tool which has been shown to enable, among others: digital signatures, public key encryption~\cite{SW14}, multiparty key agreement, broadcast encryption~\cite{BZ14}, fully homomorphic encryption~\cite{CLTV15} and witness-indistinguishable proofs~\cite{BP15}.

The first candidate construction of $i\mathcal{O}$ was published in~\cite{GGH+13}, whose security relied on the presumed hardness of multilinear maps~\cite{CLT13, LSS14, GGH15}. Unfortunately, there have been many quantum attacks on  multilinear maps~\cite{ABD16, CDPR16, CGH17}.  Recently, a new $i\mathcal{O}$ scheme was proposed under  different assumptions \cite{AJL+19}. Whether or not this scheme is resistant against quantum attacks remains to be determined.

In the quantum world, obfuscation has been studied in \cite{AJJ14,AF16arxiv}. In a nutshell (see \Cref{sec:obf-quantum} for more details), \cite{AJJ14} show a type of obfuscation for quantum circuits, but without a security reduction.  On the other hand,  the focus of \cite{AF16arxiv} is on impossibility of obfuscation for quantum circuits in a variety of scenarios. Thus, despite these works, until now, the achievability of indistinguishability obfuscation for quantum circuits has remained wide open.




%this methodology allow us to obfuscate a more general set of quantum circuit than the canonical base construction. Moreover the gate-teleportation methodology could also enable a new functionality related to the \emph{unclonability} of quantum states.

%In this work, we continue the study of $i\mathcal{O}$ in the quantum world. We give two main contributions, both assuming the existence of quantum-secure $q\mathcal{O}$ for classical circuits.
%Our first contribution echoes the private-to-public key transformation of~\cite{SW14}, by showing its applicability to the encryption of \emph{quantum} messages. In this case, the security is based on (quantum-secure one-way functions?)..., and is shown according to the notion of quantum semantic security ~\cite{BJ15,ABF+16}. Our second contribution shows how to build $i\mathcal{O}$ for certain families of quantum circuits. Namely, we show two different ways in which Clifford circuits admit an $i\mathcal{O}$ scheme. The first method is based on the canonical representation of Clifford circuits~\cite{AG04,Nest10}, while the second method is based on the principle of gate teleportation~\cite{GC99}. We  extend these schemes to circuits that include layers of non-Clifford $\tgate$-gates, as long as each Clifford layer is within a family of circuits that admit the $i\mathcal{O}$ obfuscation described above, and as long as the $\tgate$-gates are in fixed positions.
% The two above methods present different advantages: the technique using the canonical form is straightforward and in fact does not require the assumption of a quantum-secure classical $i\mathcal{O}$. By construction, the obfuscated circuits is classical, and hence can be easily communicated, stored, used and copied. In contrast, the gate-teleportation scheme requires the assumption of quantum-secure classical $i\mathcal{O}$. By construction, the obfuscated circuits are quantum. While this presents a technological challenge to communication, storage and also usage, it could enable a new functionality related to the \emph{unclonability} of quantum states (see Open Questions below).

%\Anne{note to Raza: when adding bib entries to quantum.bib, just add a quick entry at the bottom of quantum.bib. Then Sebastien will clean it up and send us the corrected .bib (Sébastien has a small contract as admin of the quantum.bib).}




\subsection{Overview of Results and Techniques}
\label{sec:techniques}

%In this work, we continue the study of computational indistinguishable obfuscation in the quantum world.
Our contribution establishes indistinguishability obfuscation for certain families of quantum circuits.  At a high level, we achieve our constructions via quantum techniques that allow \emph{information-theoretic} reductions to classical indistinguishability obfuscation. We now overview our results and techniques.

%\Anne{I need to check \cite{AG04} again: does their definition include a quantum state, as output by the obfuscator?}
%$Qi\mathcal{O}$
%and show how to construct a $Qi\mathcal{O}$ for certain families of quantum circuits.
First, we define indistinguishability obfuscation for quantum circuits ($Qi\mathcal{O}$)(\Cref{sec:QiO-Cliffords and more}). Importantly, this definition specifies that on input a classical description of a quantum circuit~$C_q$, the obfuscator outputs a \emph{pair} $(\rho, C_q^\prime)$, where $\rho$ is a quantum state and $C_q^\prime$ is a quantum circuit, with the property that
$C_q^\prime(\rho,\ket{\phi})=C_{q}(\ket{\phi})$ for all inputs $\ket{\phi}$.
%\Anne{I am currently thinking that we could let the quantum iO output a pure state, \emph{e.g.}, $\ket{\phi}$, instead of a mixed state $\rho$. This could simplify things later. But I don't know if this is general enough.}
 
In terms of constructing $Qi\mathcal{O}$, we first focus on the family of \emph{Clifford} circuits and show two methods of obfuscation: one straightforward method based on the canonical representation of Cliffords, and another based on the
% is based on the canonical representation of Clifford circuits~\cite{AG04}, while the second construction is based on the
principle of gate teleportation~\cite{GC99}.
Clifford circuits are quantum circuits that are built from the gateset $\{\xgate, \zgate, \pgate, \cnot, \hgate\}$ . They are known not to be universal for quantum computation and  are, in a certain sense, the quantum equivalent of classical \emph{linear circuits}. It is known that Clifford circuits can be efficiently simulated on a classical computer~\cite{Got98}; however, note that  this simulation is with respect to a \emph{classical} distribution, hence for a purely quantum computation, quantum circuits are required, which motivates the obfuscation of this circuit class. Next,  Clifford circuits are known not to be universal for quantum computation, but they are an important building block for fault-tolerant quantum computing, for instance, due to the fact that Cliffords admit transversal computations on many fault-tolerant codes.

\paragraph{Obfuscating Cliffords using a canonical form.} Our first construction of $Qi\mathcal{O}$ for Clifford circuits starts with the well-known fact that a canonical form is an $i\mathcal{O}$. We point out that a canonical form for Clifford circuit was presented in~\cite{AG04}; which completes this construction (we also note that an alternative canonical form was also presented in~\cite{Sel13arxiv}). This canonical form technique  does not require any assumptions. Moreover, the obfuscated circuits are classical, and hence can be easily communicated, stored, used and copied.

\paragraph{Obfuscating Cliffords using gate teleportation.} Our second construction of $Qi\mathcal{O}$ for Clifford circuits takes a very different approach. We start with the gate-teleportation scheme~\cite{GC99}: according to this, it is possible to \emph{encode} a quantum computation~$C$ into a quantum state (specifically, by preparing a collection of entangled qubit pairs, and applying~$C$ to half of this preparation). Then, in order to perform a quantum computation on a target input $\ket{\psi}$, we \emph{teleport} $\ket{\psi}$ \emph{into} the prepared entangled state. This causes the state $\ket{\psi}$ to undergo the evolution of $C$, \emph{up to some corrections}, based on the teleportation bits. If $C$ is chosen from the Clifford circuits, these corrections are relatively simple and thus the idea is to use a classical  $i\mathcal{O}$ to provide the correction function (note that our construction does not require a full $i\mathcal{O}$ but rather relies on a weaker $i\mathcal{O}$ that can obfuscate a certain family of functions.)
In contrast to the previous scheme, the gate-teleportation scheme requires the assumption of quantum-secure classical $i\mathcal{O}$ for a certain family of functions (section \ref{update function}) and the obfuscated circuits are quantum. While this presents a technological challenge to communication, storage and also usage, there could be advantages to storing quantum programs into quantum states, for instance to take advantage of their \emph{uncloneability} \cite{Aar09,BL19arxiv}.



\paragraph{Obfuscating Beyond Cliffords} Next, in our main result (\Cref{QiO:Clifford+T:family}), we generalize the gate-teleportation scheme for Clifford circuits, and show a $Qi\mathcal{O}$ obfuscator for all quantum circuits where the number of non-Clifford  gates is at most logarithmic in the circuit size. For this, we consider the commonly-used Clifford+$\tgate$ gateset, and we note that the $\tgate$ relates to the $\xgate, \zgate$ as: $\tgate \xgate^b \zgate^a=\xgate^b \zgate^{a\oplus b}\pgate^b \tgate$. This means that, if we implement a circuit $C$ with $\tgate$ gates as in the gate-teleportation scheme above, then the \emph{correction} function is no longer a simple Pauli update (as in the case for Cliffords). However, this is only partially true: since the Paulis form a basis, there is always a way to represent an update as a complex, linear combination of Pauli matrices. In particular, for the case of a $\tgate$, we note that $\pgate=(\frac{1+i}{2}) \igate + (\frac{1-i}{2})\zgate$. Hence, it \emph{is} possible to produce an update function for general quantum circuits that are encoded via gate teleportation. To illustration this, we first analyze the case of a general Clifford+$\tgate$ quantum circuit on a \emph{single} qubit (\Cref{sec:1-qubit}). Here, we are able to provide $Qi\mathcal{O}$ for all circuits. Next, for general quantum circuits, (\Cref{sec:n-qubit-circuits}), we note that the update function becomes more and more complex as the number of $\tgate$ gates increases. We show that if we limit the number of $\tgate$ gates to be logarithmic in the circuit size, we can nevertheless reach an efficient construction.


We note that,  for many other quantum cryptographic primitives, it is the case that the $\tgate$ gate the bottleneck (somewhat akin to a \emph{multiplication} in the classical case. This has been observed, \emph{e.g.}, in the context of \emph{homomorphic quantum encryption} \cite{BJ15}, instantaneous quantum computation \cite{TODO} and also.... TODO. Because of these applications, and since the $\tgate$ is also typically the bottle neck for fault-tolerant quantum computing, techniques exist to reduce the $\tgate$ count in quantum circuits~\cite{AMMR13,DMM16,AMM14}.



%In section \ref{sec:QiO-Cliffords and more} we provide a new definition of quantum indistinguishability obfuscation. We provide two methodologies for constructing a quantum indistinguishability obfuscation for Clifford circuits. One is based on canonical form of Clifford circuits  while other is based on the principle of gate teleportation.  The construction based on canonical form requires no computational assumptions and is in fact perfectly indistinguishability obfuscation (information theoratically secure), where as the gate teleportation based construction assumes the existence of a quantum secure classical $i\mathcal{O}.$ In section \ref{QiO:Clifford+T:family} we show how to a construct quantum indistinguishability obfuscation from gate teleportation and $i\mathcal{O}$ for any reversible quantum circuit that has at most logarithmic $\tgate$-gates.






%In this work we define a new definition of computational quantum obfuscation $Qi\mathcal{O}$ and provide two concrete instantiation of it. The first instantiation is based on the work of Aaronson and Gottesman~\cite{AG04}, In this paper, an efficient algorithm to compute a canonical form of any Clifford circuit was describe. The second instantiation is based on gate-teleportation \cite{} and assume an existence of a quantum-secure classical $i\mathcal{O}$ for certain class of functions (section \Cref{update function}). Using this technique we can obfuscate any $n$-qubit quantum circuit as far as the number of $\tgate$-gates are in $O(\log(n)).$






%Quantum indistinguishability  for the encryption of quantum messages was first defined in~\cite{BJ15}, and was shown to be equivalent to quantum semantic security~\cite{ABF+16}. In terms of concrete schemes, \cite{BJ15} showed... while, \cite{ABF+16} showed...
%In contrast, our contribution using $i\mathcal{O}$ shows.....

%To the best of our knowledge, our second contribution is the first that describes how to achieve $i\mathcal{O}$ for some families of quantum circuits
%(however, as noted above, the Clifford canonical form is well-known, and its extension to $i\mathcal{O}$ (see Section~\ref{sec:Clifford-$i\mathcal{O}$-canonical}) is relatively straightforward.

\subsection{More on Related Work}
\label{sec:obf-quantum}
Quantum obfuscation was first studied in \cite{AJJ14}, where a notion called
$(G,\Gamma)$-{\em indistinguishability obfuscation} was proposed,  where $G$ is a set of gates and $\Gamma$ is a set of relations satisfied by the elements of $G$. In this notion, any two circuits over the set of gates $G$ are perfectly indistinguishable if they differ by some sequence of applications of the relations in~$\Gamma.$ One of the motivations of their work was to provide a weaker definition of perfectly indistinguishable obfuscation, which is shown to be impossible under certain complexity-theoretic assumptions \cite{AJJ14}. However, to the best of our knowledge, $(G,\Gamma)$-{\em indistinguishability obfuscation} is  incomparable with  computational indistinguishability obfuscation \cite{BGI+12, GGH+13}, which is the topic of our work.

Quantum obfuscation is studied rigorously in \cite{AF16arxiv}, where the various notions of quantum obfuscation are defined (including quantum black-box obfuscation, quantum indistinguishability obfuscation, and quantum Best-Possible obfuscation. 
%
 The main contribution of their work is to extend the classical impossibility results to the quantum setting.
 %such as a generic transformation of quantum circuits into black-box-obfuscated quantum circuits is impossible \cite{AF16arxiv},  statistical indistinguishability obfuscation is impossible, up to an unlikely complexity-theoretic collapse \cite{AF16arxiv}. 
 However, no concrete instantiation was provided in their paper for any of type quantum obfuscations. 
 %They also discussed a number of applications of quantum black-box obfuscation such as CPA-secure quantum encryption, quantum fully-homomorphic encryption, and public-key quantum money, however it is not clear what impact these impossibility results have on these applications.
  We note that  \cite{AF16arxiv} shows that the existence of a computational quantum indistinguishability obfuscation implies a witness encryption scheme for all languages in \textsf{QMA}.


%%%%%%%%%%%%%%%%%%%%%%%%%%%%%%%%%%%%%%%%%%%%%%%%%%%%%
%%Section applications
%%%%%%%%%%%%%%%%%%%%%%%%%%%%
%    \subsection{Applications}
%Two settings
%
%1) If we had full iO.
%
%2) Applications of iO for the circuit class that we are doing.


%%%%%%%%%%%%%%%%%%%%%%%%%%%%%%%%%%%%%%%%%%%%%%%%%%%%
%%Section Conclusion and Open Problems
%%%%%%%%%%%%%%%%%%%%%%%%%%%%
%\subsection{Conclusion and Open Questions}
%
%The main open questions related to this work are:
%
%$i\mathcal{O}$ obfuscation for general quantum circuits
%
%applications of gate-teleportation based quantum $i\mathcal{O}$ (for instance, to unclonable programs~\cite{Aar09}).

%
\subsection*{Outline}
%\Cref{sec: iO-clifford-functions}...
The remainder of this paper is structured as follows. \Cref{sec:prelims} overviews basic notions required in this work. In \Cref{sec:QiO-Cliffords and more} we formally define indistinguishability obfuscation for quantum circuits and provide the construction for Clifford circuits. Finally, in \Cref{QiO:Clifford+T:family}, we give our main result which shows quantum indistinguishability obfuscation for quantum circuits with a logarithmic number of $\tgate$ gates. 